\input{header.tex}

%\setlength{\headsep}{-10pt}
\setlength{\parskip}{0.2em}
%\setlength{\textheight}{11 in}
\setlength{\skip\footins}{20pt}

% opening
\title{Plotting large datasets, Part 1: Static plots}
\author{Colin Leach, March 2020}
\date{\vspace{-3ex}}

\hyphenpenalty=1000

\begin{document}
	
\maketitle

Galaxy simulations, even undergraduate projects, can involve upwards of $10^7$ particles. Tough to plot, so what are some options?

I made these notes for my own use but there may (?) be other students who are interested.

The dataset used is an array with shape (3, 975000), so 975,000 sets of $(x,y,z)$ coordinates, at a time when two galaxies are tidally disrupting one another.

\section{Matplotlib hist2d}

Covered in Lab 7 of ASTR400B, which also adds density contours. That only dealt with single disks but extending it to collisions is relatively simple. Think centering, orientation and (perhaps hardest!) what $x$ and $y$ limits to set.

The result may look something like this:

{\centering \includegraphics[scale=0.5]{mpl contours, 400b code} \par}

\section{Matplotlib plus mpl\_scatter\_density}

This isn't a new problem, so someone wrote an extension to matplotlib to make density plots easier.

Website: \url{https://github.com/astrofrog/mpl-scatter-density}

Borrowing heavily from their examples, this is a first attempt:

\lstdefinestyle{py}{
	belowcaptionskip=1\baselineskip,
	breaklines=true,
	frame=L,
	xleftmargin=\parindent,
	language=Python,
	showstringspaces=false,
	basicstyle=\footnotesize\ttfamily,
	keywordstyle=\bfseries\color{green!40!black},
	commentstyle=\itshape\color{purple!40!black},
	identifierstyle=\color{blue},
	stringstyle=\color{orange},
}

\lstset{style=py} 
\begin{lstlisting}
# install with `conda install mpl-scatter-density`
import mpl_scatter_density

# Make the norm object to define the image stretch
from astropy.visualization import LogStretch
from astropy.visualization.mpl_normalize import ImageNormalize
norm = ImageNormalize(vmin=0., vmax=1000, stretch=LogStretch())

# Make the plot in the (almost) usual pyplot way
# Note the projection to invoke mpl_scatter_density
#  and the norm to use LogStretch

fig = plt.figure(figsize=(12,10))
ax = fig.add_subplot(1, 1, 1, projection='scatter_density')
density = ax.scatter_density(disks[0], disks[1], norm=norm)
plt.xlim(-120, 120)
plt.ylim(-120, 120)

# Add axis labels (standard pyplot)
fontsize = 18
plt.xlabel('x (kpc)', fontsize=fontsize)
plt.ylabel('y (kpc)', fontsize=fontsize)
fig.colorbar(density, label='Number of stars per pixel')
\end{lstlisting}

{\centering \includegraphics[scale=0.55]{msg} \par}

The colorbar needs some work, but we now get pixel-level resolution. It's a bit easier to see the tidal tails, at least on screen (printouts can be disappointing).

First impressions: a nice addition to matplotlib, quick and easy to set up and use.

\end{document}
