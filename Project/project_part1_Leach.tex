\documentclass[twocolumn]{aastex63}

\usepackage{natbib}
\usepackage{booktabs} % helps with pandas to_latex() output
\usepackage{multirow}
\usepackage{amsmath}

\graphicspath{{img/}}

\newcommand{\todo}{\color{red}{TODO}\color{black}\hspace{2mm}}

\shorttitle{MW-M31 Interactions}
\shortauthors{Colin Leach}

\begin{document}
	
\title{Future Dynamics of the Local Group. I. MW-M31 Interactions}

\author[0000-0003-3608-1546]{Colin Leach}

\begin{abstract}
	
\todo{Add a concise and intelligent summary of the paper, once I get a clearer idea what it will include (and how to seem intelligent).}
	
\end{abstract}

\keywords{Galaxy Merger -- Local Group -- Stellar Disk -- Stellar Bulge -- \textit{and more}}

%\tableofcontents


\section{Introduction}

The currently-accepted model of galaxy formation involves baryonic matter (gas and dust) falling into gravitational potential wells created by local over-densities in the Dark Matter (DM). Further gravitational collapse and Jeans fragmentation can then lead to creation of galaxies and stars. \todo{refs?}

However, decades of observational and theoretical studies tell us that, firstly, this does not account for the wide range of galaxy morphology seen at all epochs; secondly, there is no reason to suppose that galaxies continue in serene isolation after their formation.

Attempts to model interactions and mergers between galaxies with numerical simulations goes back at least to \citet{toomre_galactic_1972}. This field continues to develop, with improvements in both hardware and algorithms allowing larger particle numbers in N-body simulations  \todo{ref?} and more sophisticated treatment of gas hydrodyamics, magnetic fields and other factors.

As with all theoretical studies, it is vital to stay connected to the best experimental data as this constantly evolves, constantly comparing models against observations. Checking simulations against high-redshift galaxies is useful but inevitably approximate. A perhaps more rigorous test is to model the galaxies for which we have the most precise and detailed observational measurements: those which are (by far) closest to us. 

The largest galaxies in our Local Group (LG) are the Milky Way (MW), Andromeda (M31) and Triangulum (M33).  A simulation of MW--M31--M33 orbital evolution was described previously in \citet{marel_m31_2012}, hereafter vdM12. That paper included an extensive analysis of both N-body simulations and semi-analytic orbit integrations. The present study uses data from the same N-body simulation to carry out further computational analysis.

The simulation was based on data in \citep{marel_m31_2012-1} suggesting that M31 is approaching the MW directly with little proper motion detected by Hubble Space Telescope (HST) studies. Recent data from Gaia DR2 \citep{brown_gaia_2018} suggest that infall is slightly less radial than previously thought \citep{marel_first_2019}, leading to a slightly later first approach with a larger pericenter distance. However, detailed simulations based on that new data have not yet been carried out.

This paper will review the initial conditions and time evolution for multiple physical parameters of the simulation. Particular attention will be paid to the first MW-M31 close approach around 4 Gyr, the second approach and merger around 6 Gyr, and the structure and dynamics of the post-merger remnant.

Time probably precludes much analysis of the fate of M33, which will need to be the subject of a future paper.

\subsection{Data}

Data from one N-body simulation in vdM12 was supplied in text-file format by one of the original authors. This included position and velocity data for each particle at the current epoch ($t=0$) and 800 future time steps. For ease of analysis, this was all transferred to the open source database PostgreSQL\footnote{\url{http://www.postgresql.org}} (approximately 1.35 billion records). The same database was used to store computed summary data during the analysis.

\begin{deluxetable}{crrrr}[htb!]
	\tablecaption{Particle counts
		\label{tbl:counts}}
	\tablewidth{0pt}
	\tablehead{
		\colhead{Galaxy} & \colhead{DM Halo} & \colhead{Disk} & \colhead{Bulge} & \colhead{Total} 
		%\colhead{Name} & \colhead{Count} & \colhead{Count} & \colhead{Count} & \colhead{}
	}
	\startdata
	MW   &  250,000 &   375,000 &    50,000 &   675,000 \\
	M31  &  250,000 &   600,000 &    95,000 &   945,000 \\
	M33  &   25,000 &    46,500 &        0 &    71,500 \\
	\midrule
	LG  &  525,000 &  1,021,500 &   145,000 &  1,691,500
	\enddata
\end{deluxetable}

Particle counts for each time point are shown in Table \ref{tbl:counts} and total masses in Table \ref{tbl:aggmass}. We can see that total mass is the same for MW/M31 but our galaxy has more luminous stars (higher baryon fraction) and M31 has more dark matter (lower baryon fraction). M33 is about 10-fold lighter than either.

\begin{deluxetable}{crrrr}[htb!]
	\tablecaption{Aggregate masses ($M_\Sun \times 10^{12}$)
	\label{tbl:aggmass}}
	\tablewidth{0pt}
	\tablehead{
		\colhead{Galaxy} & \colhead{DM Halo} & \colhead{Disk} & \colhead{Bulge} & \colhead{Total}
	}
	\startdata
	MW  &      1.975 &      0.075 &       0.010 &  2.060 \\
	M31 &      1.921 &      0.120 &       0.019 &  2.060  \\
	M33 &      0.187 &      0.009 &       0.000 &  0.196  \\
	\midrule
	LG &   4.082 &      0.204 &       0.029 &  4.316
	\enddata
\end{deluxetable}

The coordinate system is approximately centered on the Milky Way at $t=0$. The center of mass (CoM) of all particles in the system is not fixed over time, moving at an average of $\left< 35.9, -26.7, 27.5 \right>$ km/s with some minor fluctuations due to numerical approximations. In contrast, the total angular momentum of the system is very small at all time points.

\subsection{Software}

The work in this report was carried out in Python using standard packages. Full details are available online\footnote{Code \url{https://github.com/colinleach/400B_Leach}\\documentation \url{https://400b-leach.readthedocs.io}}

\section{Results}

\subsection{Trajectories}

The simulation does not explicitly include a supermassive black hole (SMBH) at the center of each galaxy, but the galactic center was defined by calculating the center of mass (CoM) of the disk particles and iteratively constraining the radius of interest until convergence.

To plot motions of the three galactic CoMs it is convenient to transform to a coordinate system in which at $t=0$ they all lie in the $x,y$ plane with MW and M31 on the $x$-axis. The overall CoM is moving, as noted above, so at each time point the coordinates are translated to center it at the origin.

\begin{figure}[htb!]
	\plotone{traj_xy.pdf}
	\caption{Trajectories of each galactic center of mass in the X',Y' plane. Points are at 71 Myr intervals.
		\label{fig:traj_xy}}
\end{figure}

\begin{figure}[htb!]
	\epsscale{0.9}
	\plotone{traj_z.pdf}
	\caption{Trajectories of each galactic center of mass perpendicular to the X',Y' plane.
		\label{fig:traj_z}}
\end{figure}

\begin{figure}[hbt!]
	\epsscale{0.9}
	\plotone{rel_sep.pdf}
	\caption{Separations of galactic CoMs.
		\label{fig:rel_sep}}
\end{figure}

\begin{figure}[hbt!]
	\epsscale{0.9}
	\plotone{rel_vel.pdf}
	\caption{Relative velocities of galactic CoMs.
		\label{fig:rel_vel}}
\end{figure}


In vdM12 this is referred to as the X',Y',Z' coordinate system and their figure 2 shows multiple views of how the galaxies move through time. In this paper, Figures \ref{fig:traj_xy} and  \ref{fig:traj_z} show some alternative views in essentially the same coordinates (up to a sign; the $x$ and $z$ axes are flipped). Figure \ref{fig:traj_xy} reproduces the top left panel of vdM12. Figure \ref{fig:traj_z} shows that MW and M31 remain close to the starting plane while M33 has larger, irregular out-of-plane motions.

Relative motions of the CoMs are shown against time in Figures \ref{fig:rel_sep} and \ref{fig:rel_vel}, equivalent to figures 3 and 4 in vdM12.


There is a MW-M31 close approach with first pericenter at 3.96 Gyr with a minimum separation of 35.1 kpc, then a separation to 173 kpc and finally a convergence to 7.8 kpc at second pericenter and merger between 5.9 - 6.5 Gyr. Relative velocities spike sharply during these approaches, as potential energy is converted to kinetic energy, before declining to essentially zero.

Meanwhile, in this simulation run M33 remains separate throughout, albeit on a decaying orbit. In vdM12 the authors investigate the effect of small changes in initial conditions and estimate a 9\% chance of an M33-MW collision at first pericenter, before the M31-MW merger.

\subsection{Mass profiles and rotation curves}

Figure \ref{fig:massprof0} shows the cumulative mass profile, by particle type and in total, for each galaxy. The center of each galaxy is dominated by baryonic matter with the DM halo becoming dominant at larger radii.
\begin{figure}[bht!]
	\epsscale{1.2}
	\plotone{massprof_000}
	\caption{Mass profiles for each galaxy at the current epoch.
		\label{fig:massprof0}}
\end{figure}

Figure \ref{fig:rotcurve0} shows the rotation curves expected from these mass profiles. Without the DM halo the circular velocity would peak within a few kpc of the CoM then fall steadily at larger radii. With the more diffuse DM halo added, we see the relatively flat overall rotation curves which attracted the attention of astronomers including \citet{zwicky_rotverschiebung_1933} and \citet{rubin_rotation_1970}

\begin{figure}[bht!]
	\epsscale{1.2}
	\plotone{rotcurve_000}
	\caption{Rotation curves for each galaxy at the current epoch.
		\label{fig:rotcurve0}}
\end{figure}

\todo{set xlim, make y axis log}\ 

\subsection{Disk particles}

\subsubsection{Structure}

\todo{identify the bar?}\ 

\todo{more on spiral arms}

\subsubsection{Inclinations}

Galactic disks have a well-defined angular momentum vector which is relatively easy to calculate in this type of simulation. Figure \ref{fig:inclinations_xy} shows the angle each makes to the X'-Y' plane over time.

\begin{figure}[ht!]
	\plotone{inclinations_xy.pdf}
	\caption{Angular momentum inclination angle to the X',Y' plane for each set of galactic disk particles.
		\label{fig:inclinations_xy}}
\end{figure}

The mutual angle between galactic disks can have a significant impact on how tidal disruption and merger dynamics play out \todo{ref?}. This can be calculated from the vector dot products:

\[ \theta = \arccos (\hat{L}_1 \cdot \hat{L}_2 ) \]

Results for the MW-M31 and M33-M31 pairs are shown in Figure \ref{fig:inclinations_mutual}. For MW-M31, the angle is largely stable until near first pericenter, when tidal forces bring the two disks closer to alignment. This trend continues slowly until near second pericenter. Surprisingly, the angle increases after merger, suggesting some partitioning of particles of different origin within the remnant. \todo{CHECK THIS AGAIN!}

\begin{figure}[ht!]
	\plotone{inclinations_mutual.pdf}
	\caption{Angular momentum angles between pairs of galaxies.
		\label{fig:inclinations_mutual}}
\end{figure}

The large variations in M33-M31 angles are indicative of the extensive tidal disruption of the much smaller M33 galaxy. Details are outside the scope of the present paper.

\subsubsection{Velocity dispersion}

The changes in velocity dispersion of disk particles originating from each galaxy are shown in Figure \ref{fig:vel_disp}. The small periodic oscillation seen from the start, especially in M31, appears to be caused by deviations from radial symmetry in the disk: spiral arms and an increasingly prominent bar. Small MW spikes at initial pericenter (around 4 Gyr) and much larger ones at merger (around 6 Gyr) are clearly visible.

M33 is on an irregular, elliptical and decaying orbit about the MW-M31 merger remnant after about 6.5 Gyr. Velocity dispersion appears to peak at intervals. This perhaps corresponds to successive pericenters when M33 experiences maximal tidal disruption, but this will need further analysis.

\begin{figure}[htb!]
	\epsscale{1.1}
	\plotone{vel_disp.pdf}
	\caption{Velocity dispersion of disk particles from each galaxy over time.
		\label{fig:vel_disp}}
\end{figure}

\subsection{Galactic Bulge}

A bulge is present in the MW and M31 but not M33. This region of generally older stars extends further above and below the central plane than disk stars. Kinematics of the bulge are more typical of an elliptical galaxy than a spiral disk.

In a study of elliptical galaxies, de Vaucouleurs showed that surface brightness falls off exponential with radius and approximately as the one-fourth power of radius \citep{de_vaucouleurs_recherches_1948}. Later work found that this was too restrictive for a wider population of galaxies, so Sérsic generalized the formula to have the inverse exponential $n$ as an additional free parameter \citep{sersic_influence_1963}:

\[ \log_{10} \left( \frac{I(r)}{I_e} \right) =  -3.3307 \left[ \left( \frac{r}{R_e} \right)^{1/n} - 1 \right] \]

Here $R_e$ is the radius with which half the light is emitted,  $I_e$ is the surface brightness at $R_e$ and $n$ is the Sérsic parameter.

This formula is intended for analyzing photographic images and is in terms of light intensity. We have no brightness data in the current simulation, but for systems with few young blue stars we can assume the mass to light ratio $M/L \sim 1$. This is probably a reasonable approximation for undisturbed bulges and for an elliptical merger remnant long after the collision. $R_e$ is then the radius enclosing half the mass.

\begin{figure}[bht!]
	\epsscale{1}
	\plotone{sersic_Re}
	\caption{Half-mass radius for bulge particles.
		\label{fig:sersic_Re}}
\end{figure}

We can see from Figure \ref{fig:sersic_Re} that for each galaxy the bulge half-mass radius is fairly stable up to the collision and merger of MW and M31. After a period of disturbance, they again become stable at a higher level. The M31 bulge is more diffuse than the MW bulge throughout, and the ex-bulge stars are clearly not randomized in the merger remnant: ex-M31 stars tend towards larger radii than ex-MW stars.

\begin{figure}[bht!]
	\epsscale{1}
	\plotone{sersic_n}
	\caption{Sersic $n$ for bulge particles, with $1\sigma$ error bars.
		\label{fig:sersic_n}}
\end{figure}

The Sérsic parameter $n$ was estimated by a nonlinear least squares fit to the bulge mass profile. As shown in Figure \ref{fig:sersic_n} it is fairly constant around 5.5 for any period with meaningful data. The spikes around 6 Gyr should probably be ignored: many values in this collision period are missing, as the least-squares fit failed, and the available data has substantially larger error bars than during stable epochs.

The larger half-mass radius of M31 is reflected in the mass density profile, as shown in Figure \ref{fig:bulge_mp}. MW bulge stars have a higher central peak, M31 bulge stars are more numerous at larger radii. This is true both early in the simulation, and in the merger remnant at late times. For both galaxy bulge stars, the central peak is less pronounced post-merger.

\begin{figure}[bht!]
	\epsscale{0.82}
	\plotone{bulge_mp}
	\caption{Bulge mass density profile for both galaxies at the beginning and end of the simulation.
		\label{fig:bulge_mp}}
\end{figure}

The Sérsic fit for both galaxy bulges looks reasonable outside the central density peak, as shown for the MW in Figure \ref{fig:MW_bulge_sersic}. The plot for M31 (not included here) is very  similar.

\begin{figure}[bht!]
	\epsscale{0.8}
	\plotone{MW_bulge_sersic}
	\caption{MW bulge mass density profiles and Sérsic best fits. Time points are the beginning, the pre-merger pericenter, and the end of the simulation
		\label{fig:MW_bulge_sersic}}
\end{figure}

\subsection{Dark Matter halo}

Figure \ref{fig:rotcurve0} also adds a theoretical curve in which the DM halo is fitted by a Hernquist profile \citep{hernquist_analytical_1990}. The cumulative mass out to radius $r$ is given by
\[ M(r) = M_h \frac{r^2}{(a+r)^2} \]
where $M_h$ is the total mass of halo particles (see Table 2) and $a$ is a scale radius. Non-linear least squares fitting, similar to that used for Sérsic profiles in a previous section, gave scale radii of 61.1 kpc for MW and M31, 24.3 kpc for M33 at $t=0$.

\begin{figure}[htb!]
	%	\epsscale{0.8}
	\plotone{hernquist_a.pdf}
	\caption{Hernquist scale radius $a$ for DM halo particles originating from each galaxy, with $1\sigma$ error bars.
		\label{fig:hernquist_a}}
\end{figure}

Time evolution of the scale radius $a$ is shown in Figure \ref{fig:hernquist_a}. The MW and M31 remain very similar through first pericenter, then start to diverge with MW particles tending to a larger radius than M31. This becomes most pronounced during and after merger. The dissimilar distribution in the merger remnant will be discussed in a later section.  

The scale radius for M33 grows inexorably as the original halo is scattered by tidal forces. Figure \ref{fig:hernquist_a} also shows the increasingly wide error bars for M33: halo particles for this galaxy are no longer well fitted by a Hernquist profile.

\subsection{MW-M31 Close approach}

\subsubsection{Inclinations}

The MW and M31 disks have angular momentum vectors inclined at an angle of $52^\circ$ to each other shortly before pericenter, making this a prograde approach.



\subsubsection{Tidal tails and bridges}

%Galaxy Merger Sequence: MW and M31 tidal tail evolution and/or evolution of stellar
%bar Sean Cunningham, Steven Zhou-Wright
%1. How can you identify Tidal tails and bridges throughout the MW-M31 interaction
%sequence?
%2. How can you identify the bar? or pseudo-bulge (disk stars kicked up in the x-shape
%pattern by the bar) ? How do those structures evolve?
%3. Where do the tidal tails come from? Can you select the tail and trace them back
%to the undisturbed systems?
%4. What are the kinematics of the tidal tails over time? Do they change in velocity
%dispersion and energy?
%5. What is the morphological change of the tidal tails over time? Do they grow in
%size?
%6. are the tidal tails unbound? Do the tails return to their original galaxies?
%7. How long lived are the tidal tails? For how long might we observe the system with
%extended tails? What does this mean for our ability to identify merging galaxies?
%8. What is the mass transfer between the two galaxies? Do they exchange material?
%If so, where does this exchanged material end up? Does it rotate in the plane of
%the disk? What is the mass exchange between the MW nad M31 over time?
%59. What is the structure of the tidal tails? Are there any clumps or is it smooth?
%Toomre A., Toomre J. 1972, ApJ, 178, 623
%Barnes+2004 MNRAS 350 (model of an example major merger: Mice . Also look
%up ”antennae galaxies”)
%Privon+2013, ApJ 771
%Ji et al. 2014 A&A 566

The presence of long, symmetrical tails giving some galaxies a distinct `S'-shape has been described at least as far back as \citet{zwicky_novel_1955}. Some astronomers postulated that these were the result of tidal forces during close, glancing encounters, but this was often contested until a detailed computational study by \citet{toomre_galactic_1972}.

Reviewing a broad range of N-body simulations, \citet{barnes_dynamics_1992} noted that ``such features are clearly \textit{relics} of recent collisions rather than ongoing interactions.'' In our simulation, both MW and M31 disks remain near-circular during much of the close approach, but conspicuous tails develop as the centers then move further apart. We also see a more sparsely-populated bridge forming between the galaxies.

To determine the nature and origin of stars in this region, a manual selection was performed as in Figure \ref{fig:bridge}. Stars within the yellow rectangle (left panel) are shown with velocity vectors (center panel) and origin (right panel). Velocities are mostly moderate (mean 195 km/s, range 19-586 km/s), with relatively few stars having high kinetic energy. 


\begin{figure}[ht!]
	\epsscale{1.12}
	\plotone{selected.pdf}
	\caption{Manual selection of bridge particles at 0.33 Gyr after the first MW-M31 pericenter. The left panel shows stellar surface density and the selected region. The center panel shows velocity vectors for these stars and the right panel shows origin by galaxy and particle type. Orientation is with MW top, M31 bottom and M33 lower left. \todo{\textit{make this page-width in final layout}}
	\label{fig:bridge}}
\end{figure}

It appears from the right panel that stars in the tail regions originate in the corresponding disk. The bridge region is more mixed and appears to have a high proportion of former bulge stars. To study this further the coordinate system was transformed to place the large galaxy CoMs on the $x$-axis at $\pm 64$ kpc, as in Figure \ref{fig:bridge2}. It is clear in this view that one MW tail is oriented approximately towards the center of M31. 

\begin{figure}[htb!]
	\epsscale{1.1}
	\plotone{density_rot_300.pdf}
	\caption{View along the midplane between the galactic centers, MW on the left.
		\label{fig:bridge2}}
\end{figure}

\begin{deluxetable}{crrr}[ht!]
	\tablenum{3}
	\tablecaption{Particle counts close to the midplane 
		\label{tbl:pcounts}}
	\tablewidth{0pt}
	\tablehead{
		\colhead{} & \colhead{Bulge} & \colhead{Disk} & \colhead{Total}
	}
	\startdata
	MW      &    305 &  1317 &  1622 \\
	M31     &   1137 &     4 &  1141 \\
	\midrule
	Total     &   1442 &  1321 &  2763 \\
	\enddata
\end{deluxetable}

The different orientations mean that symmetry about the midplane is imperfect, so the ``bridge'' region was taken as $-20 < x < 30$ kpc. A count of stars in this region is shown in Table \ref{tbl:pcounts}. This confirms that the largest populations are MW disk stars (mostly in a relatively dense tail) and M31 bulge stars (more widely dispersed).

\todo{identify, trace history, trace fate}\ 

\todo{Jacobi radius}\ 

\subsubsection{Mass transfer}

Stars are scattered from galaxies even in normal times, and this can be expected to increase significantly during near-misses and collisions. To get a first impression of how many stars and DM particles may end up closer to a different galaxy, we looked at the relative distances of each particle to each of the three galaxy CoMs. It should be emphasized that kinematics is not considered at this stage, so nothing can be said about which particles are gravitationally bound.

\begin{figure}[htb!]
	\epsscale{0.9}
	\plotone{transfer_counts_mpl}
	\caption{Particles closer to a different CoM.
		\label{fig:trans_count}}
\end{figure}

Figure \ref{fig:trans_count} shows that some particles are far from their notional galaxy even at the start. This increases somewhat during first pericenter around 4 Gyr, then jumps permanently during the second pericenter and merger. The plot cuts off at 7 Gyr because it becomes meaningless to consider the MW/M31 CoMs as separate points post-merger.

\begin{figure}[htb!]
	\epsscale{1.1}
	\plotone{transfer_ptype}
	\caption{Particles closer to a different CoM.
		\label{fig:trans_p}}
\end{figure}

Figure \ref{fig:trans_p} looks at a few timepoints by particle type, showing that the overwhelming majority of these particles are from the DM halo. This is unremarkable, given the prevalence of these particles at large radii and their correspondingly weak gravitational binding.

\begin{figure}[htb!]
	\epsscale{1.1}
	\plotone{transfer_lum}
	\caption{Luminous particles closer to a different CoM (DM halo hidden).
		\label{fig:trans_l}}
\end{figure}

To focus on the baryonic matter, Figure \ref{fig:trans_l} hides the DM halo and expands the $y$-axis to show only bulge and disk particles. There are significant numbers of M31 bulge particles at all timepoints, mostly reflecting the proximity of M33. The last three bars on each panel correspond to first pericenter, apocenter, and second pericenter. M31 disk particle numbers jump at first pericenter but these apparently remain bound to the original galaxy: virtually all return to M31 before apocenter.


\subsection{MW-M31 merger}

%Galaxy Merger Sequence: Evolution of the MW/M31 Main Stellar Body (Disk, Bulge)
%throughout the Merger Sequence (prior to final coalescence)
%1. What is the density profile of the disk/bulge in the remnant? Does it follow a
%sersic profile? is it more or less concentrated than before the merger?
%2. What is the shape of the disk/bulge: bulge - does it look spherical or more
%ellipsoidal? disk - how do the spiral arms evolve?
%3. How does the velocity dispersion of both disks evolve over time? How does the
%rotation curve evolve over time? What is the ratio Vrot/σ as a function of time?
%4. How are galaxy interactions relevant for the star formation histories of galaxies?
%5. How do galaxy interactions impact the morphological classification of galaxies?
%6. How do galaxy interactions impact the growth of black holes?
%Relevant papers:
%Brooks & Christensen 2016, ASSL 418
%Querejeta + 2015 A&A 573 (connection to S0 galaxies)
%Hopkins+2008 ApJS 175

After second pericenter, the MW and M31 never fully separate and eventually merge. Their mass ratio is 1:1.6 for stellar matter and 1:1 when the DM halo is included. This is thus a `major merger' in the terminology of \todo{ref?}. A 1:1 mass ratio has been reported \citep{boylan-kolchin_dynamical_2008, ji_lifetime_2014} to lead to the shortest coalescence time.

\begin{figure}[htb!]
	\epsscale{1.0}
	\plotone{MW_M31_traj_time}
	\caption{Approach and merger in a MW-centric coordinate frame. Points are spaced at 14.3 Myr intervals.).
		\label{fig:MW_M31_traj_time}}
\end{figure}

The 3D trajectories are complex, but Figures \ref{fig:MW_M31_traj_time} and \ref{fig:MW_M31_traj_sep} are snapshots which attempts to show this. The MW CoM is always at the origin and the points show the M31 CoM at regular 14.3 Myr intervals. First pericenter is at upper left (outer), apocenter at the bottom, second pericenter in the tight reversal at upper left. The path is smooth up to 6.1 Gyr then becomes more chaotic.

\begin{figure}[htb!]
	\epsscale{1.0}
	\plotone{MW_M31_traj_sep}
	\caption{Approach and merger. Similar to Figure \ref{fig:MW_M31_traj_time} except the color coding is by separation.
		\label{fig:MW_M31_traj_sep}}
\end{figure}


\todo{changes in mass profile}

\subsubsection{Inclinations}

\todo{Relative rotation axes of disks}\ -- prograde?



\subsection{MW-M31 merger remnant}

%MW+M31 Stellar Major Merger Remnant: Stellar disk particle distribution/morphology
%1. Identify the snapshots that correspond to the merged system.
%2. What is the final stellar density profile for the combined system ? Is it well fit by
%a sersic profile? Does it agree with predictions for elliptical galaxies?
%3. What is the role of ”dry” galaxy mergers between spirals in the formation of
%elliptical galaxies?
%4. What is the distribution of stellar particles from M31 vs the MW? Are the profiles
%different?
%5. Is the 3D distribution of stars perfectly spheroidal or better fit by ellipsoids?
%6. Do you conclude that ”dry” mergers create ellipticals? or is the remnant close to
%a lenticular ?
%Relevant papers:
%Barnes, J. E., Hernquist, L. E.,1992, ApJL, 30, 705
%Duc+2013, ASPC, 447
%Querejeta + 2015 A&A 573 (connection to S0 galaxies)
%Hopkins+2008 ApJS 175

\subsubsection{Remnant shape}

\todo{how to get principal axes? boxiness?}

We can expect the remnant to settle over time into a triaxial ellipsoid \todo{ref?}. These are superficially rather featureless, as in Figure \ref{fig:rem_shape}. 

\begin{figure}[htb!]
	\epsscale{1.1}
	\plotone{remnant_shape.pdf}
	\caption{Luminous star density of the MW-M31 remnant in three orthogonal projections.
		\label{fig:rem_shape}}
\end{figure}

In observational astronomy it would be usual to fit parameters to surface brightness contours. That would also be possible for the simulation, but for a highly-determined system for which we know the mass and position of every particle there are other options.

If we combine all the baryonic matter (disk and bulge) from both MW and M31, there are $1.12 \time 10^6$ particles to consider. In the original coordinates, the moment of inertia tensor is symmetrical, $3 \times 3$:


\[ I = \begin{bmatrix}
			I_{xx} & I_{xy} & I_{xz}\\
			I_{yx} & I_{yy} & I_{yz}\\
			I_{zx} & I_{zy} & I_{zz} 
		\end{bmatrix} \]
		
\[  I_{total} \approx \begin{bmatrix}
		2.44e+04 & 2.55e+02 & 1.43e+03\\
		2.55e+02 & 1.91e+04 & -1.71e+01\\
		1.43e+03 & -1.71e+01 & 2.22e+04 
	\end{bmatrix} \]\vspace{5mm}
		
The orientation is arbitrary at this stage. To get principal axes we need the eigenvalues and eigenvectors of $I$. 

The eigenvalues give the moments of inertial about the principal axes, in arbitrary units scaled such that $A=1$:
\[ a=1.0, b= 0.85, c=0.79 \]
The luminous ellipsoid is thus triaxial (low symmetry).

The eigenvectors give an orthonormal coordinate system oriented along the principal axes:
\begin{align*}
	v_a &= \left< +0.070, -0.996 , -0.050 \right> \\
	v_b &= \left< -0.423, -0.075,  +0.903 \right> \\
	v_c &= \left< -0.904, -0.042, -0.426 \right>
\end{align*}

The moment of inertia of an ellipsoid with semi-major axes $a, b, c$ is $A = k(b^2 + c^2)$ where $k$ is a constant that depends on total mass. Other axes have the same form by symmetry. Solving for $a, b, c$ and normalizing gives:
\[ a = 1.0,\quad b = 0.94,\quad c = 0.77 \]
So the minor axis is significantly smaller than the other two: the ellipsoid is approximately prolate.

This was repeated for each subgroup by particle origin. The are the numbers involved are shown in Table 3.

\begin{deluxetable}{crrr}[htb!]
	%	\tablenum{1}
	\tablecaption{Counts of particles by origin (thousands)}
	\tablewidth{0pt}
	\tablehead{
		\colhead{Galaxy} & \colhead{Bulge} & \colhead{Disk} & \colhead{All}
	} 
	\startdata
	M31 &   95.0 &  600.0 &   695.0 \\
	MW  &   50.0 &  375.0 &   425.0 \\
	All &  145.0 &  975.0 &  1120.0 \\
	\enddata
\end{deluxetable}

The relative axis lengths are shown in Table 4. All subgroups are triaxial, but ex-MW disk particles are distinctive in retaining a particularly flattened distribution.

\begin{deluxetable}{crrr}[htb!]
	%	\tablenum{1}
	\tablecaption{Relative size of axes by particle origin}
	\tablewidth{0pt}
	\tablehead{
		\colhead{Galaxy} & \colhead{a} & \colhead{b} & \colhead{c} 
	}
	\startdata
	total &  1.0 &  0.94 &  0.77 \\
	MW disk &  1.0 &  0.90 &  0.53 \\
	MW bulge &  1.0 &  0.89 &  0.76 \\
	M31 disk &  1.0 &  0.88 &  0.71 \\
	M31 bulge &  1.0 &  0.78 &  0.71 \\
	\enddata
\end{deluxetable}

The mutual inclination angles of the major axes are shown in Table 5. Again, the former disk particles are seen to retain a distinctive structure. Clearly this collision and a single merger is not sufficient to randomize stars within the remnant. Also, this is consistent with the understanding that relaxation times are very long in collisionless systems on the scale of elliptical galaxies \citep[Section 1.2]{binney_galactic_2008}

\begin{deluxetable}{crrrrr}[htb!]
%	\tablenum{1}
	\tablecaption{Mutual inclination angles of major axis by particle origin (degrees). Suffix indicates source: \textbf{d}isk/\textbf{b}ulge.}
	\tablewidth{0pt}
	\tablehead{
		\colhead{} & \colhead{Total} & \colhead{MWd} & \colhead{MWb} & \colhead{M31d} & \colhead{M31b} 
		%\colhead{Name} & \colhead{Count} & \colhead{Count} & \colhead{Count} & \colhead{}
	}
	\startdata
	total &  -- &  45.9 &  15.0 &  83.2 &  19.4 \\
	MWd &   45.9 &   -- &  46.7 &  41.4 &  60.6 \\
	MWb &   15.0 &  46.7 &   -- &  87.6 &  32.4 \\
	M31d &   83.2 &  41.4 &  87.6 &   -- &  89.0 \\
	M31b &   19.4 &  60.6 &  32.4 &  89.0 &   -- \\
	\enddata
\end{deluxetable}


\todo{redo this for various radii cutoffs}

\subsubsection{Remnant rotational kinematics}

%MW/M31 Galaxy Major Merger Remnant: Stellar disk particle kinematics
%1. Is the stellar MW/M31 merged remnant rotating ? (create a phase diagram:
%velocity vs radius). Is it a fast or slow rotator?
%2. What is the contribution of the MW vs. M31 to the kinematics of the remnant?
%3. What is the velocity dispersion of the remnant as a function of radius
%4. Does the virial theorem work to return the total mass (stars + dark matter) of the
%remnant? Recall Lab 5 (dwarf vs. globular cluster based on velocity dispersion)
%5. Does the remnant sit on the fundamental plane?
%6. What is the specific angular momentum of the stellar remnant? does it line up
%with the dark matter halo specific angular momentum?
%7. Look at several snapshots at different points in time after the system has coalesced
%to see if the results change over time.
%8. Can ”dry” mergers create ellipticals? or is the remnant closer to a lenticular,
%(large bulge with rotating disk component) ?
%9. Could the major merger remnant be an S0 type galaxy?
%Relevant papers:
%Romanowsky+2003, Science 301, 1696
%Cox + 2006, ApJ 650
%Querejeta + 2015 A&A 573 (connection to S0 galaxies)
%Hopkins+2008 ApJS 175

Baryonic particles in the merger remnant have a small net angular momentum, and we can rotate the coordinate system to align this vector with the $z$-axis. However, the phase diagrams in Figure \ref{fig:rem_phase} show that velocities are mostly randomly distributed and few of the particles show much assymetry.

\begin{figure}[htb!]
	\epsscale{1.12}
	\plotone{remnant_phase.pdf}
	\caption{Phase diagrams of the MW-M31 remnant, orthogonal views.
	\label{fig:rem_phase}}
\end{figure}

\todo{alignment between particles of different origin?}

\subsubsection{Remnant DM halo}

%MW/M31 Halo Major Merger Remnant: Dark matter halo evolution (density / kinematics)
%1. What is the final density profile ? Is it well fit by a Hernquist profile ? Is it more
%or less concentrated than the MW or M31 before they merged?
%2. Is the 3D dark matter distribution spheroidal? or elongated like an ellipsoid?
%What do terms like prolate, oblate, or triaxial halos mean? https://astronomy.com/news/2010/
%01/astronomers-map-the-shape-of-galactic-dark-matter
%3. What are the kinematics of the dark matter halo - is it rotating? what is the
%dispersion - does the virial theorem give you the right mass?
%4. What is the distribution of dark matter particles from the M31 vs the MW? Are
%they different? are the kinematics different?
%5. what is the average specific angular momentum? Is it the same or different than
%the halos of either galaxy before they merged.
%6. What is the escape speed of the remnant as a function of radius?
%7. Where is the ”end” of the halo? How might we define this?
%Relevant Papers:
%Frenk & White 2012 Annalen der Physik 524, 507 Review article
%Abadi + 20 MNRAS, 2010 407

\todo{alignment between particles of different origin?}

\section{Discussion}

\todo{add some!}



% =====================================
% Bibliography stuff
\bibliography{project_Leach}{}
\bibliographystyle{aasjournal}

\end{document}
