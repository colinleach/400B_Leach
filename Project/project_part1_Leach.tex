\documentclass[twocolumn]{aastex63}

\usepackage{natbib}
\usepackage{booktabs} % helps with pandas to_latex() output
\usepackage{multirow}
\usepackage{amsmath}
%\usepackage{dblfloatfix}    % To enable figures at the bottom of page
\usepackage{lipsum} % insert nonsense text as placeholder

\graphicspath{{img/}}

\newcommand{\todo}{\color{red}{TODO}\color{black}\hspace{2mm}}

\shorttitle{MW-M31 Interactions}
\shortauthors{Colin Leach}

\begin{document}
	
\title{Future Dynamics of the Local Group. I. MW-M31 Interactions}

\author[0000-0003-3608-1546]{Colin Leach}

\begin{abstract}
	
Existing data from an N-body simulation of the local group is reanalyzed. Future trajectories of the Milky Way (MW), M31 and M33 are presented, with emphasis on the structural and kinematic effects of a MW-M31 close approach to 35 kpc just before 4 Gyr, then final approach and merger around 6 Gyr. The final state of the merger remnant is examined in some detail, showing that stellar material is by no means fully relaxed at the end of the simulation (11.44 Gyr). The central region is approximately prolate, with significant rotation about its long axis. At larger radii, baryonic matter appears kinematically distinct from the central region with a different rotation axis. \vspace{5mm}
	
\todo{Add further concise and intelligent summary of the paper, once I get a clearer idea what it will include (and how to seem intelligent).}\vspace{5mm}

	
\end{abstract}

\keywords{Galaxy Merger -- Local Group -- Stellar Disk -- Stellar Bulge -- Dark Matter Halo -- Hernquist Profile -- Merger Remnant}

%\tableofcontents


\section{Introduction}

The currently-accepted model of galaxy formation involves baryonic matter (gas and dust) falling into gravitational potential wells created by local over-densities in the Dark Matter (DM). Further gravitational collapse and Jeans fragmentation can then lead to creation of galaxies and stars \citep{mo_galaxy_2010}.

However, decades of observational and theoretical studies tell us that, firstly, this by itself does not account for the wide range of galaxy morphologies seen at all epochs; secondly, there is no reason to suppose that galaxies continue in serene isolation after their formation.

Attempts to model interactions and mergers between galaxies with numerical simulations goes back at least to \citet{toomre_galactic_1972}. This field continues to develop, with improvements in both hardware and algorithms allowing larger particle numbers in N-body simulations and more sophisticated treatment of gas hydrodyamics, magnetic fields and other factors \citep{bodenheimer_numerical_2007}.

As with all theoretical studies, it is vital to stay connected to the best experimental data as this constantly evolves, constantly comparing models against observations. Checking simulations against high-redshift galaxies is necessary but inevitably approximate. A perhaps more rigorous test is to model the galaxies for which we have the most precise and detailed observational measurements: those which are (by far) closest to us. 

The largest galaxies in our Local Group (LG) are the Milky Way (MW), Andromeda (M31) and Triangulum (M33).  A simulation of MW--M31--M33 orbital evolution was described previously in \citet{marel_m31_2012}, hereafter vdM12. That paper included an extensive analysis of both N-body simulations and semi-analytic orbit integrations. The present study uses data from the same N-body simulation to carry out further computational analysis.

The simulation was based on data in \citet{marel_m31_2012-1} suggesting that M31 is approaching the MW directly with little proper motion detected by Hubble Space Telescope studies. Recent data from Gaia DR2 \citep{brown_gaia_2018} suggest that infall is slightly less radial than previously thought \citep{marel_first_2019}, leading to a slightly later first approach with a larger pericenter distance. However, detailed simulations based on that new data have not yet been carried out.

This paper will review the initial conditions and time evolution for multiple physical parameters of the simulation. Particular attention will be paid to the first MW-M31 close approach around 4 Gyr, the second approach and merger around 6 Gyr, and the structure and dynamics of the post-merger remnant.

Time probably precludes much analysis of the fate of M33, which will need to be the subject of a future paper.

\subsection{Data}

Data from one N-body simulation in vdM12 was supplied in text-file format by one of the original authors. This included position and velocity data for each particle at the current epoch ($t=0$) and 800 future time steps. For ease of analysis, this was all transferred to the open source database PostgreSQL\footnote{\url{http://www.postgresql.org}} (approximately 1.35 billion records). The same database was used to store computed summary data during the analysis.

\begin{deluxetable}{crrrr}[htb!]
	\tablecaption{Particle counts
		\label{tbl:counts}}
	\tablewidth{0pt}
	\tablehead{
		\colhead{Galaxy} & \colhead{DM Halo} & \colhead{Disk} & \colhead{Bulge} & \colhead{Total} 
		%\colhead{Name} & \colhead{Count} & \colhead{Count} & \colhead{Count} & \colhead{}
	}
	\startdata
	MW   &  250,000 &   375,000 &    50,000 &   675,000 \\
	M31  &  250,000 &   600,000 &    95,000 &   945,000 \\
	M33  &   25,000 &    46,500 &        0 &    71,500 \\
	\midrule
	LG  &  525,000 &  1,021,500 &   145,000 &  1,691,500
	\enddata
\end{deluxetable}\vspace{-10mm}

Particle counts for each time point are shown in Table \ref{tbl:counts} and total masses in Table \ref{tbl:aggmass}. We can see that total mass is the same for MW/M31 but our galaxy has more dark matter (lower baryon fraction) and M31 has more luminous stars (higher baryon fraction). M33 is about 10-fold lighter than either.

\begin{deluxetable}{crrrr}[htb!]
	\tablecaption{Aggregate masses ($M_\Sun \times 10^{12}$)
	\label{tbl:aggmass}}
	\tablewidth{0pt}
	\tablehead{
		\colhead{Galaxy} & \colhead{DM Halo} & \colhead{Disk} & \colhead{Bulge} & \colhead{Total}
	}
	\startdata
	MW  &      1.975 &      0.075 &       0.010 &  2.060 \\
	M31 &      1.921 &      0.120 &       0.019 &  2.060  \\
	M33 &      0.187 &      0.009 &       0.000 &  0.196  \\
	\midrule
	LG &   4.082 &      0.204 &       0.029 &  4.316
	\enddata
\end{deluxetable}\vspace{-10mm}

The coordinate system is approximately centered on the Milky Way at $t=0$. The center of mass (CoM) of all particles in the system is not fixed over time, moving at an average of $\vec{v} = \left< 35.9, -26.7, 27.5 \right>$ km/s with some minor fluctuations due to numerical approximations. In contrast, the total angular momentum of the system is very small at all time points.

\subsection{Software}

The work in this report was carried out in Python using standard packages. Full details are available online\footnote{Code \url{https://github.com/colinleach/400B_Leach}\\documentation \url{https://400b-leach.readthedocs.io}}

\section{Results}

\subsection{Trajectories}

The simulation does not explicitly include a supermassive black hole (SMBH) at the center of each galaxy, but the galactic center was defined by calculating the center of mass (CoM) of the disk particles and iteratively constraining the radius of interest until convergence.

To plot motions of the three galactic CoMs it is convenient to transform to a coordinate system in which at $t=0$ they all lie in the $x,y$ plane with MW and M31 on the $x$-axis. The overall CoM is moving, as noted above, so at each time point the coordinates are translated to center it at the origin.

\begin{figure}[htb!]
	\plotone{traj_xy.pdf}
	\caption{Trajectories of each galactic center of mass in the X',Y' plane. Points are at 71 Myr intervals.
		\label{fig:traj_xy}}
\end{figure}

\begin{figure}[htb!]
	\epsscale{0.9}
	\plotone{traj_z.pdf}
	\caption{Trajectories of each galactic center of mass perpendicular to the X',Y' plane.
		\label{fig:traj_z}}
\end{figure}

\begin{figure}[hbt!]
	\epsscale{0.9}
	\plotone{rel_sep.pdf}
	\caption{Separations of galactic CoMs.
		\label{fig:rel_sep}}
\end{figure}

\begin{figure}[hbt!]
	\epsscale{0.9}
	\plotone{rel_vel.pdf}
	\caption{Relative velocities of galactic CoMs.
		\label{fig:rel_vel}}
\end{figure}

In vdM12 this is referred to as the X',Y',Z' coordinate system and their figure 2 shows multiple views of how the galaxies move through time. In this paper, Figures \ref{fig:traj_xy} and  \ref{fig:traj_z} show some alternative views in essentially the same coordinates (up to a sign; the $x$ and $z$ axes are flipped). Figure \ref{fig:traj_xy} reproduces the top left panel of vdM12. Figure \ref{fig:traj_z} shows that MW and M31 remain close to the starting plane while M33 has larger, irregular out-of-plane motions.

Relative motions of the CoMs are shown against time in Figures \ref{fig:rel_sep} and \ref{fig:rel_vel}, equivalent to figures 3 and 4 in vdM12.


There is a MW-M31 close approach with first pericenter at 3.96 Gyr with a minimum separation of 35.1 kpc, then a separation to 173 kpc at apocenter and finally a convergence to 7.8 kpc at second pericenter and merger between 5.9 - 6.5 Gyr. Relative velocities spike sharply during these approaches, as potential energy is converted to kinetic energy, before declining to essentially zero.

\begin{figure}[hbt!]
	\epsscale{0.6}
	\plotone{collision_a}
	\caption{\label{fig:3view_first_apo_a}}
\end{figure}\vspace{-5mm}

\begin{figure}[hbt!]
	\epsscale{0.7}
	\plotone{collision_b}
	\caption{\label{fig:3view_first_apo_b}}
\end{figure}\vspace{-5mm}

\begin{figure}[hbt!]
	\epsscale{0.8}
	\plotone{collision_c}
	\caption{\label{fig:3view_first_apo_c}}
\end{figure}

Figures \ref{fig:3view_first_apo_a} to \ref{fig:3view_first_apo_c} show a view of first apocenter as a disk density plot from three orthogonal directions. The full animation is available online\footnote{\url{https://github.com/colinleach/400B_Leach/blob/master/animations/collisions_disk.mp4}}


Meanwhile, in this simulation run M33 remains separate throughout, albeit on a decaying orbit. In vdM12 the authors investigate the effect of small changes in initial conditions and estimate a 9\% chance of an M33-MW collision at first pericenter, before the M31-MW merger.

\subsection{Mass profiles and rotation curves}

Figure \ref{fig:massprof0} shows the cumulative mass profile, by particle type and in total, for each galaxy. The center of each galaxy is dominated by baryonic matter with the DM halo becoming dominant at larger radii.
\begin{figure}[bht!]
	\epsscale{1.2}
	\plotone{massprof_000}
	\caption{Mass profiles for each galaxy at the current epoch.
		\label{fig:massprof0}}
\end{figure}

Figure \ref{fig:rotcurve0} shows the rotation curves expected from these mass profiles. Without the DM halo the circular velocity would peak within a few kpc of the CoM then fall steadily at larger radii. With the more diffuse DM halo added, we see the relatively flat overall rotation curves which attracted the attention of 20th century astronomers including \citet{zwicky_rotverschiebung_1933} and \citet{rubin_rotation_1970}

\begin{figure}[!bht!]
	\epsscale{1.2}
	\plotone{rotcurve_000}
	\caption{Rotation curves for each galaxy at the current epoch.
		\label{fig:rotcurve0}}
\end{figure}

\subsection{Stellar disk}

\subsubsection{Structure}

\todo{identify the bar?}\ 

\todo{more on spiral arms}


\begin{figure}[!bht]
	\gridline{\fig{cyl_MW_000.png}{0.2\textwidth}{}
		\fig{cyl_M31_000.png}{0.2\textwidth}{}}\vspace{-7mm}
	\gridline{\fig{cyl_MW_278.png}{0.2\textwidth}{}
		\fig{cyl_M31_278.png}{0.2\textwidth}{}}\vspace{-7mm}
	\gridline{\fig{cyl_MW_300.png}{0.2\textwidth}{}
		\fig{cyl_M31_300.png}{0.2\textwidth}{}}\vspace{-7mm}
	\caption{Disk particles, cylindrical coordinates, from MW (left) and M31 (right) at three timepoints: Start (top), first pericenter (mid), near apocenter (bottom).
		\label{fig:6cyl}}
\end{figure}

%\begin{figure}[!bht]
%	\epsscale{0.8}
%	\plotone{6cyl_grab}
%	\caption{Disk particles from MW (top) and M31 (bottom) at three timepoints; cylindrical coordinates.\\
%		\todo{\textit{make this page-width in final layout}}
%		\label{fig:6cyl}}
%\end{figure}

Disk structure and evolution may be easier to visualize if we transform to cylindrical coordinates (with the angular momentum vector along the $z$-axis) and use $r-\theta$ plots. Figure \ref{fig:6cyl} shows this for the two large galaxies at several timepoints (as labelled).  Spiral arms show up in both early in the simulation. First pericenter has little effect, but 0.3 Gyr later we see highly prominent tidal tails. These images are taken from a full animation for each galaxy, available online\footnote{\url{https://github.com/colinleach/400B_Leach/tree/master/animations}, files cyl\_M*\_disk.mp4 }

\subsubsection{Inclinations}

Galactic disks have a well-defined angular momentum vector which is relatively easy to calculate in this type of simulation. Figure \ref{fig:inclinations_xy} shows the angle each makes to the X'-Y' plane over time.

\begin{figure}[ht!]
	\plotone{inclinations_xy.pdf}
	\caption{Angular momentum inclination angle to the X',Y' plane for each set of galactic disk particles.
		\label{fig:inclinations_xy}}
\end{figure}

The mutual angle between galactic disks can have a significant impact on how tidal disruption and merger dynamics play out \todo{ref?}. This can be calculated from the vector dot products:

\[ \theta = \arccos (\hat{L}_1 \cdot \hat{L}_2 ) \]

Results for the MW-M31 and M33-M31 pairs are shown in Figure \ref{fig:inclinations_mutual}. For MW-M31, the angle is largely stable until near first pericenter, when tidal forces bring the two disks closer to alignment. This trend continues slowly until near second pericenter. Surprisingly, the angle appears to increase after merger. This suggests either some partitioning of particles of different origin within the remnant, or some addition factor that invalidates this simple analysis. A later section will discuss the complex radial dependency within the remnant, which Figure \ref{fig:inclinations_mutual} does not take into account.

\begin{figure}[ht!]
	\plotone{inclinations_mutual.pdf}
	\caption{Angular momentum angles between pairs of galaxies.
		\label{fig:inclinations_mutual}}
\end{figure}

The large variations in M33-M31 angles are indicative of the extensive tidal disruption of the much smaller M33 galaxy. Details are outside the scope of the present paper.

\subsubsection{Velocity dispersion}

The changes in velocity dispersion of disk particles originating from each galaxy are shown in Figure \ref{fig:vel_disp}. The small periodic oscillation seen from the start, especially in M31, appears to be caused by deviations from radial symmetry in the disk: spiral arms and an increasingly prominent bar. Small MW spikes at initial pericenter (around 4 Gyr) and much larger ones at merger (around 6 Gyr) are clearly visible.

M33 is on an irregular, elliptical and decaying orbit about the MW-M31 merger remnant after about 6.5 Gyr. Velocity dispersion appears to peak at intervals. This perhaps corresponds to successive pericenters when M33 experiences maximal tidal disruption, but this will need further analysis.

\begin{figure}[htb!]
	\epsscale{1.1}
	\plotone{vel_disp.pdf}
	\caption{Velocity dispersion of disk particles from each galaxy over time.
		\label{fig:vel_disp}}
\end{figure}

\todo{disk rotation curve, $V_{rot}/\sigma$}

\subsection{Stellar Bulge}

A bulge is present in the MW and M31 but not M33. This region of generally older stars extends further above and below the central plane than disk stars. Kinematics of the bulge are more typical of an elliptical galaxy than a spiral disk.

In a study of elliptical galaxies, de Vaucouleurs showed that surface brightness falls off exponentially from the center and approximately as the one-fourth power of radius \citep{de_vaucouleurs_recherches_1948}. Later work found that this was too restrictive for a wider population of galaxies, so Sérsic generalized the formula to have the inverse exponential $n$ as an additional free parameter \citep{sersic_influence_1963}:

\[ \log_{10} \left( \frac{I(r)}{I_e} \right) =  -3.3307 \left[ \left( \frac{r}{R_e} \right)^{1/n} - 1 \right] \]

Here $R_e$ is the radius with which half the light is emitted,  $I_e$ is the surface brightness at $R_e$ and $n$ is the Sérsic parameter.

This formula is intended for analyzing photographic images and is in terms of light intensity. We have no brightness data in the current simulation, but for systems with few young blue stars we can assume the stellar mass to light ratio $M_\star/L \sim 1$. This is probably a reasonable approximation for undisturbed bulges and for an elliptical merger remnant long after the collision. $R_e$ is then the radius enclosing half the mass.

\begin{figure}[bht!]
	\epsscale{1}
	\plotone{sersic_Re}
	\caption{Half-mass radius for bulge particles.
		\label{fig:sersic_Re}}
\end{figure}

We can see from Figure \ref{fig:sersic_Re} that for each galaxy the bulge half-mass radius is fairly stable up to the collision and merger of MW and M31. After a period of disturbance, they again become stable at a higher level. The M31 bulge is more diffuse than the MW bulge throughout, and the ex-bulge stars are clearly not randomized in the merger remnant: ex-M31 stars tend towards larger radii than ex-MW stars.

\begin{figure}[bht!]
	\epsscale{1}
	\plotone{sersic_n}
	\caption{Sersic $n$ for bulge particles, with $1\sigma$ error bars.
		\label{fig:sersic_n}}
\end{figure}

The Sérsic parameter $n$ was estimated by a nonlinear least squares fit to the bulge mass profile. As shown in Figure \ref{fig:sersic_n} it is fairly constant around 5.5 for any period with meaningful data. The spikes around 6 Gyr should probably be ignored: many values in this collision period are missing, as the least-squares fit failed, and the available data has substantially larger error bars than during stable epochs.

\begin{figure}[bht!]
	\epsscale{1.0}
	\plotone{bulge_mp}
	\caption{Bulge mass density profile for both galaxies at the beginning and end of the simulation.
		\label{fig:bulge_mp}}
\end{figure}

The larger half-mass radius of M31 is reflected in the mass density profile, as shown in Figure \ref{fig:bulge_mp}. MW bulge stars have a higher central peak, M31 bulge stars are more numerous at larger radii. This is true both early in the simulation and in the merger remnant at late times. For both galaxy bulge stars, the central peak is less pronounced post-merger.

The Sérsic fit for both galaxy bulges looks reasonable outside the central density peak, as shown for the MW in Figure \ref{fig:MW_bulge_sersic}. The plot for M31 (not included here) is very  similar.

\begin{figure}[bht!]
	\epsscale{1.0}
	\plotone{MW_bulge_sersic}
	\caption{MW bulge mass density profiles and Sérsic best fits. Time points are the beginning, the pre-merger pericenter, and the end of the simulation
		\label{fig:MW_bulge_sersic}}
\end{figure}

\subsection{Dark Matter halo}

\subsubsection{Halo mass profile}

Figure \ref{fig:rotcurve0} also added a theoretical curve in which the DM halo is fitted by a Hernquist profile \citep{hernquist_analytical_1990}. The cumulative mass out to radius $r$ is given by
\[ M(r) = M_h \frac{r^2}{(a+r)^2} \]
where $M_h$ is the total mass of halo particles (see Table 2) and $a$ is a scale radius which encloses a quarter of the halo mass. Non-linear least squares fitting, similar to that used for Sérsic profiles in a previous section, gave scale radii of 61.1 kpc for MW and M31, 24.3 kpc for M33 at $t=0$.

\begin{figure}[htb!]
	%	\epsscale{0.8}
	\plotone{hernquist_a.pdf}
	\caption{Hernquist scale radius $a$ for DM halo particles originating from each galaxy, with $1\sigma$ error bars.
		\label{fig:hernquist_a}}
\end{figure}

Time evolution of the scale radius $a$ is shown in Figure \ref{fig:hernquist_a}. The MW and M31 remain very similar through first pericenter, then start to diverge with MW particles tending to a larger radius than M31: the opposite of bulge particles. This becomes most pronounced during and after merger. The dissimilar distribution in the merger remnant will be discussed in a later section.  

The scale radius for M33 grows inexorably as the original halo is scattered by tidal forces. Figure \ref{fig:hernquist_a} also shows the increasingly wide error bars for M33: halo particles for this galaxy are no longer well fitted by a Hernquist profile.

\subsubsection{Halo rotation}

The specific angular momentum $\vec{h}$ can be calculated from
\[ \vec{h} = \frac{\sum_i{\vec{r_i} \times m_i \vec{v}_i}}{\sum_i{m_i}} \]

The halo specific angular momentum for each large galaxy is shown in Figure \ref{fig:halo_h}. It appears that both are barely rotating at the current epoch, but spin up rapidly during first pericenter and again around the time of merger, as tidal forces convert orbit angular momentum into spin angular momentum. Differences in the remnant will be discussed in a later section

\begin{figure}[htb!]
	%	\epsscale{0.8}
	\plotone{halo_h.pdf}
	\caption{Specific angular momentum for halo particles of each galaxy about its CoM (kpc$^2$/Myr).
		\label{fig:halo_h}}
\end{figure}

Data for M33 is omitted from Figure \ref{fig:halo_h} for clarity. The spin-up is much more dramatic for this minor galaxy, with peaks approaching 140 kpc$^2$/Myr, making it unsuitable to plot on the same axes.

\todo{effect of radius}

\subsection{MW-M31 Close approach}

\subsubsection{Inclinations}

The MW and M31 disks have angular momentum vectors inclined at an angle of $52^\circ$ to each other shortly before pericenter. The angles to their mutual orbital angular momentum are $82^\circ$ (MW) and $88^\circ$ (M31): almost perpendicular, but values less than $90^\circ$ are classed as prograde approaches.

\subsubsection{Tidal tails and bridges}

%Galaxy Merger Sequence: MW and M31 tidal tail evolution and/or evolution of stellar
%bar Sean Cunningham, Steven Zhou-Wright
%1. How can you identify Tidal tails and bridges throughout the MW-M31 interaction
%sequence?
%2. How can you identify the bar? or pseudo-bulge (disk stars kicked up in the x-shape
%pattern by the bar) ? How do those structures evolve?
%3. Where do the tidal tails come from? Can you select the tail and trace them back
%to the undisturbed systems?
%4. What are the kinematics of the tidal tails over time? Do they change in velocity
%dispersion and energy?
%5. What is the morphological change of the tidal tails over time? Do they grow in
%size?
%6. are the tidal tails unbound? Do the tails return to their original galaxies?
%7. How long lived are the tidal tails? For how long might we observe the system with
%extended tails? What does this mean for our ability to identify merging galaxies?
%8. What is the mass transfer between the two galaxies? Do they exchange material?
%If so, where does this exchanged material end up? Does it rotate in the plane of
%the disk? What is the mass exchange between the MW nad M31 over time?
%59. What is the structure of the tidal tails? Are there any clumps or is it smooth?
%Toomre A., Toomre J. 1972, ApJ, 178, 623
%Barnes+2004 MNRAS 350 (model of an example major merger: Mice . Also look
%up ”antennae galaxies”)
%Privon+2013, ApJ 771
%Ji et al. 2014 A&A 566

The presence of long, symmetrical tails giving some galaxies a distinct `S'-shape has been described at least as far back as \citet{zwicky_novel_1955}. Some astronomers postulated that these were the result of tidal forces during close, glancing encounters, but this was often contested until a detailed computational study by \citet{toomre_galactic_1972}.

Reviewing a broad range of N-body simulations, \citet{barnes_dynamics_1992} noted that ``such features are clearly \textit{relics} of recent collisions rather than ongoing interactions.'' In our simulation, both MW and M31 disks remain near-circular during much of the close approach, but conspicuous tails develop as the centers then move further apart: see Figure \ref{fig:6cyl} and the animations referred to in Footnote 3. We also see a more sparsely-populated bridge forming between the galaxies.

\begin{figure}[ht!]
	\epsscale{0.7}
	\plotone{selected_a.pdf}
	\caption{Manual selection of bridge particles at 0.33 Gyr after the first MW-M31 pericenter: stellar surface density and the selected region. Orientation is with MW top, M31 bottom and M33 lower left. 
	\label{fig:bridge_a}}
\end{figure}

To determine the nature and origin of stars in this region, a manual selection was performed as in Figure \ref{fig:bridge_a}. Stars within the yellow rectangle are shown with velocity vectors in Figure \ref{fig:bridge_b} and origin in Figure \ref{fig:bridge_c}. Velocities are diverse: mean 195 km/s (comparable to $v_{\text{circ}}$ in the disk), range 19-586 km/s. 

\begin{figure}[ht!]
	\epsscale{0.9}
	\plotone{selected_b.pdf}
	\caption{Velocity vectors for stars selected in Figure \ref{fig:bridge_a}. 
		\label{fig:bridge_b}}
\end{figure}

\begin{figure}[ht!]
	\epsscale{0.9}
	\plotone{selected_c.pdf}
	\caption{Origin by galaxy and particle type for stars selected in Figure \ref{fig:bridge_a}.
		\label{fig:bridge_c}}
\end{figure}

It appears from the right panel that stars in the tail regions originate in the corresponding disk. The bridge region is more mixed and appears to have a high proportion of former bulge stars. To study this further the coordinate system was transformed to place the large galaxy CoMs on the $x$-axis at $\pm 64$ kpc, as in Figure \ref{fig:bridge2}. It is clear in this view that one MW tail is oriented approximately towards the center of M31. 

\begin{figure}[htb!]
	\epsscale{1.1}
	\plotone{density_rot_300.pdf}
	\caption{View along the midplane between the galactic centers, MW on the left.
		\label{fig:bridge2}}
\end{figure}

\begin{deluxetable}{crrr}[ht!]
	\tablenum{3}
	\tablecaption{Particle counts close to the midplane 
		\label{tbl:pcounts}}
	\tablewidth{0pt}
	\tablehead{
		\colhead{} & \colhead{Bulge} & \colhead{Disk} & \colhead{Total}
	}
	\startdata
	MW      &    305 &  1317 &  1622 \\
	M31     &   1137 &     4 &  1141 \\
	\midrule
	Total     &   1442 &  1321 &  2763 \\
	\enddata
\end{deluxetable}

The different orientations mean that symmetry about the midplane is imperfect, so the ``bridge'' region was taken as $-20 < x < 30$ kpc. A count of stars in this region is shown in Table \ref{tbl:pcounts}. This confirms that the largest populations are MW disk stars (mostly in a relatively dense tail) and M31 bulge stars (more widely dispersed).

\todo{identify, trace history, trace fate}\ 

\todo{Jacobi radius}\ 

\todo{Tail kinematics: $\sigma$, energy}\ 

\todo{Lifetime of tails}\ 



\subsubsection{Mass transfer}

Stars are scattered from galaxies even in normal times, and this can be expected to increase significantly during near-misses and collisions. To get a first impression of how many stars and DM particles may end up closer to a different galaxy, we looked at the relative distances of each particle to each of the three galaxy CoMs. It should be emphasized that kinematics is not considered at this stage, so nothing can be said about which particles are gravitationally bound.

\begin{figure}[htb!]
	\epsscale{0.9}
	\plotone{transfer_counts_mpl}
	\caption{Particles closer to a different CoM.
		\label{fig:trans_count}}
\end{figure}

Figure \ref{fig:trans_count} shows that some particles are far from their notional galaxy even at the start. This increases somewhat during first pericenter around 4 Gyr, then jumps permanently during the second pericenter and merger. The plot cuts off at 7 Gyr because it becomes meaningless to consider the MW/M31 CoMs as separate points post-merger.

\begin{figure}[htb!]
	\epsscale{1.1}
	\plotone{transfer_ptype}
	\caption{Particles closer to a different CoM.
		\label{fig:trans_p}}
\end{figure}

Figure \ref{fig:trans_p} looks at a few timepoints by particle type, showing that the overwhelming majority of these particles are from the DM halo. This is unremarkable, given the prevalence of these particles at large radii and their correspondingly weak gravitational binding.

\begin{figure}[htb!]
	\epsscale{1.1}
	\plotone{transfer_lum}
	\caption{Luminous particles closer to a different CoM (DM halo hidden).
		\label{fig:trans_l}}
\end{figure}

To focus on the baryonic matter, Figure \ref{fig:trans_l} hides the DM halo and expands the $y$-axis to show only bulge and disk particles. There are significant numbers of M31 bulge particles at all timepoints, mostly reflecting the proximity of M33. The last three bars on each panel correspond to first pericenter, apocenter, and second pericenter. M31 disk particle numbers jump at first pericenter but these apparently remain bound to the original galaxy: virtually all return to M31 before apocenter.


\subsection{MW-M31 merger}

%Galaxy Merger Sequence: Evolution of the MW/M31 Main Stellar Body (Disk, Bulge)
%throughout the Merger Sequence (prior to final coalescence)
%1. What is the density profile of the disk/bulge in the remnant? Does it follow a
%sersic profile? is it more or less concentrated than before the merger?
%2. What is the shape of the disk/bulge: bulge - does it look spherical or more
%ellipsoidal? disk - how do the spiral arms evolve?
%3. How does the velocity dispersion of both disks evolve over time? How does the
%rotation curve evolve over time? What is the ratio Vrot/σ as a function of time?
%4. How are galaxy interactions relevant for the star formation histories of galaxies?
%5. How do galaxy interactions impact the morphological classification of galaxies?
%6. How do galaxy interactions impact the growth of black holes?
%Relevant papers:
%Brooks & Christensen 2016, ASSL 418
%Querejeta + 2015 A&A 573 (connection to S0 galaxies)
%Hopkins+2008 ApJS 175

After second pericenter, the MW and M31 never fully separate and eventually merge. Their mass ratio is 1:1.6 for stellar matter and 1:1 when the DM halo is included. This is thus a `major merger', which is generally taken to mean closer than 1:4 luminosity ratio (or mass ratio as a proxy). A 1:1 mass ratio has been reported \citep{boylan-kolchin_dynamical_2008, ji_lifetime_2014} to lead to the shortest coalescence time.

\begin{figure}[htb!]
	\epsscale{1.0}
	\plotone{MW_M31_traj_time}
	\caption{Approach and merger in a MW-centric coordinate frame. Points are spaced at 14.3 Myr intervals.).
		\label{fig:MW_M31_traj_time}}
\end{figure}

The 3D trajectories are complex, but Figures \ref{fig:MW_M31_traj_time} and \ref{fig:MW_M31_traj_sep} are snapshots which attempts to show this. The MW CoM is always at the origin and the points show the M31 CoM at regular 14.3 Myr intervals. First pericenter is at upper left (outer), apocenter at the bottom, second pericenter in the tight reversal at upper left. The path is smooth up to 6.1 Gyr then becomes more chaotic.

\begin{figure}[htb!]
	\epsscale{1.0}
	\plotone{MW_M31_traj_sep}
	\caption{Approach and merger. Similar to Figure \ref{fig:MW_M31_traj_time} except the color coding is by separation.
		\label{fig:MW_M31_traj_sep}}
\end{figure}


\todo{changes in mass profile}

\subsubsection{Inclinations}

The MW and M31 disks have angular momentum vectors inclined at an angle of $37^\circ$ to each other shortly before final approach and merger. The angles to their mutual orbital angular momentum are $74^\circ$ (MW) and $64^\circ$ (M31): more clearly prograde than at first pericenter.

\todo{discuss implications}

\subsection{Merger stellar remnant}

%MW+M31 Stellar Major Merger Remnant: Stellar disk particle distribution/morphology
%1. Identify the snapshots that correspond to the merged system.
%2. What is the final stellar density profile for the combined system ? Is it well fit by
%a sersic profile? Does it agree with predictions for elliptical galaxies?
%3. What is the role of ”dry” galaxy mergers between spirals in the formation of
%elliptical galaxies?
%4. What is the distribution of stellar particles from M31 vs the MW? Are the profiles
%different?
%5. Is the 3D distribution of stars perfectly spheroidal or better fit by ellipsoids?
%6. Do you conclude that ”dry” mergers create ellipticals? or is the remnant close to
%a lenticular ?
%Relevant papers:
%Barnes, J. E., Hernquist, L. E.,1992, ApJL, 30, 705
%Duc+2013, ASPC, 447
%Querejeta + 2015 A&A 573 (connection to S0 galaxies)
%Hopkins+2008 ApJS 175

\subsubsection{Remnant shape}

\todo{boxiness?}

We can expect the remnant to settle over time into a triaxial ellipsoid \todo{ref?}. In observational astronomy it would be usual to determine the shape by fitting ellipses to surface brightness contours. That is also possible for the simulation, but for a highly-determined system for which we know the mass and position of every particle there are other options.

If we combine all the baryonic matter (disk and bulge) from both MW and M31, there are $1.12 \times 10^6$ particles to consider. Some of these have been ejected to large radius where they have an exaggerated effect on the moment of inertia, so only those within 40 kpc of the CoM were used in the calculation. These were about 88\% of the original stellar particles from the two precursor galaxies.

In the original coordinates, the moment of inertia tensor is symmetrical ($I_{ij} = I_{ji}$), $3 \times 3$:

\[ I = \begin{bmatrix}
			I_{xx} & I_{xy} & I_{xz}\\
			I_{yx} & I_{yy} & I_{yz}\\
			I_{zx} & I_{zy} & I_{zz} 
		\end{bmatrix} \]
		
\[  I_{\text{stellar}} \approx 10^3 \times \begin{bmatrix}
		3.26 & 0.181 & 0.152\\
		0.181 & 2.97 & 0.138\\
		0.153 & 0.138 & 2.80
	\end{bmatrix} \] %\vspace{2mm}
		
The orientation is arbitrary at this stage. To get principal axes we need the eigenvalues and eigenvectors of $I$. 

The eigenvalues give the moments of inertia about the principal axes, in arbitrary units scaled such that $A=1$ and $A \ge B \ge C$:
\[ A=1.0,\quad B= 0.85,\quad C=0.80 \]

The eigenvectors give an orthonormal coordinate system oriented along the principal axes:
\begin{align*}
	\hat{v}_a &= \left< -0.844, -0.438, -0.309 \right> \\
	\hat{v}_b &= \left< -0.524, +0.797,  +0.302 \right> \\
	\hat{v}_c &= \left< -0.114, -0.416 , +0.902 \right>
\end{align*}

Determining the shape of a three-dimensional distribution of particles is known to have many subtleties \citep{maccio_concentration_2007, jing_triaxial_2002}. As a simple first approximation, the moment of inertia of an ellipsoid with semi-major axes $a, b, c$ is $A = k(b^2 + c^2)$ where $k$ is a constant that depends on total mass. Other axes have the same form by symmetry. Solving for $a, b, c$ and normalizing gives:
\[ a = 1.0,\quad b = 0.94,\quad c = 0.77 \]
So by this method the remnant is triaxial (low-symmetry, with $a \ne b \ne c$). However, the minor axis $c$ is significantly smaller than the other two: the ellipsoid is approximately oblate ($a \approx b > c$).

%The numbers involved are shown in Table 3. 
%
%\todo{needs recalc within radius cutoff}
%
%\begin{deluxetable}{crrr}[htb!]
%	\tablecaption{Counts of particles by origin (thousands)
%		\label{tbl:remn_counts}}
%	\tablewidth{0pt}
%	\tablehead{
%		\colhead{Galaxy} & \colhead{Bulge} & \colhead{Disk} & \colhead{All}
%	} 
%	\startdata
%	M31 &   95 &  600 &   695 \\
%	MW  &   50 &  375 &   425 \\
%	All &  145 &  975 &  1120 \\
%	\enddata
%\end{deluxetable}

%This was repeated for each subgroup by particle origin. The relative axis lengths are shown in Table 4. All subgroups are triaxial (low-symmetry, with $a \ne b \ne c$), but ex-MW disk particles are distinctive in retaining a particularly flattened distribution.
%
%\begin{deluxetable}{crrr}[htb!]
%	\tablecaption{Relative size of axes by particle origin
%		\label{tbl:remn_axes}}
%	\tablewidth{0pt}
%	\tablehead{
%		\colhead{Galaxy} & \colhead{a} & \colhead{b} & \colhead{c} 
%	}
%	\startdata
%	total &  1.0 &  0.95 &  0.78 \\
%	MW disk &  1.0 &  0.95 &  0.64 \\
%	MW bulge &  1.0 &  0.96 &  0.71 \\
%	M31 disk &  1.0 &  0.91 &  0.82 \\
%	M31 bulge &  1.0 &  0.95 &  0.83 \\
%	\enddata
%\end{deluxetable}
%
%The mutual inclination angles of the major axes are shown in Table 5. Again, the former disk particles are seen to retain a distinct memory of their origin. 
%
%\begin{deluxetable}{crrrrr}[htb!]
%	\tablecaption{Mutual inclination angles of major axis by particle origin (degrees). Suffix indicates source: \textbf{d}isk/\textbf{b}ulge.
%		\label{tbl:remn_axis_angles}}
%	\tablewidth{0pt}
%	\tablehead{
%		\colhead{} & \colhead{Total} & \colhead{MWd} & \colhead{MWb} & \colhead{M31d} & \colhead{M31b} 
%	}
%	\startdata
%	total &  -- &  39.9 &  26.5 &  13.7 &  21.6 \\
%	MWd &   39.9 &   -- &  15.7 &  50.7 &  19.9 \\
%	MWb &   26.5 &  15.7 &   -- &  39.2 &   4.9 \\
%	M31d &   13.7 &  50.7 &  39.2 &   -- &  34.4 \\
%	M31b &   21.6 &  19.9 &   4.9 &  34.4 &  -- \\
%	\enddata
%\end{deluxetable}

Coordinates were rotated to place the eigenvector corresponding to the major axis along the $z$-axis. By chance, this left the other eigenvectors within $7^\circ$ of the $x$- and $y$-axes. Orthogonal-view density plots are shown in Figure \ref{fig:rem_tensor}. In the mid and right panels the long axis of the density contours should be vertical, but this is clearly not the case. As \citet{jing_triaxial_2002} point out, the moment inertia tensor is sensitive to the outer boundary of the distribution. The 40 kpc spherical boundary used here is clearly too simplistic. It even appears from Figure \ref{fig:rem_tensor} that ellipticity varies with radius and is highest near the center; also that orientation varies with radius, with the outer contours having the long axis more vertical.

\begin{figure}[htb!]
	\epsscale{1.1}
	\plotone{remnant_shape_tensor.pdf}
	\caption{Density plot, oriented with the presumed major axis along $z$..
		\label{fig:rem_tensor}}
\end{figure}

This topic will be revisited after the discussion of angular momentum in a later section.

\subsubsection{Baryonic mass distribution}

\begin{figure}[htb!]
	\epsscale{0.8}
	\plotone{remnant_mp_types.pdf}
	\caption{Mass profiles of the remnant by particle type.
		\label{fig:rem_mp_type}}
\end{figure}

The mass profile for each type of particle and overall is shown in Figure \ref{fig:rem_mp_type}. This is similar to Figure \ref{fig:massprof0}, which showed the precursor galaxies, except that the radius now extends out to 100 kpc to capture the more dispersed stellar distribution in the remnant.

\begin{figure}[htb!]
	\epsscale{0.8}
	\plotone{remnant_mp_origins.pdf}
	\caption{Mass profiles of the remnant by origin.
	\label{fig:rem_mp_origin}}
\end{figure}

\begin{figure}[htb!]
	\epsscale{0.8}
	\plotone{remnant_mp_halo.pdf}
	\caption{Mass profiles of the outer part of the remnant halo.
		\label{fig:rem_mp_halo}}
\end{figure}

Previous sections have shown that MW and M31 particles remain somewhat distinct after merger. Figure \ref{fig:rem_mp_origin} compares their mass profiles. For baryonic particles, ex-M31 masses are higher than ex-MW at most radii, reflecting the higher baryonic mass fraction in M31. The opposite effect might be expected for the DM halo, whose total mass was significantly higher in the MW (Table \ref{tbl:aggmass}), but this is not seen out to 100 kpc. Figure \ref{fig:rem_mp_halo} suggests we have to go almost 1 Mpc out before MW particles become the largest halo component.

\todo{\textbf{Does stellar density profile agree with predictions for elliptical galaxies?}}

\begin{figure}[htb!]
	\epsscale{0.9}
	\plotone{remnant_lum_rho.pdf}
	\caption{Spherically-averaged density profiles of the remnant luminous matter by origin.
		\label{fig:rem_lum_rho}}
\end{figure}

\begin{figure}[htb!]
	\epsscale{0.9}
	\plotone{remnant_DM_rho.pdf}
	\caption{Spherically-averaged density profiles of the remnant halo by origin. Note the expanded $x$-axis relative to Figure \ref{fig:rem_lum_rho}
		\label{fig:rem_DM_rho}}
\end{figure}

\begin{figure}[htb!]
	\epsscale{0.9}
	\plotone{remnant_sersic_origins.pdf}
	\caption{Surface brightness profiles of the remnant by origin.
		\label{fig:rem_sersic_origin}}
\end{figure}

Figure \ref{fig:rem_sersic_origin} shows Sérsic fits to the surface brightness profile (assuming $M_\star/L \sim 1$). \todo{comment!}

The mass profile of DM particles in the remnant halo is well fit by a Hernquist profile, but there are some differences depending on origin as shown in Table \ref{tbl:remn_hq}

\begin{deluxetable}{cc}[htb!]
	\tablecaption{Best-fit Hernquist $a$ for remnant halo
		\label{tbl:remn_hq}}
	\tablewidth{0pt}
	\tablehead{
		\colhead{Origin} & \colhead{$a$ $\pm$ StdDev (kpc)}  
	}
	\startdata
	total &  $84.5 \pm 0.5$  \\
	ex-MW &   $95.2 \pm 1.4$  \\
	ex-M31 &   $82.3 \pm 0.9$
	\enddata
\end{deluxetable}

For comparison, the $a$ values for precursor galaxies at $t=0$ were substantially smaller:
\begin{align*}
	\text{MW} &= 61.6 \pm 0.5\ kpc\\
	\text{M31} &= 61.4 \pm 0.2\ kpc
\end{align*}
The overall time-dependence of Hernquist radii was shown in Figure \ref{fig:hernquist_a}.


\subsubsection{Angular momentum}

%MW/M31 Galaxy Major Merger Remnant: Stellar disk particle kinematics
%1. Is the stellar MW/M31 merged remnant rotating ? (create a phase diagram:
%velocity vs radius). Is it a fast or slow rotator?
%2. What is the contribution of the MW vs. M31 to the kinematics of the remnant?
%3. What is the velocity dispersion of the remnant as a function of radius
%4. Does the virial theorem work to return the total mass (stars + dark matter) of the
%remnant? Recall Lab 5 (dwarf vs. globular cluster based on velocity dispersion)
%5. Does the remnant sit on the fundamental plane?
%6. What is the specific angular momentum of the stellar remnant? does it line up
%with the dark matter halo specific angular momentum?
%7. Look at several snapshots at different points in time after the system has coalesced
%to see if the results change over time.
%8. Can ”dry” mergers create ellipticals? or is the remnant closer to a lenticular,
%(large bulge with rotating disk component) ?
%9. Could the major merger remnant be an S0 type galaxy?
%Relevant papers:
%Romanowsky+2003, Science 301, 1696
%Cox + 2006, ApJ 650
%Querejeta + 2015 A&A 573 (connection to S0 galaxies)
%Hopkins+2008 ApJS 175

The specific angular momentum $\vec{h}$ was calculated for all the particles in the merger remnant and various subsets, as shown in Table \ref{tbl:remn_h}. Differences tend to be small for stellar particles regardless of origin, larger for the ex-M31 halo and much larger for the ex-MW halo.

\begin{deluxetable}{lrrrr}[htb!]
	\tablecaption{Specific angular momentum components for the merger remnant at t=11.44 Gyr (kpc$^2$ / Myr)
		\label{tbl:remn_h}}
	\tablewidth{0pt}
	\tablehead{
		\colhead{} & \colhead{$\hat{h}_x$} & \colhead{$\hat{h}_y$} & \colhead{$\hat{h}_z$} & \colhead{$|h|$}  
	}
	\startdata
		total &   0.64 &  0.03 & -0.77 &  12.77 \\
		MW disk &   0.65 & -0.13 & -0.75 &   6.43 \\
		M31 disk &   0.53 & -0.21 & -0.82 &   6.28 \\
		MW bulge &   0.62 & -0.05 & -0.78 &   6.80 \\
		M31 bulge &   0.61 & -0.12 & -0.78 &   6.22 \\
		MW halo &  0.66 &  0.09 & -0.74 &  16.89 \\
		M31 halo &   0.60 & -0.08 & -0.80 &   9.42 \\
	\enddata
\end{deluxetable}

A previous section suggested that the remnant is elliptical but it is complex to define a precise shape from the mass distribution. An analysis of the angular momentum vector may provide an alternative approach.

Looking at subgroups of particle by origin, the mutual inclination angles are non-zero but generally quite small, as shown in Table \ref{tbl:remn_rot_angle}.

\begin{deluxetable}{lrrrrrrr}[htb!]
	\tablecaption{Mutual inclination angles for rotation vectors in the merger remnant at t=11.44 Gyr (degrees)
		\label{tbl:remn_rot_angle}}
	\tablewidth{0pt}
	\tablehead{
		\colhead{} & \colhead{total} & \colhead{MWd} & \colhead{M31d} & \colhead{MWb}  & \colhead{M31b}  & \colhead{MWh}  & \colhead{M31h}  
	}
	\startdata
	total &    -- &   9.0 &  15.5 &   4.5 &   8.9 &   4.2 &   6.9 \\
	MWd &    9.0 &   -- &   9.2 &   5.4 &   3.2 &  12.7 &   4.9 \\
	M31d &   15.5 &   9.2 &   -- &  11.0 &   7.0 &  19.7 &   8.7 \\
	MWb &    4.5 &   5.4 &  11.0 &   -- &   4.5 &   8.7 &   2.4 \\
	M31d &    8.9 &   3.2 &   7.0 &   4.5 &   -- &  13.1 &   2.7 \\
	MWh &    4.2 &  12.7 &  19.7 &   8.7 &  13.1 &   -- &  11.1 \\
	M31h &    6.9 &   4.9 &   8.7 &   2.4 &   2.7 &  11.1 &   -- \\
	\enddata
\end{deluxetable}

\begin{figure}[htb!]
	\epsscale{0.9}
	\plotone{rem_h_r.pdf}
	\caption{Stellar specific angular momentum of the MW-M31 remnant within various radii ($h$ has arbitrary units).
		\label{fig:rem_h_r}}
\end{figure}

That analysis used all particles originating from the MW and M31, regardless of distance from the CoM. To check the validity of this, we calculated how specific angular momentum varies with radius. Attempting to do this for thin spherical shells gave surprisingly noisy results with no clear interpretation. Instead, Figure \ref{fig:rem_h_r} shows values for all stars within various radii of the CoM. 

\begin{figure}[htb!]
	\epsscale{0.9}
	\plotone{rem_phi_theta.pdf}
	\caption{Stellar angular momentum orientation of the MW-M31 remnant within various radii (spherical coordinates).
		\label{fig:rem_phi_theta}}
\end{figure}

To investigate this further, Figure \ref{fig:rem_phi_theta} shows the orientation of $\hat{L}(r)$ in spherical coordinates, where $\phi$ is the azimuthal angle in the $x,y$ plane and $\theta$ is the polar angle downwards from the positive $z$-axis, as in Figure \ref{fig:sph_coord}.\footnote{\url{https://en.wikipedia.org/wiki/Spherical\_coordinate\_system\#/media/File:3D\_Spherical.svg}} 

Again, we see a substantial variation by radius: the remnant is not rotating as a solid body. With the major caveat that we are looking at spherically-averaged values for a structure which probably has substantial (but at this point undefined) ellipticity, the remnant is clearly far from equilibrium and has a complex structure and kinematics.

\begin{figure}[htb!]
	\epsscale{0.6}
	\plotone{240px-3D_Spherical.png}
	\caption{Spherical coordinate convention.
		\label{fig:sph_coord}}
\end{figure}

This collision and a single merger is not sufficient to randomize stars within the remnant. This is consistent with the understanding that relaxation times are very long in collisionless systems on the scale of elliptical galaxies \citep[Section 1.2]{binney_galactic_2008}. During the conditions of a galactic collision and merger there is an additional mechanism, called violent relaxation, that be significantly faster \citep[Section 4.10.2]{binney_galactic_2008}. Despite this, \citet{barnes_dynamics_1992} note that structure from the progenitors can survive the merging process. Also, there is observational evidence from rotation curves of elliptical galaxies that the inner and outer regions are sometimes decoupled, e.g. \citep{napolitano_ngc_2002}.

\todo{distinguish by origin}


\subsubsection{Remnant stellar kinematics}

It was shown in \citep{cox_kinematic_2006} that simulated mergers lead to substantially different rotational kinematics if the galaxies are gas-poor (``dry'') and the collision is dissipationless versus gas-rich, dissipational collisions. \todo{expand!}

Initially, the stellar particles (from all origins) were used to calculate an angular momentum vector $\hat{L}$, then the coordinate system was rotated to place this along the $z$-axis. We might have expected the principal axes of the ellipsoid to correspond in some simple way to this projection, but this was found to be far from the case.

More intuitive results were obtained once the $\hat{L}$ calculation was limited to stars within a constrained radius of the CoM.  Figure \ref{fig:rem_shape_10} uses a 10 kpc limit and shows that the major axis is approximately parallel to the rotation axis.

\begin{figure}[htb!]
	\epsscale{1.2}
	\plotone{remnant_shape_10.pdf}
	\caption{Luminous star density of the MW-M31 remnant in three orthogonal projections. Left panel looks down the $\hat{L}$ axis, mid/right panels have this vertical. Contours enclose 50\% (orange) and 75\% (red) of the stars.
		\label{fig:rem_shape_10}}
\end{figure}

In this inner region the stellar remnant is near-prolate. Figure \ref{fig:isophote_xy} shows the contour points from the left panel of Figure \ref{fig:rem_shape_10}, overlaid with an ellipse (solid line) that is an approximate best fit. Ellipticity $\epsilon$ is low in this midplane, around 0.15.

\begin{figure}[htb!]
	\epsscale{1.0}
	\plotone{isophote_xy.pdf}
	\caption{Contour fitting, plane perpendicular to $\hat{L}$ (left panel of Figure \ref{fig:rem_shape}).
		\label{fig:isophote_xy}}
\end{figure}

Figure \ref{fig:isophote_xy} shows a similar fit for the mid panel of Figure \ref{fig:rem_shape_10}. Ellipticity is much higher, around 0.5. The inner contour is also waisted and somewhat boxy; the outer contour appears more regular, but further statistical analysis is needed.

\begin{figure}[htb!]
	\epsscale{1.0}
	\plotone{isophote_xz.pdf}
	\caption{Contour fitting, plane parallel to $\hat{L}$ (mid panel of Figure \ref{fig:rem_shape}).
		\label{fig:isophote_xz}}
\end{figure}

Within the inner few kpc, the stellar remnant thus has an axis ratio of $1:0.55:0.47$, with the long axis along $\hat{z}$. 

\begin{figure}[htb!]
	\epsscale{1.2}
	\plotone{remnant_shape_40.pdf}
	\caption{Luminous star density of the MW-M31 remnant, extending Figure \ref{fig:rem_shape_10} to larger radii with the same orientation.
		\label{fig:rem_shape_40}}
\end{figure}

There may be a trend for the ellipticity to fall between the 50\% and 75\% contours, but the data so far is not convincing. If we extend the analysis to stars within 40 kpc of the CoM and add a 90\% contour as in Figure \ref{fig:rem_shape_40}, the picture changes significantly. Out at radii of 20-30 kpc, the overall shape is more oblate, with the short axis closer to $\hat{x}$ as shown in Figure \ref{fig:isophotes_90}. 

\begin{figure}[htb!]
	\epsscale{1.2}
	\plotone{isophotes_90.pdf}
	\caption{Contour fitting for the 90\% contour.
		\label{fig:isophotes_90}}
\end{figure}



\todo{\textbf{Is it a fast or slow rotator?}}

\todo{\textbf{Does the virial theorem work to return the total baryonic mass?}}

Figure \ref{fig:rem_phase} shows phase diagrams in various orientations. There is a fairly clear velocity asymmetry along the $x$- and $y$-axes, perpendicular to $\hat{L}$, but not along the $z$-axis.

\begin{figure}[htb!]
	\epsscale{1.2}
	\plotone{remnant_phase.pdf}
	\caption{Phase diagrams of the MW-M31 remnant, orthogonal views.
	\label{fig:rem_phase}}
\end{figure}

Mean radial velocities $\bar{v}$ and velocity dispersions $\sigma$ were calculated by binning along the $y$-axis (Figure \ref{fig:rem_disp_y}) and the $z$-axis (Figure \ref{fig:rem_disp_z}). The variation perpendicular to $\hat{L}$ shows clear rotation with $v_{\text{max}} \approx 68$ km/s, somewhat asymmetric. As expected, along $\hat{L}$ we see smaller and more random velocities with dispersion falling off more slowly from the center.

\begin{figure}[htb!]
	\epsscale{0.9}
	\plotone{remnant_stellar_disp_y_801.pdf}
	\caption{Velocity (red circles) and dispersion (blue +) by radius, $y$-axis: perpendicular to the angular momentum vector.
		\label{fig:rem_disp_y}}
\end{figure}

\begin{figure}[htb!]
	\epsscale{0.9}
	\plotone{remnant_stellar_disp_z_801.pdf}
	\caption{Velocity and dispersion by radius, $z$-axis: along the angular momentum vector.
		\label{fig:rem_disp_z}}
\end{figure}

Central velocity dispersion $\sigma_c$ is about 176 km/s, so the ratio $v_{\text{max}} / \sigma_c$ is 0.39.

For oblate isotropic rotators with ellipticity $\epsilon$, there is a relation \todo{Binney78 ref}
\[ (V/\sigma) = \sqrt{\epsilon / (1 - \epsilon)} \]
For the stellar remnant we have $\epsilon \approx 0.5$, so $(V/\sigma) \approx 1$.

\todo{Finish discussion - this is NOT oblate}

\subsection{Merger DM halo remnant}

\subsubsection{Angular momentum}

For the aggregate of all DM particles in the remnant at this time, the angular momentum vector has orientation $\hat{h} = \left< 0.64 ,  0.03, -0.76 \right>$, almost identical to the total for all particles (baryonic + DM) in the remnant. The magnitude $|h| = 13.1$ kpc$^2$/Myr is more than two orders of magnitude higher than the values for the individual galaxies about their respective CoM at the current epoch. 

We showed in Figure \ref{fig:halo_h} that halo angular momentum mostly arises from tidal forces during close approach and merger, and in Figure \ref{fig:hernquist_a} that MW halo particles subsequently have a significantly larger scale radius that M31 halo particles. As specific angular momentum is a product of radius and tangential velocity, it seems reasonable that we see a higher value for ex-MW Dark Matter when it tends to be at larger radius.

\begin{figure}[htb!]
	\epsscale{0.9}
	\plotone{rem_dm_h_r.pdf}
	\caption{Halo specific angular momentum of the MW-M31 remnant within various radii ($h$ has arbitrary units).
		\label{fig:rem_dm_h_r}}
\end{figure}

The radial dependency of angular momentum for stellar matter was previously shown to be complex (Figures \ref{fig:rem_h_r} and \ref{fig:rem_phi_theta}). The analysis was repeated for the DM halo. Figures \ref{fig:rem_dm_h_r} and \ref{fig:rem_dm_phi_theta} show that the irregularities are confined within about 60 kpc, where there is dense stellar matter. Outside this the profile is smooth to at least 1 Mpc.
	
\begin{figure}[htb!]
	\epsscale{0.9}
	\plotone{rem_dm_phi_theta.pdf}
	\caption{Halo angular momentum orientation of the MW-M31 remnant within various radii (spherical coordinates).
		\label{fig:rem_dm_phi_theta}}
\end{figure}

The coordinate system in Figure \ref{fig:rem_dm_phi_theta} is the same is that used for stellar particles in Figure \ref{fig:rem_phi_theta}. The range of values is strikingly different: in the remnant, the DM halo is \textit{not} co-rotating with the stars.
	
\todo{DM density, shape, ellipse fitting}	

\subsubsection{Virial radius}

The DM halo has no sharp outer edge, it just gradually fades into the inter-galactic medium (IGM). One popular convention is to use the $r_{200}$ or ``virial radius'' as a limit: the radius within which the average density is $200\times$ the cosmological critical density $\rho_c$.

For a flat LambdaCDM cosmology, we can calculate the critical density from
\[ \rho_c(t) = \frac{3 H^2(t)}{8 \pi G} \] 
Currently \citep{planck_collaboration_planck_2016} we have\footnote{Wendy Freedman might disagree (very eloquently), but let's go with these values for now.} \\
$H(0) \equiv H_0 = 67.74$ km/s/Mpc \\
$\rho_{c,0} = 127.35\ M_\sun/\text{kpc}^3$

The simulation ends more than 11 Gyr in the future, so we need a different value for $H(t)$. By then the universe will be well into a Dark Energy-dominated epoch with near-exponential expansion. Then $\dot{H}(t) \approx 0$ and $H(t)$ asymptotically approaches its final value of $H_{\infty} \approx 57$ km/s/Mpc.

This gives us a lower bound for $\rho_{c,\infty} \approx 90\ M_\sun/\text{kpc}^3$. Then $r_{200}$ is the radius within which the averaged remnant halo density falls below $1.8 \times 10^4\ M_\sun/\text{kpc}^3$. At the final timepoint, this is 
\[ r_{200} \approx 266\ \text{kpc} \]

The total virial mass, enclosed within the virial radius, is $2.4 \times 10^{12}\ M_\Sun$, of which 91\% is Dark Matter. This is only about 53\% of the precursor mass of the two galaxies. We can see from Figure \ref{fig:rem_mp_halo} that halo mass enclosed continues to rise to at least 2 Mpc radius, suggesting that the virial radius is quite a conservative limit.

For comparison, if we used the current value $\rho_{c,0}$ throughout, the virial radius at the final timepoint would be about 266 kpc and the virial mass falls to about $2.3 \times 10^{12}\ M_\Sun$. 

The precursor galaxies have, as expected, somewhat lower virial radii: 227 kpc for MW and 222 kpc for M31 at the current epoch (based on $\rho_{c,0}$), broadly in line with literature values, e.g. \citep{dehnen_velocity_2006}. Both have virial mass around $1.3 \times 10^{12}\ M_\Sun$, so the remnant ends up at about 88\% of the combined virial mass of the precursors (at constant $\rho_c$).

\todo{Some more discussion would be useful!}


\subsubsection{Remnant halo kinematics}

The literature consensus is that halo shapes are supported by anisotropic velocity dispersions, not spin \citep{frenk_dark_2012}. They can acquire angular momentum through tidal torques, as we already saw for each galaxy in Figure \ref{fig:halo_h}. This is often characterized by a dimensionless spin parameter:
\[ \lambda = \frac{J |E|^{1/2}}{G M^{5/2}} \]
where $J$ is the magnitude of the angular momentum vector, $E$ is the total energy and $M$ is the halo mass. Often, these are taken as the values inside the virial radius, ignoring DM particles lost to the IGM.

Total energy $E$ is the sum of kinetic energy $K$ and potential energy $W$. We have the mass and velocity of every particle so 
\[ K = \sum_i \frac{1}{2} m_i v_i^2 \]
Potential energy is more challenging to calculate. In general \citep[section 2.1]{binney_galactic_2008}:
\[ W = \frac{1}{2} \int d^3\, \mathbf{x}\, \rho(\mathbf{x})\, \Phi(\mathbf{x}) \]
We would need this to calculate the highly-disrupted situation shortly after collision and merger \citep[section 8.2]{binney_galactic_2008}. For simplicity, we concentrate here on the final timepoint about 5 Gyr after merger, and assume that the remnant halo is by then close to virial equilibrium. Then $E \approx -K$ and the calculation is very much easier.

Other than potential energy, relevant values for the remnant halo vary little over time. Figure \ref{fig:rem_data} shows that angular momentum, kinetic energy and virial mass remain within $\pm 20\%$ of their final value.

\begin{figure}[htb!]
	\epsscale{1.1}
	\plotone{halo_data.pdf}
	\caption{Relative values for the remnant halo over time; final timepoint = 1.
		\label{fig:rem_data}}
\end{figure}

The final timepoint of the simulation gives these results:
\begin{align*}
	J &= 7.7 \times 10^{12}\ M_\Sun kpc^2 / Myr\\
	K &= 3.0 \times 10^{65}\ \text{erg}\\
	M &= 2.0 \times 10^{12}\ M_\Sun
\end{align*}

Unfortunately, this gives a spin parameter $\lambda \approx 37$, about 2 orders of magnitude higher than expected.

\todo{\textbf{Figure out what went wrong!}}

\todo{\textbf{remnant DM $\sigma$}}

\todo{\textbf{$V_{esc}(r)$}}



%MW/M31 Halo Major Merger Remnant: Dark matter halo evolution (density / kinematics)
%1. What is the final density profile ? Is it well fit by a Hernquist profile ? Is it more
%or less concentrated than the MW or M31 before they merged?
%2. Is the 3D dark matter distribution spheroidal? or elongated like an ellipsoid?
%What do terms like prolate, oblate, or triaxial halos mean? https://astronomy.com/news/2010/
%01/astronomers-map-the-shape-of-galactic-dark-matter
%3. What are the kinematics of the dark matter halo - is it rotating? what is the
%dispersion - does the virial theorem give you the right mass?
%4. What is the distribution of dark matter particles from the M31 vs the MW? Are
%they different? are the kinematics different?
%5. what is the average specific angular momentum? Is it the same or different than
%the halos of either galaxy before they merged.
%6. What is the escape speed of the remnant as a function of radius?
%7. Where is the ”end” of the halo? How might we define this?
%Relevant Papers:
%Frenk & White 2012 Annalen der Physik 524, 507 Review article
%Abadi + 20 MNRAS, 2010 407

\section{Discussion and Conclusions}

\todo{\textbf{\textit{add some!}}}
\vspace{30mm}


\section{Acknowledgments}

The author is grateful to Professor Gurtina Besla for teaching the class on which this paper is based and for allowing this rather geriatric student to participate, as well as providing all the raw data from the earlier simulation described in vdM12. Also to Rixin Li for valuable coding advice. Final, the Astronomy majors deserve my thanks for patient and supportive interactions with a fellow student older than their parents, during the past four semesters. I wish them every success in the future!

This work relied on a range of open-source software packages, many of them sponsored by NumFOCUS\footnote{https://numfocus.org/} for the benefit of us all: 

\begin{itemize}
  \setlength\itemsep{-1mm}
	\item NumPy \citep{van_der_walt_numpy_2011}
	\item Matplotlib \citep{hunter_matplotlib_2007}
	\item pandas \citep{mckinney-proc-scipy-2010}
	\item Astropy \citep{astropy:2013}
	\item SciPy \citep{2020SciPy-NMeth}
	\item IPython \citep{perez_ipython_2007}
	\item Jupyter \citep{Kluyver:2016aa}
	\item conda-forge\footnote{https://conda-forge.org/}
\end{itemize}

 Additionally, mpl-scatter-density\footnote{https://github.com/astrofrog/mpl-scatter-density} and Plotly\footnote{https://plotly.com/python/} were used in preparing the figures.


% =====================================
% Bibliography stuff
\bibliography{project_Leach}{}
\bibliographystyle{aasjournal}

\end{document}
