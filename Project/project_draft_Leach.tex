\documentclass[twocolumn]{aastex63}

\usepackage{natbib}
\usepackage{booktabs} % helps with pandas to_latex() output
\usepackage{multirow}

\newcommand{\todo}{\color{red}{TODO}\color{black}\hspace{2mm}}

\shorttitle{ASTR400B report}
\shortauthors{Leach}

\begin{document}
	
\title{ASTR400B Project Report, Spring 2020}

\author[0000-0003-3608-1546]{Colin Leach}

\begin{abstract}
	
A summary to be inserted here

\end{abstract}

\section{Introduction}

The simulation of Milky Way--M31--M33 orbital evolution was described previously \citep{marel_m31_2012}. That paper included an extensive analysis of both N-body simulations and semi-analytic orbit integrations. The present study uses data from the same N-body simulation to carry out further computational analysis.

\section{Data}

Data from one N-body simulation in \citep{marel_m31_2012} was supplied in text-file format by one of the original authors. This included position and velocity data for each particle at the current epoch ($t=0$) and 800 future timesteps. For ease of analysis, this was all transferred to the open source database PostgreSQL\footnote{\url{http://www.postgresql.org}} (approximately 1.35 billion records). The same database was used to store computed summary data during the analysis.

Particle counts for each time point are shown in Table 1 and total masses in Table 2.

The coordinate system is approximately centered on the Milky Way at $t=0$. The center of mass (CoM) of all particles in the system is not fixed over time, moving at an average of (35.9, -26.7, 27.5) km/s with some minor fluctuations due to numerical approximations. In contrast, the total angular momentum of the system is very small at all time points.

\begin{deluxetable*}{crrrr}[htb!]
\tablenum{1}
\tablecaption{Particle counts}
\tablewidth{0pt}
\tablehead{
	\colhead{Galaxy} & \colhead{DM Halo} & \colhead{Disk} & \colhead{Bulge} & \colhead{Total} 
	%\colhead{Name} & \colhead{Count} & \colhead{Count} & \colhead{Count} & \colhead{}
}
\startdata
	MW   &  250,000 &   375,000 &    50,000 &   675,000 \\
	M31  &  250,000 &   600,000 &    95,000 &   945,000 \\
	M33  &   25,000 &    46,500 &        0 &    71,500 \\
	\midrule
	Local Group  &  525,000 &  1,021,500 &   145,000 &  1,691,500 \
\enddata
\end{deluxetable*}


\begin{deluxetable*}{crrrr}[htb!]
\tablenum{2}
\tablecaption{Aggregate masses ($M_\Sun \times 10^{12}$)}
\tablewidth{0pt}
\tablehead{
	\colhead{Galaxy} & \colhead{DM Halo} & \colhead{Disk} & \colhead{Bulge} & \colhead{Total}
}
\startdata
	MW  &      1.975 &      0.075 &       0.010 &  2.060 \\
    M31 &      1.921 &      0.120 &       0.019 &  2.060  \\
	M33 &      0.187 &      0.009 &       0.000 &  0.196  \\
	\midrule
	Local Group &   4.082 &      0.204 &       0.029 &  4.316 \\
\enddata
\end{deluxetable*}


\section{Software}

The work in this report was carried out in Python using standard package. Full details are available online\footnote{Code \url{https://github.com/colinleach/400B_Leach}\\documentation \url{https://400b-leach.readthedocs.io}}

\section{Results}

\subsection{Trajectories}



\subsection{Close approach}

\citep{toomre_galactic_1972}


\bibliography{project_Leach}{}
\bibliographystyle{aasjournal}

\end{document}
