%% Generated by Sphinx.
\def\sphinxdocclass{report}
\documentclass[letterpaper,10pt,english]{sphinxmanual}
\ifdefined\pdfpxdimen
   \let\sphinxpxdimen\pdfpxdimen\else\newdimen\sphinxpxdimen
\fi \sphinxpxdimen=.75bp\relax

\PassOptionsToPackage{warn}{textcomp}
\usepackage[utf8]{inputenc}
\ifdefined\DeclareUnicodeCharacter
% support both utf8 and utf8x syntaxes
  \ifdefined\DeclareUnicodeCharacterAsOptional
    \def\sphinxDUC#1{\DeclareUnicodeCharacter{"#1}}
  \else
    \let\sphinxDUC\DeclareUnicodeCharacter
  \fi
  \sphinxDUC{00A0}{\nobreakspace}
  \sphinxDUC{2500}{\sphinxunichar{2500}}
  \sphinxDUC{2502}{\sphinxunichar{2502}}
  \sphinxDUC{2514}{\sphinxunichar{2514}}
  \sphinxDUC{251C}{\sphinxunichar{251C}}
  \sphinxDUC{2572}{\textbackslash}
\fi
\usepackage{cmap}
\usepackage[T1]{fontenc}
\usepackage{amsmath,amssymb,amstext}
\usepackage{babel}



\usepackage{times}
\expandafter\ifx\csname T@LGR\endcsname\relax
\else
% LGR was declared as font encoding
  \substitutefont{LGR}{\rmdefault}{cmr}
  \substitutefont{LGR}{\sfdefault}{cmss}
  \substitutefont{LGR}{\ttdefault}{cmtt}
\fi
\expandafter\ifx\csname T@X2\endcsname\relax
  \expandafter\ifx\csname T@T2A\endcsname\relax
  \else
  % T2A was declared as font encoding
    \substitutefont{T2A}{\rmdefault}{cmr}
    \substitutefont{T2A}{\sfdefault}{cmss}
    \substitutefont{T2A}{\ttdefault}{cmtt}
  \fi
\else
% X2 was declared as font encoding
  \substitutefont{X2}{\rmdefault}{cmr}
  \substitutefont{X2}{\sfdefault}{cmss}
  \substitutefont{X2}{\ttdefault}{cmtt}
\fi


\usepackage[Bjarne]{fncychap}
\usepackage{sphinx}

\fvset{fontsize=\small}
\usepackage{geometry}


% Include hyperref last.
\usepackage{hyperref}
% Fix anchor placement for figures with captions.
\usepackage{hypcap}% it must be loaded after hyperref.
% Set up styles of URL: it should be placed after hyperref.
\urlstyle{same}
\addto\captionsenglish{\renewcommand{\contentsname}{Contents:}}

\usepackage{sphinxmessages}
\setcounter{tocdepth}{1}



\title{ASTR400B Leach}
\date{Mar 24, 2020}
\release{}
\author{Colin Leach}
\newcommand{\sphinxlogo}{\vbox{}}
\renewcommand{\releasename}{}
\makeindex
\begin{document}

\pagestyle{empty}
\sphinxmaketitle
\pagestyle{plain}
\sphinxtableofcontents
\pagestyle{normal}
\phantomsection\label{\detokenize{index::doc}}


This is documentation for code written during course ASTR 400B,
Theoretical Astrophysics, running at the University of Arizona’s
Steward Observatory, Spring 2020.

\sphinxstylestrong{Instructor:} Prof. Gurtina Besla,  \sphinxstylestrong{TA:} Rixin Li

\sphinxstylestrong{GitHub:} \sphinxurl{https://github.com/colinleach/400B\_Leach}
\begin{quote}\begin{description}
\item[{Warning}] \leavevmode
This is a student project. I will try to make it as professional as
possible, but let’s be realistic in our expectations.

\end{description}\end{quote}
\phantomsection\label{\detokenize{galaxy:module-galaxy.galaxy}}\index{galaxy.galaxy (module)@\spxentry{galaxy.galaxy}\spxextra{module}}

\chapter{Galaxy class}
\label{\detokenize{galaxy:galaxy-class}}\label{\detokenize{galaxy::doc}}
This will read in a data file for a given galaxy and snap, returning the
data in a variety of formats.
\index{Galaxy (class in galaxy.galaxy)@\spxentry{Galaxy}\spxextra{class in galaxy.galaxy}}

\begin{fulllineitems}
\phantomsection\label{\detokenize{galaxy:galaxy.galaxy.Galaxy}}\pysiglinewithargsret{\sphinxbfcode{\sphinxupquote{class }}\sphinxcode{\sphinxupquote{galaxy.galaxy.}}\sphinxbfcode{\sphinxupquote{Galaxy}}}{\emph{name}, \emph{snap=0}, \emph{datadir=None}, \emph{usesql=False}, \emph{ptype=None}, \emph{stride=1}}{}
A class to find, read and manipulate files for a single galaxy.
\begin{description}
\item[{Args:}] \leavevmode\begin{description}
\item[{name (str):}] \leavevmode
short name used in filename of type ‘name\_000.txt’, eg ‘MW’, ‘M31’.

\item[{snap (int):}] \leavevmode
Snap number, equivalent to time elapsed. Zero is starting conditions.

\item[{datadir (str):}] \leavevmode
Directory to search first for the required file. Optional, and a
default list of locations will be searched.

\item[{usesql (bool):}] \leavevmode
If True, data will be taken from a PostgreSQL database instead of
text files.

\item[{ptype (int):}] \leavevmode
Optional. Restrict data to this particle type, for speed. 
Only valid with usesql=True.

\item[{stride (int):}] \leavevmode
Optional. For stride=n, get every nth row in the table.
Only valid with usesql=True.

\end{description}

\item[{Class attributes:}] \leavevmode\begin{description}
\item[{filepath (\sphinxtitleref{pathlib.Path} object):}] \leavevmode
directory containing the data file

\item[{filename (str):}] \leavevmode
in \sphinxtitleref{name\_snap.txt} format, something like ‘MW\_000.txt’

\item[{data (np.ndarray):}] \leavevmode
type, mass, position\_xyz, velocity\_xyz for each particle

\end{description}

\end{description}
\index{read\_db() (galaxy.galaxy.Galaxy method)@\spxentry{read\_db()}\spxextra{galaxy.galaxy.Galaxy method}}

\begin{fulllineitems}
\phantomsection\label{\detokenize{galaxy:galaxy.galaxy.Galaxy.read_db}}\pysiglinewithargsret{\sphinxbfcode{\sphinxupquote{read\_db}}}{\emph{ptype}, \emph{stride}}{}
Get relevant data from a PostgreSQL database and format it to be 
identical to that read from test files.
\begin{description}
\item[{Args:}] \leavevmode\begin{description}
\item[{ptype (int):}] \leavevmode
Optional. Restrict data to this particle type.

\item[{stride (int):}] \leavevmode
Optional. For stride=n, get every nth row in the table.

\end{description}

\item[{Changes:}] \leavevmode
\sphinxtitleref{self.time}, \sphinxtitleref{self.particle\_count} and \sphinxtitleref{self.data} are set.

\end{description}

Returns: nothing

\end{fulllineitems}

\index{get\_filepath() (galaxy.galaxy.Galaxy method)@\spxentry{get\_filepath()}\spxextra{galaxy.galaxy.Galaxy method}}

\begin{fulllineitems}
\phantomsection\label{\detokenize{galaxy:galaxy.galaxy.Galaxy.get_filepath}}\pysiglinewithargsret{\sphinxbfcode{\sphinxupquote{get\_filepath}}}{\emph{datadir}}{}~\begin{description}
\item[{Args:}] \leavevmode
datadir (str): path to search first for the required file

\item[{Returns:}] \leavevmode
\sphinxtitleref{pathlib.Path} object. A directory containing the file.

\item[{Raises:}] \leavevmode
FileNotFoundError

\end{description}

Pretty boring housekeeping code, but may make things more resilient.

\end{fulllineitems}

\index{read\_file() (galaxy.galaxy.Galaxy method)@\spxentry{read\_file()}\spxextra{galaxy.galaxy.Galaxy method}}

\begin{fulllineitems}
\phantomsection\label{\detokenize{galaxy:galaxy.galaxy.Galaxy.read_file}}\pysiglinewithargsret{\sphinxbfcode{\sphinxupquote{read\_file}}}{}{}
Read in a datafile in np.ndarray format, store in \sphinxtitleref{self.data}.
\begin{description}
\item[{Requires:}] \leavevmode
\sphinxtitleref{self.path} and \sphinxtitleref{self.filename} are already set.

\item[{Changes:}] \leavevmode
\sphinxtitleref{self.time}, \sphinxtitleref{self.particle\_count} and \sphinxtitleref{self.data} are set.

\end{description}

Returns: nothing

\end{fulllineitems}

\index{type2name() (galaxy.galaxy.Galaxy method)@\spxentry{type2name()}\spxextra{galaxy.galaxy.Galaxy method}}

\begin{fulllineitems}
\phantomsection\label{\detokenize{galaxy:galaxy.galaxy.Galaxy.type2name}}\pysiglinewithargsret{\sphinxbfcode{\sphinxupquote{type2name}}}{\emph{particle\_type}}{}
Args: particle\_type (int): valid values are 1, 2, or 3

Returns: typename (str): ‘DM’, ‘disk’ or ‘bulge’

\end{fulllineitems}

\index{name2type() (galaxy.galaxy.Galaxy method)@\spxentry{name2type()}\spxextra{galaxy.galaxy.Galaxy method}}

\begin{fulllineitems}
\phantomsection\label{\detokenize{galaxy:galaxy.galaxy.Galaxy.name2type}}\pysiglinewithargsret{\sphinxbfcode{\sphinxupquote{name2type}}}{\emph{typename}}{}
Args: typename (str): valid values are ‘DM’, ‘disk’ or ‘bulge’

Returns: particle\_type (int): 1, 2, or 3 as used in data files

\end{fulllineitems}

\index{filter\_by\_type() (galaxy.galaxy.Galaxy method)@\spxentry{filter\_by\_type()}\spxextra{galaxy.galaxy.Galaxy method}}

\begin{fulllineitems}
\phantomsection\label{\detokenize{galaxy:galaxy.galaxy.Galaxy.filter_by_type}}\pysiglinewithargsret{\sphinxbfcode{\sphinxupquote{filter\_by\_type}}}{\emph{particle\_type}, \emph{dataset=None}}{}
Subsets the data to a single particle type.
\begin{description}
\item[{Args:}] \leavevmode
particle\_type (int): for particles, 1=DM, 2=disk, 3=bulge
dataset (array including a type column): defaults to self.data

\item[{Kwargs:}] \leavevmode
dataset (np.ndarray): optionally, a starting dataset other than self.data

\end{description}

Returns: np.ndarray: subset data

\end{fulllineitems}

\index{single\_particle\_properties() (galaxy.galaxy.Galaxy method)@\spxentry{single\_particle\_properties()}\spxextra{galaxy.galaxy.Galaxy method}}

\begin{fulllineitems}
\phantomsection\label{\detokenize{galaxy:galaxy.galaxy.Galaxy.single_particle_properties}}\pysiglinewithargsret{\sphinxbfcode{\sphinxupquote{single\_particle\_properties}}}{\emph{particle\_type=None}, \emph{particle\_num=0}}{}
Calculates distance from the origin and magnitude of the velocity.
\begin{description}
\item[{Kwargs:}] \leavevmode\begin{description}
\item[{particle\_type (int):}] \leavevmode
a subset of the data filtered by 1=DM, 2=disk, 3=bulge

\item[{particle\_num (int):}] \leavevmode
zero\sphinxhyphen{}based index to an array of particles

\end{description}

\item[{returns:}] \leavevmode\begin{description}
\item[{3\sphinxhyphen{}tuple of}] \leavevmode
Euclidean distance from origin (kpc),
Euclidean velocity magnitude (km/s),
particle mass (M\_sun)

\end{description}

\end{description}

\end{fulllineitems}

\index{all\_particle\_properties() (galaxy.galaxy.Galaxy method)@\spxentry{all\_particle\_properties()}\spxextra{galaxy.galaxy.Galaxy method}}

\begin{fulllineitems}
\phantomsection\label{\detokenize{galaxy:galaxy.galaxy.Galaxy.all_particle_properties}}\pysiglinewithargsret{\sphinxbfcode{\sphinxupquote{all\_particle\_properties}}}{\emph{particle\_type=None}, \emph{as\_table=True}}{}
Calculates distances from the origin and magnitude of the velocities
for all particles (default) or a specied particle type.
\begin{description}
\item[{Kwargs: }] \leavevmode\begin{description}
\item[{particle\_type (int):}] \leavevmode
A subset of the data filtered by 1=DM, 2=disk, 3=bulge

\item[{as\_table (boolean): Return type. }] \leavevmode
If True, an astropy QTable with units. 
If False, np.ndarrays for position and velocity

\end{description}

\item[{Returns:}] \leavevmode
QTable: 
The full list, optionally with units, optionally filtered by type.

\end{description}

\end{fulllineitems}

\index{component\_count() (galaxy.galaxy.Galaxy method)@\spxentry{component\_count()}\spxextra{galaxy.galaxy.Galaxy method}}

\begin{fulllineitems}
\phantomsection\label{\detokenize{galaxy:galaxy.galaxy.Galaxy.component_count}}\pysiglinewithargsret{\sphinxbfcode{\sphinxupquote{component\_count}}}{\emph{particle\_type=None}}{}~\begin{description}
\item[{Kwargs: particle\_type (int):}] \leavevmode
a subset of the data filtered by 1=DM, 2=disk, 3=bulge

\item[{Returns: Quantity:}] \leavevmode
The number of particles in the galaxy of this type

\end{description}

\end{fulllineitems}

\index{all\_component\_counts() (galaxy.galaxy.Galaxy method)@\spxentry{all\_component\_counts()}\spxextra{galaxy.galaxy.Galaxy method}}

\begin{fulllineitems}
\phantomsection\label{\detokenize{galaxy:galaxy.galaxy.Galaxy.all_component_counts}}\pysiglinewithargsret{\sphinxbfcode{\sphinxupquote{all\_component\_counts}}}{}{}~\begin{description}
\item[{Returns: list:}] \leavevmode
The aggregate masses of particles of each type in the galaxy
Ordered as {[}halo, disk, bulge{]}

\end{description}

\end{fulllineitems}

\index{component\_mass() (galaxy.galaxy.Galaxy method)@\spxentry{component\_mass()}\spxextra{galaxy.galaxy.Galaxy method}}

\begin{fulllineitems}
\phantomsection\label{\detokenize{galaxy:galaxy.galaxy.Galaxy.component_mass}}\pysiglinewithargsret{\sphinxbfcode{\sphinxupquote{component\_mass}}}{\emph{particle\_type=None}}{}~\begin{description}
\item[{Kwargs: particle\_type (int):}] \leavevmode
a subset of the data filtered by 1=DM, 2=disk, 3=bulge

\item[{Returns: Quantity:}] \leavevmode
The aggregate mass of all particles in the galaxy of this type

\end{description}

\end{fulllineitems}

\index{all\_component\_masses() (galaxy.galaxy.Galaxy method)@\spxentry{all\_component\_masses()}\spxextra{galaxy.galaxy.Galaxy method}}

\begin{fulllineitems}
\phantomsection\label{\detokenize{galaxy:galaxy.galaxy.Galaxy.all_component_masses}}\pysiglinewithargsret{\sphinxbfcode{\sphinxupquote{all\_component\_masses}}}{}{}~\begin{description}
\item[{Returns: list:}] \leavevmode
The aggregate masses of particles of each type in the galaxy

\end{description}

\end{fulllineitems}

\index{get\_array() (galaxy.galaxy.Galaxy method)@\spxentry{get\_array()}\spxextra{galaxy.galaxy.Galaxy method}}

\begin{fulllineitems}
\phantomsection\label{\detokenize{galaxy:galaxy.galaxy.Galaxy.get_array}}\pysiglinewithargsret{\sphinxbfcode{\sphinxupquote{get\_array}}}{}{}
Returns: all particle data in \sphinxtitleref{np.ndarray} format

Pretty superfluous in Python (which has no private class members)

\end{fulllineitems}

\index{get\_df() (galaxy.galaxy.Galaxy method)@\spxentry{get\_df()}\spxextra{galaxy.galaxy.Galaxy method}}

\begin{fulllineitems}
\phantomsection\label{\detokenize{galaxy:galaxy.galaxy.Galaxy.get_df}}\pysiglinewithargsret{\sphinxbfcode{\sphinxupquote{get\_df}}}{}{}
Returns: data as pandas dataframe

\end{fulllineitems}

\index{get\_qtable() (galaxy.galaxy.Galaxy method)@\spxentry{get\_qtable()}\spxextra{galaxy.galaxy.Galaxy method}}

\begin{fulllineitems}
\phantomsection\label{\detokenize{galaxy:galaxy.galaxy.Galaxy.get_qtable}}\pysiglinewithargsret{\sphinxbfcode{\sphinxupquote{get\_qtable}}}{}{}
Returns: data as astropy QTable, with units

\end{fulllineitems}


\end{fulllineitems}

\phantomsection\label{\detokenize{galaxies:module-galaxy.galaxies}}\index{galaxy.galaxies (module)@\spxentry{galaxy.galaxies}\spxextra{module}}

\chapter{Galaxies class}
\label{\detokenize{galaxies:galaxies-class}}\label{\detokenize{galaxies::doc}}
This stores and manipulates data for multiple galaxies and snaps.
\index{Galaxies (class in galaxy.galaxies)@\spxentry{Galaxies}\spxextra{class in galaxy.galaxies}}

\begin{fulllineitems}
\phantomsection\label{\detokenize{galaxies:galaxy.galaxies.Galaxies}}\pysiglinewithargsret{\sphinxbfcode{\sphinxupquote{class }}\sphinxcode{\sphinxupquote{galaxy.galaxies.}}\sphinxbfcode{\sphinxupquote{Galaxies}}}{\emph{names=(\textquotesingle{}MW\textquotesingle{}}, \emph{\textquotesingle{}M31\textquotesingle{}}, \emph{\textquotesingle{}M33\textquotesingle{})}, \emph{snaps=(0}, \emph{0}, \emph{0)}, \emph{datadir=None}, \emph{usesql=False}, \emph{ptype=None}, \emph{stride=1}}{}
A class to manipulate data for multiple galaxies.
\begin{description}
\item[{Kwargs:}] \leavevmode\begin{description}
\item[{names (iterable of str):}] \leavevmode
short names used in filename of type ‘name\_000.txt’, eg ‘MW’, ‘M31’.

\item[{snaps (iterable of int):}] \leavevmode
Snap number, equivalent to time elapsed. Zero is starting conditions.

\item[{datadir (str):}] \leavevmode
Directory to search first for the required file. Optional, and a
default list of locations will be searched.

\item[{usesql (bool):}] \leavevmode
If True, data will be taken from a PostgreSQL database instead of
text files.

\item[{ptype (int):}] \leavevmode
Optional. Restrict data to this particle type, for speed. 
Only valid with usesql=True.

\item[{stride (int):}] \leavevmode
Optional. For stride=n, get every nth row in the table.
Only valid with usesql=True.

\end{description}

\item[{Class attributes:}] \leavevmode\begin{description}
\item[{path (\sphinxtitleref{pathlib.Path} object):}] \leavevmode
directory (probably) containing the data files

\item[{filenames (list of str):}] \leavevmode
in \sphinxtitleref{name\_snap} format, something like ‘MW\_000’ (no extension)

\item[{galaxies (dict):}] \leavevmode
key is filename, value is the corresponding Galaxy object

\end{description}

\end{description}
\index{read\_data\_files() (galaxy.galaxies.Galaxies method)@\spxentry{read\_data\_files()}\spxextra{galaxy.galaxies.Galaxies method}}

\begin{fulllineitems}
\phantomsection\label{\detokenize{galaxies:galaxy.galaxies.Galaxies.read_data_files}}\pysiglinewithargsret{\sphinxbfcode{\sphinxupquote{read\_data\_files}}}{}{}
Attempts to create a Galaxy object for each name/snap combination
set in \sphinxtitleref{self.names} and \sphinxtitleref{self.snaps}

No return value.
Sets \sphinxtitleref{self.galaxies}, a dictionary keyed on \sphinxtitleref{name\_snap}

\end{fulllineitems}

\index{get\_pivot() (galaxy.galaxies.Galaxies method)@\spxentry{get\_pivot()}\spxextra{galaxy.galaxies.Galaxies method}}

\begin{fulllineitems}
\phantomsection\label{\detokenize{galaxies:galaxy.galaxies.Galaxies.get_pivot}}\pysiglinewithargsret{\sphinxbfcode{\sphinxupquote{get\_pivot}}}{\emph{aggfunc}, \emph{values=\textquotesingle{}m\textquotesingle{}}}{}
Generic method to make a pandas pivot table from the 9 combinations of 
galaxy and particle type.
\begin{description}
\item[{Args:}] \leavevmode
aggfunc (str): ‘count’, ‘sum’, etc as aggregation method
values (str): column name to aggregate

\end{description}

Returns: pandas dataframe

\end{fulllineitems}

\index{get\_counts\_pivot() (galaxy.galaxies.Galaxies method)@\spxentry{get\_counts\_pivot()}\spxextra{galaxy.galaxies.Galaxies method}}

\begin{fulllineitems}
\phantomsection\label{\detokenize{galaxies:galaxy.galaxies.Galaxies.get_counts_pivot}}\pysiglinewithargsret{\sphinxbfcode{\sphinxupquote{get\_counts\_pivot}}}{}{}
Pivots on \sphinxtitleref{count(‘m)}.

Returns: pandas dataframe

\end{fulllineitems}

\index{get\_masses\_pivot() (galaxy.galaxies.Galaxies method)@\spxentry{get\_masses\_pivot()}\spxextra{galaxy.galaxies.Galaxies method}}

\begin{fulllineitems}
\phantomsection\label{\detokenize{galaxies:galaxy.galaxies.Galaxies.get_masses_pivot}}\pysiglinewithargsret{\sphinxbfcode{\sphinxupquote{get\_masses\_pivot}}}{}{}
Pivots on \sphinxtitleref{sum(‘m)}.

Returns: pandas dataframe

\end{fulllineitems}

\index{get\_full\_df() (galaxy.galaxies.Galaxies method)@\spxentry{get\_full\_df()}\spxextra{galaxy.galaxies.Galaxies method}}

\begin{fulllineitems}
\phantomsection\label{\detokenize{galaxies:galaxy.galaxies.Galaxies.get_full_df}}\pysiglinewithargsret{\sphinxbfcode{\sphinxupquote{get\_full\_df}}}{}{}
Combined data for all input files.
\begin{description}
\item[{Returns:}] \leavevmode
Concatenated pandas dataframe from all galaxies
Includes ‘name’ and ‘snap’ columns

\end{description}

\end{fulllineitems}

\index{get\_coms() (galaxy.galaxies.Galaxies method)@\spxentry{get\_coms()}\spxextra{galaxy.galaxies.Galaxies method}}

\begin{fulllineitems}
\phantomsection\label{\detokenize{galaxies:galaxy.galaxies.Galaxies.get_coms}}\pysiglinewithargsret{\sphinxbfcode{\sphinxupquote{get\_coms}}}{\emph{tolerance=0.1}, \emph{ptypes=(1}, \emph{2}, \emph{3)}}{}
Center of Mass determination for all galaxies. 
Defaults to all particle types, but \sphinxtitleref{ptypes=(2,)} may be more useful.
\begin{description}
\item[{Args:}] \leavevmode
tolerance (float): convergence criterion (kpc)

\item[{Returns:}] \leavevmode
QTable with COM positions and velocities
colnames: {[}‘name’, ‘ptype’, ‘x’, ‘y’, ‘z’, ‘vx’, ‘vy’, ‘vz’, ‘R’, ‘V’{]}

\end{description}

\end{fulllineitems}

\index{separations() (galaxy.galaxies.Galaxies method)@\spxentry{separations()}\spxextra{galaxy.galaxies.Galaxies method}}

\begin{fulllineitems}
\phantomsection\label{\detokenize{galaxies:galaxy.galaxies.Galaxies.separations}}\pysiglinewithargsret{\sphinxbfcode{\sphinxupquote{separations}}}{\emph{g1}, \emph{g2}}{}
Position and velocity of galaxy g2 COM relative to g1 COM. 
Uses only disk particles for the COM determination.
\begin{description}
\item[{Args:}] \leavevmode
g1, g2 (str): galaxies matching entries in self.filenames

\item[{Returns:}] \leavevmode
Dictionary containing relative position, distance, velocities in
Cartesian and radial coordinates

\end{description}

\end{fulllineitems}

\index{total\_com() (galaxy.galaxies.Galaxies method)@\spxentry{total\_com()}\spxextra{galaxy.galaxies.Galaxies method}}

\begin{fulllineitems}
\phantomsection\label{\detokenize{galaxies:galaxy.galaxies.Galaxies.total_com}}\pysiglinewithargsret{\sphinxbfcode{\sphinxupquote{total\_com}}}{}{}
Center of Mass determination for the local group.

Uses all particles of all types. Position and velocity should be conserved 
quantities, subject to numerical imprecision in the sim.
\begin{description}
\item[{Returns:}] \leavevmode
position, velocity: 3\sphinxhyphen{}vectors

\end{description}

\end{fulllineitems}

\index{total\_angmom() (galaxy.galaxies.Galaxies method)@\spxentry{total\_angmom()}\spxextra{galaxy.galaxies.Galaxies method}}

\begin{fulllineitems}
\phantomsection\label{\detokenize{galaxies:galaxy.galaxies.Galaxies.total_angmom}}\pysiglinewithargsret{\sphinxbfcode{\sphinxupquote{total\_angmom}}}{\emph{origin}}{}
Calculate angular momentum summed over all particles in the local group,
abot point \sphinxtitleref{origin}.
\begin{description}
\item[{Arg:}] \leavevmode
origin (3\sphinxhyphen{}vector): x,y,z coordinates

\item[{Returns:}] \leavevmode
angular momentum: 3\sphinxhyphen{}vector

\end{description}

\end{fulllineitems}


\end{fulllineitems}

\phantomsection\label{\detokenize{centerofmass:module-galaxy.centerofmass}}\index{galaxy.centerofmass (module)@\spxentry{galaxy.centerofmass}\spxextra{module}}

\chapter{CenterOfMass class}
\label{\detokenize{centerofmass:centerofmass-class}}\label{\detokenize{centerofmass::doc}}
Determines position and velocity of the COM for a galaxy/particle
type combination.
\index{CenterOfMass (class in galaxy.centerofmass)@\spxentry{CenterOfMass}\spxextra{class in galaxy.centerofmass}}

\begin{fulllineitems}
\phantomsection\label{\detokenize{centerofmass:galaxy.centerofmass.CenterOfMass}}\pysiglinewithargsret{\sphinxbfcode{\sphinxupquote{class }}\sphinxcode{\sphinxupquote{galaxy.centerofmass.}}\sphinxbfcode{\sphinxupquote{CenterOfMass}}}{\emph{gal}, \emph{ptype=2}}{}
Class to define COM position and velocity properties of a given galaxy 
and simulation snapshot
\begin{description}
\item[{Args:}] \leavevmode\begin{description}
\item[{gal (Galaxy object):}] \leavevmode
The desired galaxy/snap to operate on

\item[{ptype (int):}] \leavevmode
for particles, 1=DM/halo, 2=disk, 3=bulge

\end{description}

\item[{Throws:}] \leavevmode
ValueError, if there are no particles of this type in this galaxy
(typically, halo particles in M33)

\end{description}
\index{com\_define() (galaxy.centerofmass.CenterOfMass method)@\spxentry{com\_define()}\spxextra{galaxy.centerofmass.CenterOfMass method}}

\begin{fulllineitems}
\phantomsection\label{\detokenize{centerofmass:galaxy.centerofmass.CenterOfMass.com_define}}\pysiglinewithargsret{\sphinxbfcode{\sphinxupquote{com\_define}}}{\emph{xyz}, \emph{m}}{}
Function to compute the center of mass position or velocity generically
\begin{description}
\item[{Args: }] \leavevmode\begin{description}
\item[{xyz (array with shape (3, N)):}] \leavevmode
(x, y, z) positions or velocities

\item[{m (1\sphinxhyphen{}D array):}] \leavevmode
particle masses

\end{description}

\item[{Returns: }] \leavevmode
3\sphinxhyphen{}element array, the center of mass coordinates

\end{description}

\end{fulllineitems}

\index{com\_p() (galaxy.centerofmass.CenterOfMass method)@\spxentry{com\_p()}\spxextra{galaxy.centerofmass.CenterOfMass method}}

\begin{fulllineitems}
\phantomsection\label{\detokenize{centerofmass:galaxy.centerofmass.CenterOfMass.com_p}}\pysiglinewithargsret{\sphinxbfcode{\sphinxupquote{com\_p}}}{\emph{delta=0.1}, \emph{vol\_dec=2.0}}{}
Function to specifically return the center of mass position and velocity    .
\begin{description}
\item[{Kwargs:                                                                                                           }] \leavevmode
delta (tolerance)

\item[{Returns: }] \leavevmode
One 3\sphinxhyphen{}vector, coordinates of the center of mass position (kpc)

\end{description}

\end{fulllineitems}

\index{com\_v() (galaxy.centerofmass.CenterOfMass method)@\spxentry{com\_v()}\spxextra{galaxy.centerofmass.CenterOfMass method}}

\begin{fulllineitems}
\phantomsection\label{\detokenize{centerofmass:galaxy.centerofmass.CenterOfMass.com_v}}\pysiglinewithargsret{\sphinxbfcode{\sphinxupquote{com\_v}}}{\emph{xyz\_com}}{}
Center of Mass velocity

Args: X, Y, Z positions of the COM (no units)

Returns: 3\sphinxhyphen{}Vector of COM velocities

\end{fulllineitems}

\index{center\_com() (galaxy.centerofmass.CenterOfMass method)@\spxentry{center\_com()}\spxextra{galaxy.centerofmass.CenterOfMass method}}

\begin{fulllineitems}
\phantomsection\label{\detokenize{centerofmass:galaxy.centerofmass.CenterOfMass.center_com}}\pysiglinewithargsret{\sphinxbfcode{\sphinxupquote{center\_com}}}{\emph{com\_p=None}, \emph{com\_v=None}}{}
Positions and velocities of disk particles relative to the CoM
\begin{description}
\item[{Returns}] \leavevmode{[}two (3, N) arrays{]}
CoM\sphinxhyphen{}centric position and velocity

\end{description}

\end{fulllineitems}

\index{angular\_momentum() (galaxy.centerofmass.CenterOfMass method)@\spxentry{angular\_momentum()}\spxextra{galaxy.centerofmass.CenterOfMass method}}

\begin{fulllineitems}
\phantomsection\label{\detokenize{centerofmass:galaxy.centerofmass.CenterOfMass.angular_momentum}}\pysiglinewithargsret{\sphinxbfcode{\sphinxupquote{angular\_momentum}}}{\emph{com\_p=None}, \emph{com\_v=None}}{}~\begin{description}
\item[{Returns: }] \leavevmode\begin{description}
\item[{L}] \leavevmode{[}3\sphinxhyphen{}vector as array{]}
The (x,y,x) components of the angular momentum vector about the CoM,
summed over all disk particles

\item[{pos, v}] \leavevmode{[}arrays with shape (3, N){]}
Position and velocity for each particle

\end{description}

\end{description}

\end{fulllineitems}

\index{rotate\_frame() (galaxy.centerofmass.CenterOfMass method)@\spxentry{rotate\_frame()}\spxextra{galaxy.centerofmass.CenterOfMass method}}

\begin{fulllineitems}
\phantomsection\label{\detokenize{centerofmass:galaxy.centerofmass.CenterOfMass.rotate_frame}}\pysiglinewithargsret{\sphinxbfcode{\sphinxupquote{rotate\_frame}}}{\emph{to\_axis=None}, \emph{com\_p=None}, \emph{com\_v=None}}{}~\begin{description}
\item[{Arg: to\_axis (3\sphinxhyphen{}vector)}] \leavevmode
Angular momentum vector will be aligned to this (default z\sphinxhyphen{}hat)

\item[{Returns: (positions, velocities), two arrays of shape (3, N)}] \leavevmode
New values for every particle. \sphinxtitleref{self.data} remains unchanged.

\end{description}

Based on Rodrigues’ rotation formula
Ref: \sphinxurl{https://en.wikipedia.org/wiki/Rodrigues\%27\_rotation\_formula}

\end{fulllineitems}


\end{fulllineitems}

\phantomsection\label{\detokenize{massprofile:module-galaxy.massprofile}}\index{galaxy.massprofile (module)@\spxentry{galaxy.massprofile}\spxextra{module}}

\chapter{MassProfile class}
\label{\detokenize{massprofile:massprofile-class}}\label{\detokenize{massprofile::doc}}
Calculates mass vs. radius relations and rotation curves for a given galaxy.
\index{MassProfile (class in galaxy.massprofile)@\spxentry{MassProfile}\spxextra{class in galaxy.massprofile}}

\begin{fulllineitems}
\phantomsection\label{\detokenize{massprofile:galaxy.massprofile.MassProfile}}\pysiglinewithargsret{\sphinxbfcode{\sphinxupquote{class }}\sphinxcode{\sphinxupquote{galaxy.massprofile.}}\sphinxbfcode{\sphinxupquote{MassProfile}}}{\emph{gal}, \emph{com\_p=None}}{}
Class to define mass enclosed as a function of radius and circular velocity
profiles for a given galaxy and simulation snapshot
\begin{description}
\item[{Args:}] \leavevmode\begin{description}
\item[{gal (Galaxy object):}] \leavevmode
The desired galaxy/snap to operate on

\end{description}

\end{description}
\index{mass\_enclosed() (galaxy.massprofile.MassProfile method)@\spxentry{mass\_enclosed()}\spxextra{galaxy.massprofile.MassProfile method}}

\begin{fulllineitems}
\phantomsection\label{\detokenize{massprofile:galaxy.massprofile.MassProfile.mass_enclosed}}\pysiglinewithargsret{\sphinxbfcode{\sphinxupquote{mass\_enclosed}}}{\emph{radii}, \emph{ptype=None}}{}
Calculate the mass within a given radius of the CoM 
for a given type of particle.
\begin{description}
\item[{Args:}] \leavevmode
radii (array of distances): spheres to integrate over
ptype (int): particle type from (1,2,3), or None for total

\item[{Returns:}] \leavevmode
array of masses, in units of M\_sun

\end{description}

\end{fulllineitems}

\index{mass\_enclosed\_total() (galaxy.massprofile.MassProfile method)@\spxentry{mass\_enclosed\_total()}\spxextra{galaxy.massprofile.MassProfile method}}

\begin{fulllineitems}
\phantomsection\label{\detokenize{massprofile:galaxy.massprofile.MassProfile.mass_enclosed_total}}\pysiglinewithargsret{\sphinxbfcode{\sphinxupquote{mass\_enclosed\_total}}}{\emph{radii}}{}
Calculate the mass within a given radius of the CoM, 
summed for all types of particle.
\begin{description}
\item[{Args:}] \leavevmode
radii (array of distances): spheres to integrate over

\item[{Returns:}] \leavevmode
array of masses, in units of M\_sun

\end{description}

\end{fulllineitems}

\index{halo\_mass() (galaxy.massprofile.MassProfile method)@\spxentry{halo\_mass()}\spxextra{galaxy.massprofile.MassProfile method}}

\begin{fulllineitems}
\phantomsection\label{\detokenize{massprofile:galaxy.massprofile.MassProfile.halo_mass}}\pysiglinewithargsret{\sphinxbfcode{\sphinxupquote{halo\_mass}}}{}{}
Utility function to get a parameter for Hernquist mass

\end{fulllineitems}

\index{hernquist\_mass() (galaxy.massprofile.MassProfile method)@\spxentry{hernquist\_mass()}\spxextra{galaxy.massprofile.MassProfile method}}

\begin{fulllineitems}
\phantomsection\label{\detokenize{massprofile:galaxy.massprofile.MassProfile.hernquist_mass}}\pysiglinewithargsret{\sphinxbfcode{\sphinxupquote{hernquist\_mass}}}{\emph{r}, \emph{a}, \emph{M\_halo=None}}{}
Calculate the mass enclosed for a theoretical profile
\begin{description}
\item[{Args:}] \leavevmode
r (Quantity, units of kpc): distance from center
a (Quantity, units of kpc): scale radius
M\_halo (Quantity, units of M\_sun): total DM mass (optional)

\item[{Returns:}] \leavevmode
Total DM mass enclosed within r (M\_sun)

\end{description}

\end{fulllineitems}

\index{circular\_velocity() (galaxy.massprofile.MassProfile method)@\spxentry{circular\_velocity()}\spxextra{galaxy.massprofile.MassProfile method}}

\begin{fulllineitems}
\phantomsection\label{\detokenize{massprofile:galaxy.massprofile.MassProfile.circular_velocity}}\pysiglinewithargsret{\sphinxbfcode{\sphinxupquote{circular\_velocity}}}{\emph{radii}, \emph{ptype=None}}{}
Calculate orbital velocity at a given radius from the CoM 
for a given type of particle.
\begin{description}
\item[{Args:}] \leavevmode
radii (array of distances): circular orbit
ptype (int): particle type from (1,2,3), or None for total

\item[{Returns:}] \leavevmode
array of circular speeds, in units of km/s

\end{description}

\end{fulllineitems}

\index{circular\_velocity\_total() (galaxy.massprofile.MassProfile method)@\spxentry{circular\_velocity\_total()}\spxextra{galaxy.massprofile.MassProfile method}}

\begin{fulllineitems}
\phantomsection\label{\detokenize{massprofile:galaxy.massprofile.MassProfile.circular_velocity_total}}\pysiglinewithargsret{\sphinxbfcode{\sphinxupquote{circular\_velocity\_total}}}{\emph{radii}}{}
Syntactic sugar for circular\_velocity(radii, ptype=None)

\end{fulllineitems}


\end{fulllineitems}

\phantomsection\label{\detokenize{timecourse:module-galaxy.timecourse}}\index{galaxy.timecourse (module)@\spxentry{galaxy.timecourse}\spxextra{module}}

\chapter{TimeCourse class}
\label{\detokenize{timecourse:timecourse-class}}\label{\detokenize{timecourse::doc}}
Various methods to work with data across a series of snaps (timepoints).
\index{TimeCourse (class in galaxy.timecourse)@\spxentry{TimeCourse}\spxextra{class in galaxy.timecourse}}

\begin{fulllineitems}
\phantomsection\label{\detokenize{timecourse:galaxy.timecourse.TimeCourse}}\pysiglinewithargsret{\sphinxbfcode{\sphinxupquote{class }}\sphinxcode{\sphinxupquote{galaxy.timecourse.}}\sphinxbfcode{\sphinxupquote{TimeCourse}}}{\emph{datadir=\textquotesingle{}.\textquotesingle{}}, \emph{usesql=False}}{}~\index{write\_com\_ang\_mom() (galaxy.timecourse.TimeCourse method)@\spxentry{write\_com\_ang\_mom()}\spxextra{galaxy.timecourse.TimeCourse method}}

\begin{fulllineitems}
\phantomsection\label{\detokenize{timecourse:galaxy.timecourse.TimeCourse.write_com_ang_mom}}\pysiglinewithargsret{\sphinxbfcode{\sphinxupquote{write\_com\_ang\_mom}}}{\emph{galname}, \emph{start=0}, \emph{end=801}, \emph{n=5}, \emph{show\_progress=True}}{}
Function that loops over all the desired snapshots to compute the COM pos and vel as a 
function of time.
\begin{description}
\item[{inputs:}] \leavevmode\begin{description}
\item[{galname (str):}] \leavevmode
‘MW’, ‘M31’ or ‘M33’

\item[{start, end (int):}] \leavevmode
first and last snap numbers to include

\item[{n (int):}] \leavevmode
stride length for the sequence

\item[{datadir (str):}] \leavevmode
path to the input data

\item[{show\_progress (bool):}] \leavevmode
prints each snap number as it is processed

\end{description}

\item[{returns: }] \leavevmode
Two text files saved to disk.

\end{description}

\end{fulllineitems}

\index{write\_total\_com() (galaxy.timecourse.TimeCourse method)@\spxentry{write\_total\_com()}\spxextra{galaxy.timecourse.TimeCourse method}}

\begin{fulllineitems}
\phantomsection\label{\detokenize{timecourse:galaxy.timecourse.TimeCourse.write_total_com}}\pysiglinewithargsret{\sphinxbfcode{\sphinxupquote{write\_total\_com}}}{\emph{start=0}, \emph{end=801}, \emph{n=1}, \emph{show\_progress=True}}{}
Function that loops over all the desired snapshots to compute the overall COM 
pos and vel as a function of time. Uses all particles in all galaxies.
\begin{description}
\item[{inputs:}] \leavevmode\begin{description}
\item[{start, end (int):}] \leavevmode
first and last snap numbers to include

\item[{n (int):}] \leavevmode
stride length for the sequence

\item[{show\_progress (bool):}] \leavevmode
prints each snap number as it is processed

\end{description}

\item[{output: }] \leavevmode
Text file saved to disk.

\end{description}

\end{fulllineitems}

\index{write\_total\_angmom() (galaxy.timecourse.TimeCourse method)@\spxentry{write\_total\_angmom()}\spxextra{galaxy.timecourse.TimeCourse method}}

\begin{fulllineitems}
\phantomsection\label{\detokenize{timecourse:galaxy.timecourse.TimeCourse.write_total_angmom}}\pysiglinewithargsret{\sphinxbfcode{\sphinxupquote{write\_total\_angmom}}}{\emph{start=0}, \emph{end=801}, \emph{n=1}, \emph{show\_progress=True}}{}
Function that loops over all the desired snapshots to compute the overall 
angular momentum as a function of time. Uses all particles in all galaxies.
\begin{description}
\item[{inputs:}] \leavevmode\begin{description}
\item[{start, end (int):}] \leavevmode
first and last snap numbers to include

\item[{n (int):}] \leavevmode
stride length for the sequence

\item[{show\_progress (bool):}] \leavevmode
prints each snap number as it is processed

\end{description}

\item[{output: }] \leavevmode
Text file saved to disk.

\end{description}

\end{fulllineitems}

\index{write\_vel\_disp() (galaxy.timecourse.TimeCourse method)@\spxentry{write\_vel\_disp()}\spxextra{galaxy.timecourse.TimeCourse method}}

\begin{fulllineitems}
\phantomsection\label{\detokenize{timecourse:galaxy.timecourse.TimeCourse.write_vel_disp}}\pysiglinewithargsret{\sphinxbfcode{\sphinxupquote{write\_vel\_disp}}}{\emph{galname}, \emph{start=0}, \emph{end=801}, \emph{n=1}, \emph{show\_progress=True}}{}
Function that loops over all the desired snapshots to compute the veocity dispersion
sigma as a function of time.
\begin{description}
\item[{inputs:}] \leavevmode\begin{description}
\item[{galname (str):}] \leavevmode
‘MW’, ‘M31’ or ‘M33’

\item[{start, end (int):}] \leavevmode
first and last snap numbers to include

\item[{n (int):}] \leavevmode
stride length for the sequence

\item[{datadir (str):}] \leavevmode
path to the input data

\item[{show\_progress (bool):}] \leavevmode
prints each snap number as it is processed

\end{description}

\item[{returns: }] \leavevmode
Text file saved to disk.

\end{description}

\end{fulllineitems}

\index{read\_file() (galaxy.timecourse.TimeCourse method)@\spxentry{read\_file()}\spxextra{galaxy.timecourse.TimeCourse method}}

\begin{fulllineitems}
\phantomsection\label{\detokenize{timecourse:galaxy.timecourse.TimeCourse.read_file}}\pysiglinewithargsret{\sphinxbfcode{\sphinxupquote{read\_file}}}{\emph{fullname}}{}
General method for file input. Note that the format is for summary files,
(one line per snap), not the raw per\sphinxhyphen{}particle files.

\end{fulllineitems}

\index{read\_com\_file() (galaxy.timecourse.TimeCourse method)@\spxentry{read\_com\_file()}\spxextra{galaxy.timecourse.TimeCourse method}}

\begin{fulllineitems}
\phantomsection\label{\detokenize{timecourse:galaxy.timecourse.TimeCourse.read_com_file}}\pysiglinewithargsret{\sphinxbfcode{\sphinxupquote{read\_com\_file}}}{\emph{galaxy}, \emph{datadir=\textquotesingle{}.\textquotesingle{}}}{}
Get CoM summary from file.
\begin{description}
\item[{Args:}] \leavevmode\begin{description}
\item[{galaxy (str): }] \leavevmode
‘MW’, ‘M31’, ‘M33’

\item[{datadir (str):}] \leavevmode
path to file

\end{description}

\item[{Returns:}] \leavevmode
np.array with 802 rows, one per snap

\end{description}

\end{fulllineitems}

\index{read\_angmom\_file() (galaxy.timecourse.TimeCourse method)@\spxentry{read\_angmom\_file()}\spxextra{galaxy.timecourse.TimeCourse method}}

\begin{fulllineitems}
\phantomsection\label{\detokenize{timecourse:galaxy.timecourse.TimeCourse.read_angmom_file}}\pysiglinewithargsret{\sphinxbfcode{\sphinxupquote{read\_angmom\_file}}}{\emph{galaxy}, \emph{datadir=\textquotesingle{}.\textquotesingle{}}}{}
Get CoM summary from file.
\begin{description}
\item[{Args:}] \leavevmode\begin{description}
\item[{galaxy (str): }] \leavevmode
‘MW’, ‘M31’, ‘M33’

\item[{datadir (str):}] \leavevmode
path to file

\end{description}

\item[{Returns:}] \leavevmode
np.array with 802 rows, one per snap

\end{description}

\end{fulllineitems}

\index{read\_total\_com\_file() (galaxy.timecourse.TimeCourse method)@\spxentry{read\_total\_com\_file()}\spxextra{galaxy.timecourse.TimeCourse method}}

\begin{fulllineitems}
\phantomsection\label{\detokenize{timecourse:galaxy.timecourse.TimeCourse.read_total_com_file}}\pysiglinewithargsret{\sphinxbfcode{\sphinxupquote{read\_total\_com\_file}}}{\emph{galaxy}, \emph{datadir=\textquotesingle{}.\textquotesingle{}}}{}
Get CoM summary from file.
\begin{description}
\item[{Args:}] \leavevmode\begin{description}
\item[{galaxy (str): }] \leavevmode
‘MW’, ‘M31’, ‘M33’

\item[{datadir (str):}] \leavevmode
path to file

\end{description}

\item[{Returns:}] \leavevmode
np.array with 802 rows, one per snap

\end{description}

\end{fulllineitems}

\index{write\_db\_tables() (galaxy.timecourse.TimeCourse method)@\spxentry{write\_db\_tables()}\spxextra{galaxy.timecourse.TimeCourse method}}

\begin{fulllineitems}
\phantomsection\label{\detokenize{timecourse:galaxy.timecourse.TimeCourse.write_db_tables}}\pysiglinewithargsret{\sphinxbfcode{\sphinxupquote{write\_db\_tables}}}{\emph{datadir=\textquotesingle{}.\textquotesingle{}}, \emph{do\_com=False}, \emph{do\_angmom=False}, \emph{do\_totalcom=False}, \emph{do\_totalangmom=False}}{}
Adds data to the \sphinxtitleref{centerofmass}, \sphinxtitleref{angmom} and \sphinxtitleref{totalcom} tables in the 
\sphinxtitleref{galaxy} database

\end{fulllineitems}

\index{read\_com\_db() (galaxy.timecourse.TimeCourse method)@\spxentry{read\_com\_db()}\spxextra{galaxy.timecourse.TimeCourse method}}

\begin{fulllineitems}
\phantomsection\label{\detokenize{timecourse:galaxy.timecourse.TimeCourse.read_com_db}}\pysiglinewithargsret{\sphinxbfcode{\sphinxupquote{read\_com\_db}}}{\emph{galaxy=None}, \emph{snaprange=(0}, \emph{801)}}{}
Retrieves CoM positions from postgres for a range of snaps.
\begin{description}
\item[{Args:}] \leavevmode\begin{description}
\item[{galaxy (str):}] \leavevmode
Optional, defaults to all. Can be ‘MW, ‘M31 , ‘M33

\item[{snaprange (pair of ints):}] \leavevmode
Optional, defaults to all. First and last snap to include.
This is NOT the {[}first, last+1{]} convention of Python.

\end{description}

\end{description}

\end{fulllineitems}

\index{read\_angmom\_db() (galaxy.timecourse.TimeCourse method)@\spxentry{read\_angmom\_db()}\spxextra{galaxy.timecourse.TimeCourse method}}

\begin{fulllineitems}
\phantomsection\label{\detokenize{timecourse:galaxy.timecourse.TimeCourse.read_angmom_db}}\pysiglinewithargsret{\sphinxbfcode{\sphinxupquote{read\_angmom\_db}}}{\emph{galaxy=None}, \emph{snaprange=(0}, \emph{801)}}{}
Retrieves disk angular momentum from postgres for a range of snaps.
\begin{description}
\item[{Args:}] \leavevmode\begin{description}
\item[{galaxy (str):}] \leavevmode
Optional, defaults to all. Can be ‘MW, ‘M31 , ‘M33’

\item[{snaprange (pair of ints):}] \leavevmode
Optional, defaults to all. First and last snap to include.
This is NOT the {[}first, last+1{]} convention of Python.

\end{description}

\end{description}

\end{fulllineitems}

\index{read\_total\_com\_db() (galaxy.timecourse.TimeCourse method)@\spxentry{read\_total\_com\_db()}\spxextra{galaxy.timecourse.TimeCourse method}}

\begin{fulllineitems}
\phantomsection\label{\detokenize{timecourse:galaxy.timecourse.TimeCourse.read_total_com_db}}\pysiglinewithargsret{\sphinxbfcode{\sphinxupquote{read\_total\_com\_db}}}{\emph{snaprange=(0}, \emph{801)}}{}
Retrieves total CoM positions from postgres for a range of snaps.
\begin{description}
\item[{Args:}] \leavevmode\begin{description}
\item[{snaprange (pair of ints):}] \leavevmode
Optional, defaults to all. First and last snap to include.
This is NOT the {[}first, last+1{]} convention of Python.

\end{description}

\end{description}

\end{fulllineitems}

\index{get\_one\_com() (galaxy.timecourse.TimeCourse method)@\spxentry{get\_one\_com()}\spxextra{galaxy.timecourse.TimeCourse method}}

\begin{fulllineitems}
\phantomsection\label{\detokenize{timecourse:galaxy.timecourse.TimeCourse.get_one_com}}\pysiglinewithargsret{\sphinxbfcode{\sphinxupquote{get\_one\_com}}}{\emph{gal}, \emph{snap}}{}
Gets a CoM from postgres for the specified galaxy and snap.
\begin{description}
\item[{Args:}] \leavevmode\begin{description}
\item[{gal (str): }] \leavevmode
Can be ‘MW, ‘M31 , ‘M33’

\item[{snap (int):}] \leavevmode
The timepoint.

\end{description}

\end{description}

\end{fulllineitems}

\index{read\_total\_angmom\_db() (galaxy.timecourse.TimeCourse method)@\spxentry{read\_total\_angmom\_db()}\spxextra{galaxy.timecourse.TimeCourse method}}

\begin{fulllineitems}
\phantomsection\label{\detokenize{timecourse:galaxy.timecourse.TimeCourse.read_total_angmom_db}}\pysiglinewithargsret{\sphinxbfcode{\sphinxupquote{read\_total\_angmom\_db}}}{\emph{snaprange=(0}, \emph{801)}}{}
\end{fulllineitems}


\end{fulllineitems}



\chapter{utilities module}
\label{\detokenize{utilities:utilities-module}}\label{\detokenize{utilities::doc}}
A collection of useful functions that don’t fit into any of the other classes.

\phantomsection\label{\detokenize{utilities:module-galaxy.utilities}}\index{galaxy.utilities (module)@\spxentry{galaxy.utilities}\spxextra{module}}\index{wolf\_mass() (in module galaxy.utilities)@\spxentry{wolf\_mass()}\spxextra{in module galaxy.utilities}}

\begin{fulllineitems}
\phantomsection\label{\detokenize{utilities:galaxy.utilities.wolf_mass}}\pysiglinewithargsret{\sphinxcode{\sphinxupquote{galaxy.utilities.}}\sphinxbfcode{\sphinxupquote{wolf\_mass}}}{\emph{sigma}, \emph{Re}}{}
Wolf mass estimator from Wolf+ 2010
\begin{description}
\item[{Args:}] \leavevmode\begin{description}
\item[{sigma}] \leavevmode{[}{]}
1D line of sight velocity dispersion in km/s

\item[{Re}] \leavevmode{[}{]}
2D radius enclosing half the stellar mass in pc

\end{description}

\end{description}

Returns: estimate of the dynamical mass within the half light radius in Msun

\end{fulllineitems}

\index{sersic() (in module galaxy.utilities)@\spxentry{sersic()}\spxextra{in module galaxy.utilities}}

\begin{fulllineitems}
\phantomsection\label{\detokenize{utilities:galaxy.utilities.sersic}}\pysiglinewithargsret{\sphinxcode{\sphinxupquote{galaxy.utilities.}}\sphinxbfcode{\sphinxupquote{sersic}}}{\emph{R}, \emph{Re}, \emph{n}, \emph{Mtot}}{}~\begin{description}
\item[{Input}] \leavevmode\begin{description}
\item[{R:}] \leavevmode
radius (kpc)

\item[{Re:}] \leavevmode
half mass radius (kpc)

\item[{n:}] \leavevmode
sersic index

\item[{Mtot:}] \leavevmode
total stellar mass

\end{description}

\item[{Returns}] \leavevmode
Surface Brightness profile in Lsun/kpc\textasciicircum{}2

\end{description}

\end{fulllineitems}

\phantomsection\label{\detokenize{db:module-galaxy.db}}\index{galaxy.db (module)@\spxentry{galaxy.db}\spxextra{module}}

\chapter{DB class}
\label{\detokenize{db:db-class}}\label{\detokenize{db::doc}}
A wrapper for connections to the PostgreSQL database
\index{DB (class in galaxy.db)@\spxentry{DB}\spxextra{class in galaxy.db}}

\begin{fulllineitems}
\phantomsection\label{\detokenize{db:galaxy.db.DB}}\pysigline{\sphinxbfcode{\sphinxupquote{class }}\sphinxcode{\sphinxupquote{galaxy.db.}}\sphinxbfcode{\sphinxupquote{DB}}}
A simple wrapper class for connecting to the PostgreSQL database.

Takes no arguments. Relies on having connection information in
\sphinxtitleref{\textasciitilde{}/dbconn.yaml}.
\index{read\_params() (galaxy.db.DB method)@\spxentry{read\_params()}\spxextra{galaxy.db.DB method}}

\begin{fulllineitems}
\phantomsection\label{\detokenize{db:galaxy.db.DB.read_params}}\pysiglinewithargsret{\sphinxbfcode{\sphinxupquote{read\_params}}}{}{}
Needs the yaml parameter file to be in the user’s home directory

\end{fulllineitems}

\index{get\_cursor() (galaxy.db.DB method)@\spxentry{get\_cursor()}\spxextra{galaxy.db.DB method}}

\begin{fulllineitems}
\phantomsection\label{\detokenize{db:galaxy.db.DB.get_cursor}}\pysiglinewithargsret{\sphinxbfcode{\sphinxupquote{get\_cursor}}}{}{}
A simple getter method

\end{fulllineitems}

\index{run\_query() (galaxy.db.DB method)@\spxentry{run\_query()}\spxextra{galaxy.db.DB method}}

\begin{fulllineitems}
\phantomsection\label{\detokenize{db:galaxy.db.DB.run_query}}\pysiglinewithargsret{\sphinxbfcode{\sphinxupquote{run\_query}}}{\emph{query}}{}
Runs a SQL query (typically SELECT)

Returns results in Python list format 
(not numpy, which would need a dtype list)

\end{fulllineitems}


\end{fulllineitems}



\chapter{Indices and tables}
\label{\detokenize{index:indices-and-tables}}\begin{itemize}
\item {} 
\DUrole{xref,std,std-ref}{genindex}

\item {} 
\DUrole{xref,std,std-ref}{modindex}

\item {} 
\DUrole{xref,std,std-ref}{search}

\end{itemize}


\renewcommand{\indexname}{Python Module Index}
\begin{sphinxtheindex}
\let\bigletter\sphinxstyleindexlettergroup
\bigletter{g}
\item\relax\sphinxstyleindexentry{galaxy.centerofmass}\sphinxstyleindexpageref{centerofmass:\detokenize{module-galaxy.centerofmass}}
\item\relax\sphinxstyleindexentry{galaxy.db}\sphinxstyleindexpageref{db:\detokenize{module-galaxy.db}}
\item\relax\sphinxstyleindexentry{galaxy.galaxies}\sphinxstyleindexpageref{galaxies:\detokenize{module-galaxy.galaxies}}
\item\relax\sphinxstyleindexentry{galaxy.galaxy}\sphinxstyleindexpageref{galaxy:\detokenize{module-galaxy.galaxy}}
\item\relax\sphinxstyleindexentry{galaxy.massprofile}\sphinxstyleindexpageref{massprofile:\detokenize{module-galaxy.massprofile}}
\item\relax\sphinxstyleindexentry{galaxy.timecourse}\sphinxstyleindexpageref{timecourse:\detokenize{module-galaxy.timecourse}}
\item\relax\sphinxstyleindexentry{galaxy.utilities}\sphinxstyleindexpageref{utilities:\detokenize{module-galaxy.utilities}}
\end{sphinxtheindex}

\renewcommand{\indexname}{Index}
\printindex
\end{document}