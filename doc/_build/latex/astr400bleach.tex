%% Generated by Sphinx.
\def\sphinxdocclass{report}
\documentclass[letterpaper,10pt,english]{sphinxmanual}
\ifdefined\pdfpxdimen
   \let\sphinxpxdimen\pdfpxdimen\else\newdimen\sphinxpxdimen
\fi \sphinxpxdimen=.75bp\relax

\PassOptionsToPackage{warn}{textcomp}
\usepackage[utf8]{inputenc}
\ifdefined\DeclareUnicodeCharacter
% support both utf8 and utf8x syntaxes
  \ifdefined\DeclareUnicodeCharacterAsOptional
    \def\sphinxDUC#1{\DeclareUnicodeCharacter{"#1}}
  \else
    \let\sphinxDUC\DeclareUnicodeCharacter
  \fi
  \sphinxDUC{00A0}{\nobreakspace}
  \sphinxDUC{2500}{\sphinxunichar{2500}}
  \sphinxDUC{2502}{\sphinxunichar{2502}}
  \sphinxDUC{2514}{\sphinxunichar{2514}}
  \sphinxDUC{251C}{\sphinxunichar{251C}}
  \sphinxDUC{2572}{\textbackslash}
\fi
\usepackage{cmap}
\usepackage[T1]{fontenc}
\usepackage{amsmath,amssymb,amstext}
\usepackage{babel}



\usepackage{times}
\expandafter\ifx\csname T@LGR\endcsname\relax
\else
% LGR was declared as font encoding
  \substitutefont{LGR}{\rmdefault}{cmr}
  \substitutefont{LGR}{\sfdefault}{cmss}
  \substitutefont{LGR}{\ttdefault}{cmtt}
\fi
\expandafter\ifx\csname T@X2\endcsname\relax
  \expandafter\ifx\csname T@T2A\endcsname\relax
  \else
  % T2A was declared as font encoding
    \substitutefont{T2A}{\rmdefault}{cmr}
    \substitutefont{T2A}{\sfdefault}{cmss}
    \substitutefont{T2A}{\ttdefault}{cmtt}
  \fi
\else
% X2 was declared as font encoding
  \substitutefont{X2}{\rmdefault}{cmr}
  \substitutefont{X2}{\sfdefault}{cmss}
  \substitutefont{X2}{\ttdefault}{cmtt}
\fi


\usepackage[Bjarne]{fncychap}
\usepackage{sphinx}

\fvset{fontsize=\small}
\usepackage{geometry}


% Include hyperref last.
\usepackage{hyperref}
% Fix anchor placement for figures with captions.
\usepackage{hypcap}% it must be loaded after hyperref.
% Set up styles of URL: it should be placed after hyperref.
\urlstyle{same}
\addto\captionsenglish{\renewcommand{\contentsname}{Contents:}}

\usepackage{sphinxmessages}
\setcounter{tocdepth}{1}



\title{ASTR400B Leach}
\date{Jan 26, 2020}
\release{}
\author{Colin Leach}
\newcommand{\sphinxlogo}{\vbox{}}
\renewcommand{\releasename}{}
\makeindex
\begin{document}

\pagestyle{empty}
\sphinxmaketitle
\pagestyle{plain}
\sphinxtableofcontents
\pagestyle{normal}
\phantomsection\label{\detokenize{index::doc}}


This is documentation for code written during course ASTR 400B,
Theoretical Astrophysics, running at the University of Arizona’s
Steward Observatory, Spring 2020.

\sphinxstylestrong{Instructor:} Prof. Gurtina Besla,  \sphinxstylestrong{TA:} Rixin Li

\sphinxstylestrong{GitHub:} \sphinxurl{https://github.com/colinleach/400B\_Leach}
\phantomsection\label{\detokenize{galaxy:module-galaxy.galaxy}}\index{galaxy.galaxy (module)@\spxentry{galaxy.galaxy}\spxextra{module}}

\chapter{Galaxy class}
\label{\detokenize{galaxy:galaxy-class}}\label{\detokenize{galaxy::doc}}
This will read in a data file for a given galaxy and snap, returning the
data in a variety of formats.
\index{Galaxy (class in galaxy.galaxy)@\spxentry{Galaxy}\spxextra{class in galaxy.galaxy}}

\begin{fulllineitems}
\phantomsection\label{\detokenize{galaxy:galaxy.galaxy.Galaxy}}\pysiglinewithargsret{\sphinxbfcode{\sphinxupquote{class }}\sphinxcode{\sphinxupquote{galaxy.galaxy.}}\sphinxbfcode{\sphinxupquote{Galaxy}}}{\emph{name}, \emph{snap=0}, \emph{datadir=None}}{}~\begin{description}
\item[{Args: }] \leavevmode\begin{description}
\item[{name (str): }] \leavevmode
short name used in filename of type ‘name\_000.txt’, eg ‘MW’, ‘M31’.

\end{description}

\item[{Kwargs:}] \leavevmode\begin{description}
\item[{snap (int): }] \leavevmode
Snap number, equivalent to time elapsed. Zero is starting conditions.

\item[{datadir (str):}] \leavevmode
Directory to search first for the required file. Optional, and a 
default list of locations will be searched.

\end{description}

\item[{Class attributes:}] \leavevmode\begin{description}
\item[{path (\sphinxtitleref{pathlib.Path} object): }] \leavevmode
directory containing the data file

\item[{filename (str):}] \leavevmode
in \sphinxtitleref{name\_snap.txt} format, something like ‘MW\_000.txt’

\item[{data (np.ndarray):}] \leavevmode
type, mass, position\_xyz, velocity\_xyz for each particle

\end{description}

\end{description}

A class to find, read and manipulate files for a single galaxy.
\index{get\_filepath() (galaxy.galaxy.Galaxy method)@\spxentry{get\_filepath()}\spxextra{galaxy.galaxy.Galaxy method}}

\begin{fulllineitems}
\phantomsection\label{\detokenize{galaxy:galaxy.galaxy.Galaxy.get_filepath}}\pysiglinewithargsret{\sphinxbfcode{\sphinxupquote{get\_filepath}}}{\emph{datadir}}{}~\begin{description}
\item[{Args:}] \leavevmode
datadir (str): path to search first for the required file

\item[{Returns:   }] \leavevmode
\sphinxtitleref{pathlib.Path} object. A directory containing the file.

\item[{Raises:}] \leavevmode
FileNotFoundError

\end{description}

Pretty boring housekeeping code, but may make things more resilient.

\end{fulllineitems}

\index{read\_file() (galaxy.galaxy.Galaxy method)@\spxentry{read\_file()}\spxextra{galaxy.galaxy.Galaxy method}}

\begin{fulllineitems}
\phantomsection\label{\detokenize{galaxy:galaxy.galaxy.Galaxy.read_file}}\pysiglinewithargsret{\sphinxbfcode{\sphinxupquote{read\_file}}}{}{}
Read in a datafile in np.ndarray format, store in \sphinxtitleref{self.data}.
\begin{description}
\item[{Requires:}] \leavevmode
\sphinxtitleref{self.path} and \sphinxtitleref{self.filename} are already set.

\item[{Returns: }] \leavevmode
nothing

\end{description}

\end{fulllineitems}

\index{filter\_by\_type() (galaxy.galaxy.Galaxy method)@\spxentry{filter\_by\_type()}\spxextra{galaxy.galaxy.Galaxy method}}

\begin{fulllineitems}
\phantomsection\label{\detokenize{galaxy:galaxy.galaxy.Galaxy.filter_by_type}}\pysiglinewithargsret{\sphinxbfcode{\sphinxupquote{filter\_by\_type}}}{\emph{type}, \emph{dataset=None}}{}~\begin{description}
\item[{Args:}] \leavevmode
type (int): for particles, 1=DM, 2=disk, 3=bulge

\item[{Kwargs:}] \leavevmode
dataset (np.ndarray): a starting dataset other than self.data

\item[{Returns: }] \leavevmode
np.ndarray: subset data

\end{description}

\end{fulllineitems}

\index{single\_particle\_properties() (galaxy.galaxy.Galaxy method)@\spxentry{single\_particle\_properties()}\spxextra{galaxy.galaxy.Galaxy method}}

\begin{fulllineitems}
\phantomsection\label{\detokenize{galaxy:galaxy.galaxy.Galaxy.single_particle_properties}}\pysiglinewithargsret{\sphinxbfcode{\sphinxupquote{single\_particle\_properties}}}{\emph{type=None}, \emph{particle\_num=0}}{}~\begin{description}
\item[{Kwargs:}] \leavevmode\begin{description}
\item[{type (int): }] \leavevmode
a subset of the data filtered by 1=DM, 2=disk, 3=bulge

\item[{particle\_num (int): }] \leavevmode
zero\sphinxhyphen{}based index to an array of particles

\end{description}

\item[{returns: }] \leavevmode\begin{description}
\item[{3\sphinxhyphen{}tuple of}] \leavevmode
Euclidean distance from CoM (kpc),
Euclidean velocity magnitude (km/s),
particle mass (M\_sun)

\end{description}

\end{description}

\end{fulllineitems}

\index{all\_particle\_properties() (galaxy.galaxy.Galaxy method)@\spxentry{all\_particle\_properties()}\spxextra{galaxy.galaxy.Galaxy method}}

\begin{fulllineitems}
\phantomsection\label{\detokenize{galaxy:galaxy.galaxy.Galaxy.all_particle_properties}}\pysiglinewithargsret{\sphinxbfcode{\sphinxupquote{all\_particle\_properties}}}{\emph{type=None}}{}~\begin{description}
\item[{Kwargs:}] \leavevmode\begin{description}
\item[{type (int): }] \leavevmode
a subset of the data filtered by 1=DM, 2=disk, 3=bulge

\end{description}

\item[{Returns:}] \leavevmode
QTable: The full list with units, optionally filtered by type.

\end{description}

\end{fulllineitems}

\index{get\_array() (galaxy.galaxy.Galaxy method)@\spxentry{get\_array()}\spxextra{galaxy.galaxy.Galaxy method}}

\begin{fulllineitems}
\phantomsection\label{\detokenize{galaxy:galaxy.galaxy.Galaxy.get_array}}\pysiglinewithargsret{\sphinxbfcode{\sphinxupquote{get\_array}}}{}{}~\begin{description}
\item[{Returns: }] \leavevmode
data in \sphinxtitleref{np.ndarray} format

\end{description}

Pretty superfluous in Python (which has no private class members)

\end{fulllineitems}

\index{get\_df() (galaxy.galaxy.Galaxy method)@\spxentry{get\_df()}\spxextra{galaxy.galaxy.Galaxy method}}

\begin{fulllineitems}
\phantomsection\label{\detokenize{galaxy:galaxy.galaxy.Galaxy.get_df}}\pysiglinewithargsret{\sphinxbfcode{\sphinxupquote{get\_df}}}{}{}
Returns:
data as pandas dataframe

\end{fulllineitems}

\index{get\_qtable() (galaxy.galaxy.Galaxy method)@\spxentry{get\_qtable()}\spxextra{galaxy.galaxy.Galaxy method}}

\begin{fulllineitems}
\phantomsection\label{\detokenize{galaxy:galaxy.galaxy.Galaxy.get_qtable}}\pysiglinewithargsret{\sphinxbfcode{\sphinxupquote{get\_qtable}}}{}{}
Returns:
data as astropy QTable, with units

\end{fulllineitems}


\end{fulllineitems}



\chapter{Indices and tables}
\label{\detokenize{index:indices-and-tables}}\begin{itemize}
\item {} 
\DUrole{xref,std,std-ref}{genindex}

\item {} 
\DUrole{xref,std,std-ref}{modindex}

\item {} 
\DUrole{xref,std,std-ref}{search}

\end{itemize}


\renewcommand{\indexname}{Python Module Index}
\begin{sphinxtheindex}
\let\bigletter\sphinxstyleindexlettergroup
\bigletter{g}
\item\relax\sphinxstyleindexentry{galaxy.galaxy}\sphinxstyleindexpageref{galaxy:\detokenize{module-galaxy.galaxy}}
\end{sphinxtheindex}

\renewcommand{\indexname}{Index}
\printindex
\end{document}