%% Generated by Sphinx.
\def\sphinxdocclass{report}
\documentclass[letterpaper,10pt,english]{sphinxmanual}
\ifdefined\pdfpxdimen
   \let\sphinxpxdimen\pdfpxdimen\else\newdimen\sphinxpxdimen
\fi \sphinxpxdimen=.75bp\relax

\PassOptionsToPackage{warn}{textcomp}
\usepackage[utf8]{inputenc}
\ifdefined\DeclareUnicodeCharacter
% support both utf8 and utf8x syntaxes
  \ifdefined\DeclareUnicodeCharacterAsOptional
    \def\sphinxDUC#1{\DeclareUnicodeCharacter{"#1}}
  \else
    \let\sphinxDUC\DeclareUnicodeCharacter
  \fi
  \sphinxDUC{00A0}{\nobreakspace}
  \sphinxDUC{2500}{\sphinxunichar{2500}}
  \sphinxDUC{2502}{\sphinxunichar{2502}}
  \sphinxDUC{2514}{\sphinxunichar{2514}}
  \sphinxDUC{251C}{\sphinxunichar{251C}}
  \sphinxDUC{2572}{\textbackslash}
\fi
\usepackage{cmap}
\usepackage[T1]{fontenc}
\usepackage{amsmath,amssymb,amstext}
\usepackage{babel}



\usepackage{times}
\expandafter\ifx\csname T@LGR\endcsname\relax
\else
% LGR was declared as font encoding
  \substitutefont{LGR}{\rmdefault}{cmr}
  \substitutefont{LGR}{\sfdefault}{cmss}
  \substitutefont{LGR}{\ttdefault}{cmtt}
\fi
\expandafter\ifx\csname T@X2\endcsname\relax
  \expandafter\ifx\csname T@T2A\endcsname\relax
  \else
  % T2A was declared as font encoding
    \substitutefont{T2A}{\rmdefault}{cmr}
    \substitutefont{T2A}{\sfdefault}{cmss}
    \substitutefont{T2A}{\ttdefault}{cmtt}
  \fi
\else
% X2 was declared as font encoding
  \substitutefont{X2}{\rmdefault}{cmr}
  \substitutefont{X2}{\sfdefault}{cmss}
  \substitutefont{X2}{\ttdefault}{cmtt}
\fi


\usepackage[Bjarne]{fncychap}
\usepackage{sphinx}

\fvset{fontsize=\small}
\usepackage{geometry}


% Include hyperref last.
\usepackage{hyperref}
% Fix anchor placement for figures with captions.
\usepackage{hypcap}% it must be loaded after hyperref.
% Set up styles of URL: it should be placed after hyperref.
\urlstyle{same}
\addto\captionsenglish{\renewcommand{\contentsname}{Contents:}}

\usepackage{sphinxmessages}
\setcounter{tocdepth}{1}



\title{ASTR400B Leach}
\date{May 22, 2020}
\release{}
\author{Colin Leach}
\newcommand{\sphinxlogo}{\vbox{}}
\renewcommand{\releasename}{}
\makeindex
\begin{document}

\pagestyle{empty}
\sphinxmaketitle
\pagestyle{plain}
\sphinxtableofcontents
\pagestyle{normal}
\phantomsection\label{\detokenize{index::doc}}


This is documentation for code written during course ASTR 400B,
Theoretical Astrophysics, running at the University of Arizona’s
Steward Observatory, Spring 2020.

\sphinxstylestrong{Instructor:} Prof. Gurtina Besla,  \sphinxstylestrong{TA:} Rixin Li

\sphinxstylestrong{GitHub:} \sphinxurl{https://github.com/colinleach/400B\_Leach}
\begin{quote}\begin{description}
\item[{Warning}] \leavevmode
This is a student project. I will try to make it as professional as
possible, but let’s be realistic in our expectations.

\end{description}\end{quote}
\phantomsection\label{\detokenize{galaxy:module-galaxy.galaxy}}\index{galaxy.galaxy (module)@\spxentry{galaxy.galaxy}\spxextra{module}}

\chapter{Galaxy class}
\label{\detokenize{galaxy:galaxy-class}}\label{\detokenize{galaxy::doc}}
This will read in a data file for a given galaxy and snap, returning the
data in a variety of formats.
\index{Galaxy (class in galaxy.galaxy)@\spxentry{Galaxy}\spxextra{class in galaxy.galaxy}}

\begin{fulllineitems}
\phantomsection\label{\detokenize{galaxy:galaxy.galaxy.Galaxy}}\pysiglinewithargsret{\sphinxbfcode{\sphinxupquote{class }}\sphinxcode{\sphinxupquote{galaxy.galaxy.}}\sphinxbfcode{\sphinxupquote{Galaxy}}}{\emph{name}, \emph{snap=0}, \emph{datadir=None}, \emph{usesql=False}, \emph{ptype=None}, \emph{stride=1}}{}
A class to find, read and manipulate files for a single galaxy.
\begin{description}
\item[{Args:}] \leavevmode\begin{description}
\item[{name (str):}] \leavevmode
short name used in filename of type ‘name\_000.txt’, eg ‘MW’, ‘M31’.

\item[{snap (int):}] \leavevmode
Snap number, equivalent to time elapsed. Zero is starting conditions.

\item[{datadir (str):}] \leavevmode
Directory to search first for the required file. Optional, and a
default list of locations will be searched.

\item[{usesql (bool):}] \leavevmode
If True, data will be taken from a PostgreSQL database instead of
text files.

\item[{ptype (int or list):}] \leavevmode
Optional. Restrict data to this particle type, for speed. 
Only valid with usesql=True.

\item[{stride (int):}] \leavevmode
Optional. For stride=n, get every nth row in the table.
Only valid with usesql=True.

\end{description}

\item[{Class attributes:}] \leavevmode\begin{description}
\item[{filepath (\sphinxtitleref{pathlib.Path} object):}] \leavevmode
directory containing the data file

\item[{filename (str):}] \leavevmode
in \sphinxtitleref{name\_snap.txt} format, something like ‘MW\_000.txt’

\item[{data (np.ndarray):}] \leavevmode
type, mass, position\_xyz, velocity\_xyz for each particle

\end{description}

\end{description}
\index{read\_db() (galaxy.galaxy.Galaxy method)@\spxentry{read\_db()}\spxextra{galaxy.galaxy.Galaxy method}}

\begin{fulllineitems}
\phantomsection\label{\detokenize{galaxy:galaxy.galaxy.Galaxy.read_db}}\pysiglinewithargsret{\sphinxbfcode{\sphinxupquote{read\_db}}}{\emph{ptype}, \emph{stride}}{}
Get relevant data from a PostgreSQL database and format it to be 
identical to that read from test files.
\begin{description}
\item[{Args:}] \leavevmode\begin{description}
\item[{ptype (int):}] \leavevmode
Optional. Restrict data to this particle type.

\item[{stride (int):}] \leavevmode
Optional. For stride=n, get every nth row in the table.

\end{description}

\item[{Changes:}] \leavevmode
\sphinxtitleref{self.time}, \sphinxtitleref{self.particle\_count} and \sphinxtitleref{self.data} are set.

\end{description}

Returns: nothing

\end{fulllineitems}

\index{get\_filepath() (galaxy.galaxy.Galaxy method)@\spxentry{get\_filepath()}\spxextra{galaxy.galaxy.Galaxy method}}

\begin{fulllineitems}
\phantomsection\label{\detokenize{galaxy:galaxy.galaxy.Galaxy.get_filepath}}\pysiglinewithargsret{\sphinxbfcode{\sphinxupquote{get\_filepath}}}{\emph{datadir}}{}~\begin{description}
\item[{Args:}] \leavevmode
datadir (str): path to search first for the required file

\item[{Returns:}] \leavevmode
\sphinxtitleref{pathlib.Path} object. A directory containing the file.

\item[{Raises:}] \leavevmode
FileNotFoundError

\end{description}

Pretty boring housekeeping code, but may make things more resilient.

\end{fulllineitems}

\index{read\_file() (galaxy.galaxy.Galaxy method)@\spxentry{read\_file()}\spxextra{galaxy.galaxy.Galaxy method}}

\begin{fulllineitems}
\phantomsection\label{\detokenize{galaxy:galaxy.galaxy.Galaxy.read_file}}\pysiglinewithargsret{\sphinxbfcode{\sphinxupquote{read\_file}}}{}{}
Read in a datafile in np.ndarray format, store in \sphinxtitleref{self.data}.
\begin{description}
\item[{Requires:}] \leavevmode
\sphinxtitleref{self.path} and \sphinxtitleref{self.filename} are already set.

\item[{Changes:}] \leavevmode
\sphinxtitleref{self.time}, \sphinxtitleref{self.particle\_count} and \sphinxtitleref{self.data} are set.

\end{description}

Returns: nothing

\end{fulllineitems}

\index{type2name() (galaxy.galaxy.Galaxy method)@\spxentry{type2name()}\spxextra{galaxy.galaxy.Galaxy method}}

\begin{fulllineitems}
\phantomsection\label{\detokenize{galaxy:galaxy.galaxy.Galaxy.type2name}}\pysiglinewithargsret{\sphinxbfcode{\sphinxupquote{type2name}}}{\emph{particle\_type}}{}
Args: particle\_type (int): valid values are 1, 2, or 3

Returns: typename (str): ‘DM’, ‘disk’ or ‘bulge’

\end{fulllineitems}

\index{name2type() (galaxy.galaxy.Galaxy method)@\spxentry{name2type()}\spxextra{galaxy.galaxy.Galaxy method}}

\begin{fulllineitems}
\phantomsection\label{\detokenize{galaxy:galaxy.galaxy.Galaxy.name2type}}\pysiglinewithargsret{\sphinxbfcode{\sphinxupquote{name2type}}}{\emph{typename}}{}
Args: typename (str): valid values are ‘DM’, ‘disk’ or ‘bulge’

Returns: particle\_type (int): 1, 2, or 3 as used in data files

\end{fulllineitems}

\index{filter\_by\_type() (galaxy.galaxy.Galaxy method)@\spxentry{filter\_by\_type()}\spxextra{galaxy.galaxy.Galaxy method}}

\begin{fulllineitems}
\phantomsection\label{\detokenize{galaxy:galaxy.galaxy.Galaxy.filter_by_type}}\pysiglinewithargsret{\sphinxbfcode{\sphinxupquote{filter\_by\_type}}}{\emph{particle\_type}, \emph{dataset=None}}{}
Subsets the data to a single particle type.
\begin{description}
\item[{Args:}] \leavevmode
particle\_type (int): for particles, 1=DM, 2=disk, 3=bulge
dataset (array including a type column): defaults to self.data

\item[{Kwargs:}] \leavevmode
dataset (np.ndarray): optionally, a starting dataset other than self.data

\end{description}

Returns: np.ndarray: subset data

\end{fulllineitems}

\index{single\_particle\_properties() (galaxy.galaxy.Galaxy method)@\spxentry{single\_particle\_properties()}\spxextra{galaxy.galaxy.Galaxy method}}

\begin{fulllineitems}
\phantomsection\label{\detokenize{galaxy:galaxy.galaxy.Galaxy.single_particle_properties}}\pysiglinewithargsret{\sphinxbfcode{\sphinxupquote{single\_particle\_properties}}}{\emph{particle\_type=None}, \emph{particle\_num=0}}{}
Calculates distance from the origin and magnitude of the velocity.
\begin{description}
\item[{Kwargs:}] \leavevmode\begin{description}
\item[{particle\_type (int):}] \leavevmode
a subset of the data filtered by 1=DM, 2=disk, 3=bulge

\item[{particle\_num (int):}] \leavevmode
zero\sphinxhyphen{}based index to an array of particles

\end{description}

\item[{returns:}] \leavevmode\begin{description}
\item[{3\sphinxhyphen{}tuple of}] \leavevmode
Euclidean distance from origin (kpc),
Euclidean velocity magnitude (km/s),
particle mass (M\_sun)

\end{description}

\end{description}

\end{fulllineitems}

\index{all\_particle\_properties() (galaxy.galaxy.Galaxy method)@\spxentry{all\_particle\_properties()}\spxextra{galaxy.galaxy.Galaxy method}}

\begin{fulllineitems}
\phantomsection\label{\detokenize{galaxy:galaxy.galaxy.Galaxy.all_particle_properties}}\pysiglinewithargsret{\sphinxbfcode{\sphinxupquote{all\_particle\_properties}}}{\emph{particle\_type=None}, \emph{as\_table=True}}{}
Calculates distances from the origin and magnitude of the velocities
for all particles (default) or a specied particle type.
\begin{description}
\item[{Kwargs: }] \leavevmode\begin{description}
\item[{particle\_type (int):}] \leavevmode
A subset of the data filtered by 1=DM, 2=disk, 3=bulge

\item[{as\_table (boolean): Return type. }] \leavevmode
If True, an astropy QTable with units. 
If False, np.ndarrays for position and velocity

\end{description}

\item[{Returns:}] \leavevmode
QTable: 
The full list, optionally with units, optionally filtered by type.

\end{description}

\end{fulllineitems}

\index{component\_count() (galaxy.galaxy.Galaxy method)@\spxentry{component\_count()}\spxextra{galaxy.galaxy.Galaxy method}}

\begin{fulllineitems}
\phantomsection\label{\detokenize{galaxy:galaxy.galaxy.Galaxy.component_count}}\pysiglinewithargsret{\sphinxbfcode{\sphinxupquote{component\_count}}}{\emph{particle\_type=None}}{}~\begin{description}
\item[{Kwargs: particle\_type (int):}] \leavevmode
a subset of the data filtered by 1=DM, 2=disk, 3=bulge

\item[{Returns: Quantity:}] \leavevmode
The number of particles in the galaxy of this type

\end{description}

\end{fulllineitems}

\index{all\_component\_counts() (galaxy.galaxy.Galaxy method)@\spxentry{all\_component\_counts()}\spxextra{galaxy.galaxy.Galaxy method}}

\begin{fulllineitems}
\phantomsection\label{\detokenize{galaxy:galaxy.galaxy.Galaxy.all_component_counts}}\pysiglinewithargsret{\sphinxbfcode{\sphinxupquote{all\_component\_counts}}}{}{}~\begin{description}
\item[{Returns: list:}] \leavevmode
The number of particles of each type in the galaxy
Ordered as {[}halo, disk, bulge{]}

\end{description}

\end{fulllineitems}

\index{component\_mass() (galaxy.galaxy.Galaxy method)@\spxentry{component\_mass()}\spxextra{galaxy.galaxy.Galaxy method}}

\begin{fulllineitems}
\phantomsection\label{\detokenize{galaxy:galaxy.galaxy.Galaxy.component_mass}}\pysiglinewithargsret{\sphinxbfcode{\sphinxupquote{component\_mass}}}{\emph{particle\_type=None}}{}~\begin{description}
\item[{Kwargs: particle\_type (int):}] \leavevmode
a subset of the data filtered by 1=DM, 2=disk, 3=bulge

\item[{Returns: Quantity:}] \leavevmode
The aggregate mass of all particles in the galaxy of this type

\end{description}

\end{fulllineitems}

\index{all\_component\_masses() (galaxy.galaxy.Galaxy method)@\spxentry{all\_component\_masses()}\spxextra{galaxy.galaxy.Galaxy method}}

\begin{fulllineitems}
\phantomsection\label{\detokenize{galaxy:galaxy.galaxy.Galaxy.all_component_masses}}\pysiglinewithargsret{\sphinxbfcode{\sphinxupquote{all\_component\_masses}}}{}{}~\begin{description}
\item[{Returns: list:}] \leavevmode
The aggregate masses of particles of each type in the galaxy

\end{description}

\end{fulllineitems}

\index{get\_array() (galaxy.galaxy.Galaxy method)@\spxentry{get\_array()}\spxextra{galaxy.galaxy.Galaxy method}}

\begin{fulllineitems}
\phantomsection\label{\detokenize{galaxy:galaxy.galaxy.Galaxy.get_array}}\pysiglinewithargsret{\sphinxbfcode{\sphinxupquote{get\_array}}}{}{}
Returns: all particle data in \sphinxtitleref{np.ndarray} format

Pretty superfluous in Python (which has no private class members)

\end{fulllineitems}

\index{get\_df() (galaxy.galaxy.Galaxy method)@\spxentry{get\_df()}\spxextra{galaxy.galaxy.Galaxy method}}

\begin{fulllineitems}
\phantomsection\label{\detokenize{galaxy:galaxy.galaxy.Galaxy.get_df}}\pysiglinewithargsret{\sphinxbfcode{\sphinxupquote{get\_df}}}{}{}
Returns: data as pandas dataframe

\end{fulllineitems}

\index{get\_qtable() (galaxy.galaxy.Galaxy method)@\spxentry{get\_qtable()}\spxextra{galaxy.galaxy.Galaxy method}}

\begin{fulllineitems}
\phantomsection\label{\detokenize{galaxy:galaxy.galaxy.Galaxy.get_qtable}}\pysiglinewithargsret{\sphinxbfcode{\sphinxupquote{get\_qtable}}}{}{}
Returns: data as astropy QTable, with units

\end{fulllineitems}

\index{xyz() (galaxy.galaxy.Galaxy method)@\spxentry{xyz()}\spxextra{galaxy.galaxy.Galaxy method}}

\begin{fulllineitems}
\phantomsection\label{\detokenize{galaxy:galaxy.galaxy.Galaxy.xyz}}\pysiglinewithargsret{\sphinxbfcode{\sphinxupquote{xyz}}}{}{}
Returns position as a (3,N) array

\end{fulllineitems}

\index{vxyz() (galaxy.galaxy.Galaxy method)@\spxentry{vxyz()}\spxextra{galaxy.galaxy.Galaxy method}}

\begin{fulllineitems}
\phantomsection\label{\detokenize{galaxy:galaxy.galaxy.Galaxy.vxyz}}\pysiglinewithargsret{\sphinxbfcode{\sphinxupquote{vxyz}}}{}{}
Returns velocity as a (3,N) array

\end{fulllineitems}

\index{m() (galaxy.galaxy.Galaxy method)@\spxentry{m()}\spxextra{galaxy.galaxy.Galaxy method}}

\begin{fulllineitems}
\phantomsection\label{\detokenize{galaxy:galaxy.galaxy.Galaxy.m}}\pysiglinewithargsret{\sphinxbfcode{\sphinxupquote{m}}}{}{}
Conventience method to return array of masses

\end{fulllineitems}


\end{fulllineitems}

\phantomsection\label{\detokenize{galaxies:module-galaxy.galaxies}}\index{galaxy.galaxies (module)@\spxentry{galaxy.galaxies}\spxextra{module}}

\chapter{Galaxies class}
\label{\detokenize{galaxies:galaxies-class}}\label{\detokenize{galaxies::doc}}
This stores and manipulates data for multiple galaxies and snaps.
\index{Galaxies (class in galaxy.galaxies)@\spxentry{Galaxies}\spxextra{class in galaxy.galaxies}}

\begin{fulllineitems}
\phantomsection\label{\detokenize{galaxies:galaxy.galaxies.Galaxies}}\pysiglinewithargsret{\sphinxbfcode{\sphinxupquote{class }}\sphinxcode{\sphinxupquote{galaxy.galaxies.}}\sphinxbfcode{\sphinxupquote{Galaxies}}}{\emph{names=(\textquotesingle{}MW\textquotesingle{}}, \emph{\textquotesingle{}M31\textquotesingle{}}, \emph{\textquotesingle{}M33\textquotesingle{})}, \emph{snaps=(0}, \emph{0}, \emph{0)}, \emph{datadir=None}, \emph{usesql=False}, \emph{ptype=None}, \emph{stride=1}}{}
A class to manipulate data for multiple galaxies.
\begin{description}
\item[{Kwargs:}] \leavevmode\begin{description}
\item[{names (iterable of str):}] \leavevmode
short names used in filename of type ‘name\_000.txt’, eg ‘MW’, ‘M31’.

\item[{snaps (iterable of int):}] \leavevmode
Snap number, equivalent to time elapsed. Zero is starting conditions.

\item[{datadir (str):}] \leavevmode
Directory to search first for the required file. Optional, and a
default list of locations will be searched.

\item[{usesql (bool):}] \leavevmode
If True, data will be taken from a PostgreSQL database instead of
text files.

\item[{ptype (int):}] \leavevmode
Optional. Restrict data to this particle type, for speed. 
Only valid with usesql=True.

\item[{stride (int):}] \leavevmode
Optional. For stride=n, get every nth row in the table.
Only valid with usesql=True.

\end{description}

\item[{Class attributes:}] \leavevmode\begin{description}
\item[{path (\sphinxtitleref{pathlib.Path} object):}] \leavevmode
directory (probably) containing the data files

\item[{filenames (list of str):}] \leavevmode
in \sphinxtitleref{name\_snap} format, something like ‘MW\_000’ (no extension)

\item[{galaxies (dict):}] \leavevmode
key is filename, value is the corresponding Galaxy object

\end{description}

\end{description}
\index{read\_data\_files() (galaxy.galaxies.Galaxies method)@\spxentry{read\_data\_files()}\spxextra{galaxy.galaxies.Galaxies method}}

\begin{fulllineitems}
\phantomsection\label{\detokenize{galaxies:galaxy.galaxies.Galaxies.read_data_files}}\pysiglinewithargsret{\sphinxbfcode{\sphinxupquote{read\_data\_files}}}{}{}
Attempts to create a Galaxy object for each name/snap combination
set in \sphinxtitleref{self.names} and \sphinxtitleref{self.snaps}

No return value.
Sets \sphinxtitleref{self.galaxies}, a dictionary keyed on \sphinxtitleref{name\_snap}

\end{fulllineitems}

\index{get\_pivot() (galaxy.galaxies.Galaxies method)@\spxentry{get\_pivot()}\spxextra{galaxy.galaxies.Galaxies method}}

\begin{fulllineitems}
\phantomsection\label{\detokenize{galaxies:galaxy.galaxies.Galaxies.get_pivot}}\pysiglinewithargsret{\sphinxbfcode{\sphinxupquote{get\_pivot}}}{\emph{aggfunc}, \emph{values=\textquotesingle{}m\textquotesingle{}}}{}
Generic method to make a pandas pivot table from the 9 combinations of 
galaxy and particle type.
\begin{description}
\item[{Args:}] \leavevmode
aggfunc (str): ‘count’, ‘sum’, etc as aggregation method
values (str): column name to aggregate

\end{description}

Returns: pandas dataframe

\end{fulllineitems}

\index{get\_counts\_pivot() (galaxy.galaxies.Galaxies method)@\spxentry{get\_counts\_pivot()}\spxextra{galaxy.galaxies.Galaxies method}}

\begin{fulllineitems}
\phantomsection\label{\detokenize{galaxies:galaxy.galaxies.Galaxies.get_counts_pivot}}\pysiglinewithargsret{\sphinxbfcode{\sphinxupquote{get\_counts\_pivot}}}{}{}
Pivots on \sphinxtitleref{count(‘m)}.

Returns: pandas dataframe

\end{fulllineitems}

\index{get\_masses\_pivot() (galaxy.galaxies.Galaxies method)@\spxentry{get\_masses\_pivot()}\spxextra{galaxy.galaxies.Galaxies method}}

\begin{fulllineitems}
\phantomsection\label{\detokenize{galaxies:galaxy.galaxies.Galaxies.get_masses_pivot}}\pysiglinewithargsret{\sphinxbfcode{\sphinxupquote{get\_masses\_pivot}}}{}{}
Pivots on \sphinxtitleref{sum(‘m)}.

Returns: pandas dataframe

\end{fulllineitems}

\index{get\_full\_df() (galaxy.galaxies.Galaxies method)@\spxentry{get\_full\_df()}\spxextra{galaxy.galaxies.Galaxies method}}

\begin{fulllineitems}
\phantomsection\label{\detokenize{galaxies:galaxy.galaxies.Galaxies.get_full_df}}\pysiglinewithargsret{\sphinxbfcode{\sphinxupquote{get\_full\_df}}}{}{}
Combined data for all input files.
\begin{description}
\item[{Returns:}] \leavevmode
Concatenated pandas dataframe from all galaxies
Includes ‘name’ and ‘snap’ columns

\end{description}

\end{fulllineitems}

\index{get\_coms() (galaxy.galaxies.Galaxies method)@\spxentry{get\_coms()}\spxextra{galaxy.galaxies.Galaxies method}}

\begin{fulllineitems}
\phantomsection\label{\detokenize{galaxies:galaxy.galaxies.Galaxies.get_coms}}\pysiglinewithargsret{\sphinxbfcode{\sphinxupquote{get\_coms}}}{\emph{tolerance=0.1}, \emph{ptypes=(1}, \emph{2}, \emph{3)}}{}
Center of Mass determination for all galaxies. 
Defaults to all particle types, but \sphinxtitleref{ptypes=(2,)} may be more useful.
\begin{description}
\item[{Args:}] \leavevmode
tolerance (float): convergence criterion (kpc)

\item[{Returns:}] \leavevmode
QTable with COM positions and velocities
colnames: {[}‘name’, ‘ptype’, ‘x’, ‘y’, ‘z’, ‘vx’, ‘vy’, ‘vz’, ‘R’, ‘V’{]}

\end{description}

\end{fulllineitems}

\index{separations() (galaxy.galaxies.Galaxies method)@\spxentry{separations()}\spxextra{galaxy.galaxies.Galaxies method}}

\begin{fulllineitems}
\phantomsection\label{\detokenize{galaxies:galaxy.galaxies.Galaxies.separations}}\pysiglinewithargsret{\sphinxbfcode{\sphinxupquote{separations}}}{\emph{g1}, \emph{g2}}{}
Position and velocity of galaxy g2 COM relative to g1 COM. 
Uses only disk particles for the COM determination.
\begin{description}
\item[{Args:}] \leavevmode
g1, g2 (str): galaxies matching entries in self.filenames

\item[{Returns:}] \leavevmode
Dictionary containing relative position, distance, velocities in
Cartesian and radial coordinates

\end{description}

\end{fulllineitems}

\index{total\_com() (galaxy.galaxies.Galaxies method)@\spxentry{total\_com()}\spxextra{galaxy.galaxies.Galaxies method}}

\begin{fulllineitems}
\phantomsection\label{\detokenize{galaxies:galaxy.galaxies.Galaxies.total_com}}\pysiglinewithargsret{\sphinxbfcode{\sphinxupquote{total\_com}}}{}{}
Center of Mass determination for the local group.

Uses all particles of all types. Position and velocity should be conserved 
quantities, subject to numerical imprecision in the sim.
\begin{description}
\item[{Returns:}] \leavevmode
position, velocity: 3\sphinxhyphen{}vectors

\end{description}

\end{fulllineitems}

\index{total\_angmom() (galaxy.galaxies.Galaxies method)@\spxentry{total\_angmom()}\spxextra{galaxy.galaxies.Galaxies method}}

\begin{fulllineitems}
\phantomsection\label{\detokenize{galaxies:galaxy.galaxies.Galaxies.total_angmom}}\pysiglinewithargsret{\sphinxbfcode{\sphinxupquote{total\_angmom}}}{\emph{origin}}{}
Calculate angular momentum summed over all particles in the local group,
abot point \sphinxtitleref{origin}.
\begin{description}
\item[{Arg:}] \leavevmode
origin (3\sphinxhyphen{}vector): x,y,z coordinates

\item[{Returns:}] \leavevmode
angular momentum: 3\sphinxhyphen{}vector

\end{description}

\end{fulllineitems}


\end{fulllineitems}

\phantomsection\label{\detokenize{centerofmass:module-galaxy.centerofmass}}\index{galaxy.centerofmass (module)@\spxentry{galaxy.centerofmass}\spxextra{module}}

\chapter{CenterOfMass class}
\label{\detokenize{centerofmass:centerofmass-class}}\label{\detokenize{centerofmass::doc}}
Determines position and velocity of the COM for a galaxy/particle
type combination.
\index{CenterOfMass (class in galaxy.centerofmass)@\spxentry{CenterOfMass}\spxextra{class in galaxy.centerofmass}}

\begin{fulllineitems}
\phantomsection\label{\detokenize{centerofmass:galaxy.centerofmass.CenterOfMass}}\pysiglinewithargsret{\sphinxbfcode{\sphinxupquote{class }}\sphinxcode{\sphinxupquote{galaxy.centerofmass.}}\sphinxbfcode{\sphinxupquote{CenterOfMass}}}{\emph{gal}, \emph{ptype=2}}{}
Class to define COM position and velocity properties of a given galaxy 
and simulation snapshot
\begin{description}
\item[{Args:}] \leavevmode\begin{description}
\item[{gal (Galaxy object):}] \leavevmode
The desired galaxy/snap to operate on

\item[{ptype (int):}] \leavevmode
for particles, 1=DM/halo, 2=disk, 3=bulge

\end{description}

\item[{Throws:}] \leavevmode
ValueError, if there are no particles of this type in this galaxy
(typically, halo particles in M33)

\end{description}
\index{com\_define() (galaxy.centerofmass.CenterOfMass method)@\spxentry{com\_define()}\spxextra{galaxy.centerofmass.CenterOfMass method}}

\begin{fulllineitems}
\phantomsection\label{\detokenize{centerofmass:galaxy.centerofmass.CenterOfMass.com_define}}\pysiglinewithargsret{\sphinxbfcode{\sphinxupquote{com\_define}}}{\emph{xyz}, \emph{m}}{}
Function to compute the center of mass position or velocity generically
\begin{description}
\item[{Args: }] \leavevmode\begin{description}
\item[{xyz (array with shape (3, N)):}] \leavevmode
(x, y, z) positions or velocities

\item[{m (1\sphinxhyphen{}D array):}] \leavevmode
particle masses

\end{description}

\item[{Returns: }] \leavevmode
3\sphinxhyphen{}element array, the center of mass coordinates

\end{description}

\end{fulllineitems}

\index{com\_p() (galaxy.centerofmass.CenterOfMass method)@\spxentry{com\_p()}\spxextra{galaxy.centerofmass.CenterOfMass method}}

\begin{fulllineitems}
\phantomsection\label{\detokenize{centerofmass:galaxy.centerofmass.CenterOfMass.com_p}}\pysiglinewithargsret{\sphinxbfcode{\sphinxupquote{com\_p}}}{\emph{delta=0.1}, \emph{vol\_dec=2.0}}{}
Function to specifically return the center of mass position and velocity    .
\begin{description}
\item[{Kwargs:                                                                                                           }] \leavevmode
delta (tolerance)

\item[{Returns: }] \leavevmode
One 3\sphinxhyphen{}vector, coordinates of the center of mass position (kpc)

\end{description}

\end{fulllineitems}

\index{com\_v() (galaxy.centerofmass.CenterOfMass method)@\spxentry{com\_v()}\spxextra{galaxy.centerofmass.CenterOfMass method}}

\begin{fulllineitems}
\phantomsection\label{\detokenize{centerofmass:galaxy.centerofmass.CenterOfMass.com_v}}\pysiglinewithargsret{\sphinxbfcode{\sphinxupquote{com\_v}}}{\emph{xyz\_com}}{}
Center of Mass velocity

Args: X, Y, Z positions of the COM (no units)

Returns: 3\sphinxhyphen{}Vector of COM velocities

\end{fulllineitems}

\index{center\_com() (galaxy.centerofmass.CenterOfMass method)@\spxentry{center\_com()}\spxextra{galaxy.centerofmass.CenterOfMass method}}

\begin{fulllineitems}
\phantomsection\label{\detokenize{centerofmass:galaxy.centerofmass.CenterOfMass.center_com}}\pysiglinewithargsret{\sphinxbfcode{\sphinxupquote{center\_com}}}{\emph{com\_p=None}, \emph{com\_v=None}}{}
Positions and velocities of disk particles relative to the CoM
\begin{description}
\item[{Returns}] \leavevmode{[}two (3, N) arrays{]}
CoM\sphinxhyphen{}centric position and velocity

\end{description}

\end{fulllineitems}

\index{angular\_momentum() (galaxy.centerofmass.CenterOfMass method)@\spxentry{angular\_momentum()}\spxextra{galaxy.centerofmass.CenterOfMass method}}

\begin{fulllineitems}
\phantomsection\label{\detokenize{centerofmass:galaxy.centerofmass.CenterOfMass.angular_momentum}}\pysiglinewithargsret{\sphinxbfcode{\sphinxupquote{angular\_momentum}}}{\emph{com\_p=None}, \emph{com\_v=None}, \emph{r\_lim=None}}{}~\begin{description}
\item[{Returns: }] \leavevmode\begin{description}
\item[{L}] \leavevmode{[}3\sphinxhyphen{}vector as array{]}
The (x,y,x) components of the angular momentum vector about the CoM,
summed over all disk particles

\item[{pos, v}] \leavevmode{[}arrays with shape (3, N){]}
Position and velocity for each particle

\item[{r\_lim}] \leavevmode{[}float{]}
Radius to include in calculation (implicit kpc, no units)

\end{description}

\end{description}

\end{fulllineitems}

\index{rotate\_frame() (galaxy.centerofmass.CenterOfMass method)@\spxentry{rotate\_frame()}\spxextra{galaxy.centerofmass.CenterOfMass method}}

\begin{fulllineitems}
\phantomsection\label{\detokenize{centerofmass:galaxy.centerofmass.CenterOfMass.rotate_frame}}\pysiglinewithargsret{\sphinxbfcode{\sphinxupquote{rotate\_frame}}}{\emph{to\_axis=None}, \emph{com\_p=None}, \emph{com\_v=None}, \emph{r\_lim=None}}{}~\begin{description}
\item[{Arg: to\_axis (3\sphinxhyphen{}vector)}] \leavevmode
Angular momentum vector will be aligned to this (default z\sphinxhyphen{}hat)

\item[{Returns: (positions, velocities), two arrays of shape (3, N)}] \leavevmode
New values for every particle. \sphinxtitleref{self.data} remains unchanged.

\end{description}

Based on Rodrigues’ rotation formula
Ref: \sphinxurl{https://en.wikipedia.org/wiki/Rodrigues\%27\_rotation\_formula}

\end{fulllineitems}

\index{shell\_h() (galaxy.centerofmass.CenterOfMass method)@\spxentry{shell\_h()}\spxextra{galaxy.centerofmass.CenterOfMass method}}

\begin{fulllineitems}
\phantomsection\label{\detokenize{centerofmass:galaxy.centerofmass.CenterOfMass.shell_h}}\pysiglinewithargsret{\sphinxbfcode{\sphinxupquote{shell\_h}}}{\emph{radii}, \emph{m}, \emph{xyz}, \emph{vxyz}}{}
Calculate specific angular momentum of spherical shells.
\begin{description}
\item[{Args:}] \leavevmode\begin{description}
\item[{radii ((M,0) array of float):}] \leavevmode
boundaries of shells (implicit kpc from center)

\item[{m ((N,) array of float):}] \leavevmode
masses (implicit Msun)

\item[{xyz, vxyz ((3,N) arrays of float):}] \leavevmode
positions and velocities (implicit kpc, km/s)

\end{description}

\item[{Returns:}] \leavevmode\begin{description}
\item[{rad ((M\sphinxhyphen{}1,) array): }] \leavevmode
midpoints of shells (implicit kpc)

\item[{L ((M\sphinxhyphen{}1,3) array of float):}] \leavevmode
total angular momentum

\item[{h ((M\sphinxhyphen{}1,3) array of float):}] \leavevmode
specific angular momentum, i.e. L/m

\end{description}

\end{description}

\end{fulllineitems}

\index{sphere\_h() (galaxy.centerofmass.CenterOfMass method)@\spxentry{sphere\_h()}\spxextra{galaxy.centerofmass.CenterOfMass method}}

\begin{fulllineitems}
\phantomsection\label{\detokenize{centerofmass:galaxy.centerofmass.CenterOfMass.sphere_h}}\pysiglinewithargsret{\sphinxbfcode{\sphinxupquote{sphere\_h}}}{\emph{radii}, \emph{m}, \emph{xyz}, \emph{vxyz}}{}
Calculate specific angular momentum within spheres.
\begin{description}
\item[{Args:}] \leavevmode\begin{description}
\item[{radii ((M,0) array of float):}] \leavevmode
boundaries of shells (implicit kpc from center)

\item[{m ((N,) array of float):}] \leavevmode
masses (implicit Msun)

\item[{xyz, vxyz ((3,N) arrays of float):}] \leavevmode
positions and velocities (implicit kpc, km/s)

\end{description}

\item[{Returns:}] \leavevmode\begin{description}
\item[{L ((M,3) array of float):}] \leavevmode
total angular momentum

\item[{h ((M,3) array of float):}] \leavevmode
specific angular momentum, i.e. L/m

\end{description}

\end{description}

\end{fulllineitems}

\index{disp\_by\_radius() (galaxy.centerofmass.CenterOfMass method)@\spxentry{disp\_by\_radius()}\spxextra{galaxy.centerofmass.CenterOfMass method}}

\begin{fulllineitems}
\phantomsection\label{\detokenize{centerofmass:galaxy.centerofmass.CenterOfMass.disp_by_radius}}\pysiglinewithargsret{\sphinxbfcode{\sphinxupquote{disp\_by\_radius}}}{\emph{x}, \emph{vy}, \emph{xbins}, \emph{binwidth=None}}{}
Calculate mean velocity and dispersion sigma for a set of radius bins
\begin{description}
\item[{Args:}] \leavevmode\begin{description}
\item[{x (array of float):}] \leavevmode
lateral distance from CoM

\item[{vy (array of float):}] \leavevmode
radial velocity

\item[{xbins (array of float)}] \leavevmode
midpoints of equally\sphinxhyphen{}spaced bins

\item[{binwidth (float):}] \leavevmode
optional and probably useless

\end{description}

\end{description}

\end{fulllineitems}


\end{fulllineitems}

\phantomsection\label{\detokenize{massprofile:module-galaxy.massprofile}}\index{galaxy.massprofile (module)@\spxentry{galaxy.massprofile}\spxextra{module}}

\chapter{MassProfile class}
\label{\detokenize{massprofile:massprofile-class}}\label{\detokenize{massprofile::doc}}
Calculates mass vs. radius relations and rotation curves for a given galaxy.
\index{MassProfile (class in galaxy.massprofile)@\spxentry{MassProfile}\spxextra{class in galaxy.massprofile}}

\begin{fulllineitems}
\phantomsection\label{\detokenize{massprofile:galaxy.massprofile.MassProfile}}\pysiglinewithargsret{\sphinxbfcode{\sphinxupquote{class }}\sphinxcode{\sphinxupquote{galaxy.massprofile.}}\sphinxbfcode{\sphinxupquote{MassProfile}}}{\emph{gal}, \emph{com\_p=None}}{}
Class to define mass enclosed as a function of radius and circular velocity
profiles for a given galaxy and simulation snapshot
\begin{description}
\item[{Args:}] \leavevmode\begin{description}
\item[{gal (Galaxy object):}] \leavevmode
The desired galaxy/snap to operate on

\item[{com\_p (3\sphinxhyphen{}vector):}] \leavevmode
Optional. The position of the disk CoM.

\end{description}

\end{description}
\index{mass\_enclosed() (galaxy.massprofile.MassProfile method)@\spxentry{mass\_enclosed()}\spxextra{galaxy.massprofile.MassProfile method}}

\begin{fulllineitems}
\phantomsection\label{\detokenize{massprofile:galaxy.massprofile.MassProfile.mass_enclosed}}\pysiglinewithargsret{\sphinxbfcode{\sphinxupquote{mass\_enclosed}}}{\emph{radii}, \emph{ptype=None}}{}
Calculate the mass within a given radius of the CoM 
for a given type of particle.
\begin{description}
\item[{Args:}] \leavevmode
radii (array of distances): spheres to integrate over
ptype (int): particle type from (1,2,3), or None for total

\item[{Returns:}] \leavevmode
array of masses, in units of M\_sun

\end{description}

\end{fulllineitems}

\index{mass\_enclosed\_total() (galaxy.massprofile.MassProfile method)@\spxentry{mass\_enclosed\_total()}\spxextra{galaxy.massprofile.MassProfile method}}

\begin{fulllineitems}
\phantomsection\label{\detokenize{massprofile:galaxy.massprofile.MassProfile.mass_enclosed_total}}\pysiglinewithargsret{\sphinxbfcode{\sphinxupquote{mass\_enclosed\_total}}}{\emph{radii}}{}
Calculate the mass within a given radius of the CoM, 
summed for all types of particle.
\begin{description}
\item[{Args:}] \leavevmode
radii (array of distances): spheres to integrate over

\item[{Returns:}] \leavevmode
array of masses, in units of M\_sun

\end{description}

\end{fulllineitems}

\index{halo\_mass() (galaxy.massprofile.MassProfile method)@\spxentry{halo\_mass()}\spxextra{galaxy.massprofile.MassProfile method}}

\begin{fulllineitems}
\phantomsection\label{\detokenize{massprofile:galaxy.massprofile.MassProfile.halo_mass}}\pysiglinewithargsret{\sphinxbfcode{\sphinxupquote{halo\_mass}}}{}{}
Utility function to get a parameter for Hernquist mass

\end{fulllineitems}

\index{hernquist\_mass() (galaxy.massprofile.MassProfile method)@\spxentry{hernquist\_mass()}\spxextra{galaxy.massprofile.MassProfile method}}

\begin{fulllineitems}
\phantomsection\label{\detokenize{massprofile:galaxy.massprofile.MassProfile.hernquist_mass}}\pysiglinewithargsret{\sphinxbfcode{\sphinxupquote{hernquist\_mass}}}{\emph{r}, \emph{a}, \emph{M\_halo=None}}{}
Calculate the mass enclosed for a theoretical profile
\begin{description}
\item[{Args:}] \leavevmode\begin{description}
\item[{r (Quantity, units of kpc): }] \leavevmode
distance from center

\item[{a (Quantity, units of kpc): }] \leavevmode
scale radius

\item[{M\_halo (Quantity, units of M\_sun): }] \leavevmode
total DM mass (optional)

\end{description}

\item[{Returns:}] \leavevmode
Total DM mass enclosed within r (M\_sun)

\end{description}

\end{fulllineitems}

\index{circular\_velocity() (galaxy.massprofile.MassProfile method)@\spxentry{circular\_velocity()}\spxextra{galaxy.massprofile.MassProfile method}}

\begin{fulllineitems}
\phantomsection\label{\detokenize{massprofile:galaxy.massprofile.MassProfile.circular_velocity}}\pysiglinewithargsret{\sphinxbfcode{\sphinxupquote{circular\_velocity}}}{\emph{radii}, \emph{ptype=None}}{}
Calculate orbital velocity at a given radius from the CoM 
for a given type of particle.
\begin{description}
\item[{Args:}] \leavevmode\begin{description}
\item[{radii (array of distances): }] \leavevmode
circular orbit

\item[{ptype (int): }] \leavevmode
particle type from (1,2,3), or None for total

\end{description}

\item[{Returns:}] \leavevmode
array of circular speeds, in units of km/s

\end{description}

\end{fulllineitems}

\index{circular\_velocity\_total() (galaxy.massprofile.MassProfile method)@\spxentry{circular\_velocity\_total()}\spxextra{galaxy.massprofile.MassProfile method}}

\begin{fulllineitems}
\phantomsection\label{\detokenize{massprofile:galaxy.massprofile.MassProfile.circular_velocity_total}}\pysiglinewithargsret{\sphinxbfcode{\sphinxupquote{circular\_velocity\_total}}}{\emph{radii}}{}
Syntactic sugar for circular\_velocity(radii, ptype=None)

\end{fulllineitems}

\index{circular\_velocity\_hernquist() (galaxy.massprofile.MassProfile method)@\spxentry{circular\_velocity\_hernquist()}\spxextra{galaxy.massprofile.MassProfile method}}

\begin{fulllineitems}
\phantomsection\label{\detokenize{massprofile:galaxy.massprofile.MassProfile.circular_velocity_hernquist}}\pysiglinewithargsret{\sphinxbfcode{\sphinxupquote{circular\_velocity\_hernquist}}}{\emph{radii}, \emph{a}, \emph{M\_halo=None}}{}
Theoretical V\_circ assuming halo mass follows a Hernquist profile
\begin{description}
\item[{Args:}] \leavevmode\begin{description}
\item[{radii (array of distances): }] \leavevmode
circular orbit

\item[{a (Quantity, units of kpc): }] \leavevmode
scale radius

\item[{M\_halo (Quantity, units of M\_sun): }] \leavevmode
total DM mass (optional)

\end{description}

\end{description}

\end{fulllineitems}

\index{fit\_hernquist\_a() (galaxy.massprofile.MassProfile method)@\spxentry{fit\_hernquist\_a()}\spxextra{galaxy.massprofile.MassProfile method}}

\begin{fulllineitems}
\phantomsection\label{\detokenize{massprofile:galaxy.massprofile.MassProfile.fit_hernquist_a}}\pysiglinewithargsret{\sphinxbfcode{\sphinxupquote{fit\_hernquist\_a}}}{\emph{r\_inner=1}, \emph{r\_outer=30}, \emph{get\_details=False}}{}
Get \sphinxtitleref{scipy.optimize} to do a non\sphinxhyphen{}linear least squares fit to find
the best scale radius \sphinxtitleref{a} for the Hernquist profile.
\begin{description}
\item[{Args:}] \leavevmode\begin{description}
\item[{r\_inner (numeric):}] \leavevmode
Optional. Minimum radius to consider (implicit kpc). 
Avoid values \textless{} 1 as they cause numeric problems.

\item[{r\_outer (numeric):}] \leavevmode
Optional. Maximum radius to consider (implicit kpc).

\end{description}

\end{description}

\end{fulllineitems}

\index{sersic() (galaxy.massprofile.MassProfile method)@\spxentry{sersic()}\spxextra{galaxy.massprofile.MassProfile method}}

\begin{fulllineitems}
\phantomsection\label{\detokenize{massprofile:galaxy.massprofile.MassProfile.sersic}}\pysiglinewithargsret{\sphinxbfcode{\sphinxupquote{sersic}}}{\emph{R}, \emph{Re}, \emph{n}, \emph{Mtot}}{}
Function that returns Sersic Profile for an Elliptical System
(See in\sphinxhyphen{}class lab 6)
\begin{description}
\item[{Input}] \leavevmode\begin{description}
\item[{R:}] \leavevmode
radius (kpc)

\item[{Re:}] \leavevmode
half mass radius (kpc)

\item[{n:}] \leavevmode
sersic index

\item[{Mtot:}] \leavevmode
total stellar mass

\end{description}

\item[{Returns}] \leavevmode
Surface Brightness profile in Lsun/kpc\textasciicircum{}2

\end{description}

\end{fulllineitems}

\index{bulge\_Re() (galaxy.massprofile.MassProfile method)@\spxentry{bulge\_Re()}\spxextra{galaxy.massprofile.MassProfile method}}

\begin{fulllineitems}
\phantomsection\label{\detokenize{massprofile:galaxy.massprofile.MassProfile.bulge_Re}}\pysiglinewithargsret{\sphinxbfcode{\sphinxupquote{bulge\_Re}}}{\emph{R}}{}
Find the radius enclosing half the bulge mass.
\begin{description}
\item[{Args:}] \leavevmode\begin{description}
\item[{R (array of Quantity):}] \leavevmode
Radii to consider (kpc)

\end{description}

\item[{Returns:}] \leavevmode\begin{description}
\item[{Re (Quantity) :}] \leavevmode
Radius enclosing half light/mass (kpc)

\item[{bulge\_total (numeric):}] \leavevmode
Mass of entire bulge (M\_sun, no units)

\item[{bulgeI (array of Quantity):}] \leavevmode
Surface brightness at radii R (kpc\textasciicircum{}\sphinxhyphen{}2), assuming M/L=1

\end{description}

\end{description}

\end{fulllineitems}

\index{fit\_sersic\_n() (galaxy.massprofile.MassProfile method)@\spxentry{fit\_sersic\_n()}\spxextra{galaxy.massprofile.MassProfile method}}

\begin{fulllineitems}
\phantomsection\label{\detokenize{massprofile:galaxy.massprofile.MassProfile.fit_sersic_n}}\pysiglinewithargsret{\sphinxbfcode{\sphinxupquote{fit\_sersic\_n}}}{\emph{R}, \emph{Re}, \emph{bulge\_total}, \emph{bulgeI}}{}
Get \sphinxtitleref{scipy.optimize} to do a non\sphinxhyphen{}linear least squares fit to find
the best value of \sphinxtitleref{n} for a Sersic profile.
\begin{description}
\item[{Args:}] \leavevmode\begin{description}
\item[{R (array of quantity):}] \leavevmode
Radii at which to calculate fit (kpc)

\item[{Re (Quantity) :}] \leavevmode
Radius enclosing half light/mass (kpc)

\item[{bulge\_total (numeric):}] \leavevmode
Mass of entire bulge (M\_sun, no units)

\item[{bulgeI (array of Quantity):}] \leavevmode
Surface brightness at radii R (kpc\textasciicircum{}\sphinxhyphen{}2)

\end{description}

\item[{Returns:}] \leavevmode
best \sphinxtitleref{n} value and error estimate

\end{description}

\end{fulllineitems}

\index{density\_profile\_shell() (galaxy.massprofile.MassProfile method)@\spxentry{density\_profile\_shell()}\spxextra{galaxy.massprofile.MassProfile method}}

\begin{fulllineitems}
\phantomsection\label{\detokenize{massprofile:galaxy.massprofile.MassProfile.density_profile_shell}}\pysiglinewithargsret{\sphinxbfcode{\sphinxupquote{density\_profile\_shell}}}{\emph{radii}, \emph{m}, \emph{xyz}}{}
Calculates mass density in successive spherical shells
\begin{description}
\item[{Arg:}] \leavevmode\begin{description}
\item[{radii (array of float):}] \leavevmode
boundary values beteen shells (implicit kpc, no units)

\item[{m (shape (N,) array of float):}] \leavevmode
particle masses (implicit Msun, no units)

\item[{xyz ((3,N) array of float):}] \leavevmode
particle cartesian coordinates

\end{description}

\item[{Returns:}] \leavevmode\begin{description}
\item[{r\_annuli: geometric mean of boundaries }] \leavevmode
(array is one shorter than input radii)

\item[{rho: densities (Msun/kpc\textasciicircum{}3)}] \leavevmode
(same length as r\_annuli)

\end{description}

\end{description}

\end{fulllineitems}

\index{density\_profile\_sphere() (galaxy.massprofile.MassProfile method)@\spxentry{density\_profile\_sphere()}\spxextra{galaxy.massprofile.MassProfile method}}

\begin{fulllineitems}
\phantomsection\label{\detokenize{massprofile:galaxy.massprofile.MassProfile.density_profile_sphere}}\pysiglinewithargsret{\sphinxbfcode{\sphinxupquote{density\_profile\_sphere}}}{\emph{radii}, \emph{m}, \emph{xyz}}{}
Calculates average mass density within successive spherical radii
\begin{description}
\item[{Arg:}] \leavevmode\begin{description}
\item[{radii (array of float):}] \leavevmode
boundary values beteen shells (implicit kpc, no units)

\item[{m (shape (N,) array of float):}] \leavevmode
particle masses (implicit Msun, no units)

\item[{xyz ((3,N) array of float):}] \leavevmode
particle cartesian coordinates

\end{description}

\item[{Returns:}] \leavevmode\begin{description}
\item[{rho: densities (Msun/kpc\textasciicircum{}3)}] \leavevmode
(same length as radii)

\end{description}

\end{description}

\end{fulllineitems}

\index{virial\_radius() (galaxy.massprofile.MassProfile method)@\spxentry{virial\_radius()}\spxextra{galaxy.massprofile.MassProfile method}}

\begin{fulllineitems}
\phantomsection\label{\detokenize{massprofile:galaxy.massprofile.MassProfile.virial_radius}}\pysiglinewithargsret{\sphinxbfcode{\sphinxupquote{virial\_radius}}}{\emph{r\_min=20}, \emph{r\_max=1000}, \emph{rho\_c=None}}{}
Calculates radius where DM density falls to 200x critical density
for the universe.
\begin{description}
\item[{Args:}] \leavevmode\begin{description}
\item[{r\_min, r\_max (floats)}] \leavevmode
optional, limits for search (implicit kpc, no units)

\item[{rho\_c (float or Quantity)}] \leavevmode
optional, critical density for chosen cosmology

\end{description}

\item[{Returns:}] \leavevmode
r\_200 (float): virial radius, implicit kpc

\end{description}

\end{fulllineitems}

\index{virial\_mass() (galaxy.massprofile.MassProfile method)@\spxentry{virial\_mass()}\spxextra{galaxy.massprofile.MassProfile method}}

\begin{fulllineitems}
\phantomsection\label{\detokenize{massprofile:galaxy.massprofile.MassProfile.virial_mass}}\pysiglinewithargsret{\sphinxbfcode{\sphinxupquote{virial\_mass}}}{\emph{r\_200=None}, \emph{ptype=None}}{}
Mass enclosed by the virial radius

\end{fulllineitems}


\end{fulllineitems}

\phantomsection\label{\detokenize{timecourse:module-galaxy.timecourse}}\index{galaxy.timecourse (module)@\spxentry{galaxy.timecourse}\spxextra{module}}

\chapter{TimeCourse class}
\label{\detokenize{timecourse:timecourse-class}}\label{\detokenize{timecourse::doc}}
Various methods to work with data across a series of snaps (timepoints).
\index{TimeCourse (class in galaxy.timecourse)@\spxentry{TimeCourse}\spxextra{class in galaxy.timecourse}}

\begin{fulllineitems}
\phantomsection\label{\detokenize{timecourse:galaxy.timecourse.TimeCourse}}\pysiglinewithargsret{\sphinxbfcode{\sphinxupquote{class }}\sphinxcode{\sphinxupquote{galaxy.timecourse.}}\sphinxbfcode{\sphinxupquote{TimeCourse}}}{\emph{datadir=\textquotesingle{}.\textquotesingle{}}, \emph{usesql=False}}{}
A commection of methods for generating, reading and writing summary data for 
parameters that change over the timecourse of the simulation.
\begin{description}
\item[{These fall into a few groups:}] \leavevmode\begin{description}
\item[{\sphinxtitleref{write\_xxx()}}] \leavevmode{[}{]}
Methods that loop through the raw data, calculate parameters and write 
the results to file. Can be slow (hours) to run but Only run once. 
See the \sphinxtitleref{data} folder for the resulting files, one line per snap.

\item[{\sphinxtitleref{read\_xxx\_file(})\textasciigrave{} :}] \leavevmode
Read in the summary files and return a numpy array. These rely on the
generic \sphinxtitleref{read\_file()} method.

\item[{\sphinxtitleref{read\_xxx\_db()} :}] \leavevmode
Get the summary data from postgres instead of a text file. 
The returned array should be identical to the \sphinxtitleref{read\_xxx\_file()} group.

\item[{\sphinxtitleref{write\_db\_tables()} :}] \leavevmode
Read a text file, insert the contents to a postgres table.

\item[{\sphinxtitleref{get\_one\_com()} :}] \leavevmode
Convenience method to return a single CoM position.

\end{description}

\end{description}
\index{write\_com\_ang\_mom() (galaxy.timecourse.TimeCourse method)@\spxentry{write\_com\_ang\_mom()}\spxextra{galaxy.timecourse.TimeCourse method}}

\begin{fulllineitems}
\phantomsection\label{\detokenize{timecourse:galaxy.timecourse.TimeCourse.write_com_ang_mom}}\pysiglinewithargsret{\sphinxbfcode{\sphinxupquote{write\_com\_ang\_mom}}}{\emph{galname}, \emph{start=0}, \emph{end=801}, \emph{n=5}, \emph{show\_progress=True}}{}
Function that loops over all the desired snapshots to compute the COM pos and vel as a 
function of time.
\begin{description}
\item[{inputs:}] \leavevmode\begin{description}
\item[{galname (str):}] \leavevmode
‘MW’, ‘M31’ or ‘M33’

\item[{start, end (int):}] \leavevmode
first and last snap numbers to include

\item[{n (int):}] \leavevmode
stride length for the sequence

\item[{datadir (str):}] \leavevmode
path to the input data

\item[{show\_progress (bool):}] \leavevmode
prints each snap number as it is processed

\end{description}

\item[{returns: }] \leavevmode
Two text files saved to disk.

\end{description}

\end{fulllineitems}

\index{write\_total\_com() (galaxy.timecourse.TimeCourse method)@\spxentry{write\_total\_com()}\spxextra{galaxy.timecourse.TimeCourse method}}

\begin{fulllineitems}
\phantomsection\label{\detokenize{timecourse:galaxy.timecourse.TimeCourse.write_total_com}}\pysiglinewithargsret{\sphinxbfcode{\sphinxupquote{write\_total\_com}}}{\emph{start=0}, \emph{end=801}, \emph{n=1}, \emph{show\_progress=True}}{}
Function that loops over all the desired snapshots to compute the overall COM 
pos and vel as a function of time. Uses all particles in all galaxies.
\begin{description}
\item[{inputs:}] \leavevmode\begin{description}
\item[{start, end (int):}] \leavevmode
first and last snap numbers to include

\item[{n (int):}] \leavevmode
stride length for the sequence

\item[{show\_progress (bool):}] \leavevmode
prints each snap number as it is processed

\end{description}

\item[{output: }] \leavevmode
Text file saved to disk.

\end{description}

\end{fulllineitems}

\index{write\_total\_angmom() (galaxy.timecourse.TimeCourse method)@\spxentry{write\_total\_angmom()}\spxextra{galaxy.timecourse.TimeCourse method}}

\begin{fulllineitems}
\phantomsection\label{\detokenize{timecourse:galaxy.timecourse.TimeCourse.write_total_angmom}}\pysiglinewithargsret{\sphinxbfcode{\sphinxupquote{write\_total\_angmom}}}{\emph{start=0}, \emph{end=801}, \emph{n=1}, \emph{show\_progress=True}}{}
Function that loops over all the desired snapshots to compute the overall 
angular momentum as a function of time. Uses all particles in all galaxies.
\begin{description}
\item[{inputs:}] \leavevmode\begin{description}
\item[{start, end (int):}] \leavevmode
first and last snap numbers to include

\item[{n (int):}] \leavevmode
stride length for the sequence

\item[{show\_progress (bool):}] \leavevmode
prints each snap number as it is processed

\end{description}

\item[{output: }] \leavevmode
Text file saved to disk.

\end{description}

\end{fulllineitems}

\index{write\_vel\_disp() (galaxy.timecourse.TimeCourse method)@\spxentry{write\_vel\_disp()}\spxextra{galaxy.timecourse.TimeCourse method}}

\begin{fulllineitems}
\phantomsection\label{\detokenize{timecourse:galaxy.timecourse.TimeCourse.write_vel_disp}}\pysiglinewithargsret{\sphinxbfcode{\sphinxupquote{write\_vel\_disp}}}{\emph{galname}, \emph{start=0}, \emph{end=801}, \emph{n=1}, \emph{show\_progress=True}}{}
Function that loops over all the desired snapshots to compute the veocity dispersion
sigma as a function of time.
\begin{description}
\item[{inputs:}] \leavevmode\begin{description}
\item[{galname (str):}] \leavevmode
‘MW’, ‘M31’ or ‘M33’

\item[{start, end (int):}] \leavevmode
first and last snap numbers to include

\item[{n (int):}] \leavevmode
stride length for the sequence

\item[{datadir (str):}] \leavevmode
path to the input data

\item[{show\_progress (bool):}] \leavevmode
prints each snap number as it is processed

\end{description}

\item[{returns: }] \leavevmode
Text file saved to disk.

\end{description}

\end{fulllineitems}

\index{write\_LG\_normal() (galaxy.timecourse.TimeCourse method)@\spxentry{write\_LG\_normal()}\spxextra{galaxy.timecourse.TimeCourse method}}

\begin{fulllineitems}
\phantomsection\label{\detokenize{timecourse:galaxy.timecourse.TimeCourse.write_LG_normal}}\pysiglinewithargsret{\sphinxbfcode{\sphinxupquote{write\_LG\_normal}}}{\emph{start=0}, \emph{end=801}}{}
Calculates the normal to a plane containing the three galaxy CoMs.
\begin{description}
\item[{Args:}] \leavevmode\begin{description}
\item[{start, end (int):}] \leavevmode
first and last snap numbers to include

\end{description}

\item[{output: }] \leavevmode
Text file saved to disk.

\end{description}

\end{fulllineitems}

\index{write\_rel\_motion() (galaxy.timecourse.TimeCourse method)@\spxentry{write\_rel\_motion()}\spxextra{galaxy.timecourse.TimeCourse method}}

\begin{fulllineitems}
\phantomsection\label{\detokenize{timecourse:galaxy.timecourse.TimeCourse.write_rel_motion}}\pysiglinewithargsret{\sphinxbfcode{\sphinxupquote{write\_rel\_motion}}}{\emph{start=0}, \emph{end=801}}{}
\end{fulllineitems}

\index{read\_file() (galaxy.timecourse.TimeCourse method)@\spxentry{read\_file()}\spxextra{galaxy.timecourse.TimeCourse method}}

\begin{fulllineitems}
\phantomsection\label{\detokenize{timecourse:galaxy.timecourse.TimeCourse.read_file}}\pysiglinewithargsret{\sphinxbfcode{\sphinxupquote{read\_file}}}{\emph{fullname}}{}
General method for file input. Note that the format is for summary files,
(one line per snap), not the raw per\sphinxhyphen{}particle files.

\end{fulllineitems}

\index{read\_com\_file() (galaxy.timecourse.TimeCourse method)@\spxentry{read\_com\_file()}\spxextra{galaxy.timecourse.TimeCourse method}}

\begin{fulllineitems}
\phantomsection\label{\detokenize{timecourse:galaxy.timecourse.TimeCourse.read_com_file}}\pysiglinewithargsret{\sphinxbfcode{\sphinxupquote{read\_com\_file}}}{\emph{galaxy}, \emph{datadir=\textquotesingle{}.\textquotesingle{}}}{}
Get CoM summary from file.
\begin{description}
\item[{Args:}] \leavevmode\begin{description}
\item[{galaxy (str): }] \leavevmode
‘MW’, ‘M31’, ‘M33’

\item[{datadir (str):}] \leavevmode
path to file

\end{description}

\item[{Returns:}] \leavevmode
np.array with 802 rows, one per snap

\end{description}

\end{fulllineitems}

\index{read\_angmom\_file() (galaxy.timecourse.TimeCourse method)@\spxentry{read\_angmom\_file()}\spxextra{galaxy.timecourse.TimeCourse method}}

\begin{fulllineitems}
\phantomsection\label{\detokenize{timecourse:galaxy.timecourse.TimeCourse.read_angmom_file}}\pysiglinewithargsret{\sphinxbfcode{\sphinxupquote{read\_angmom\_file}}}{\emph{galaxy}, \emph{datadir=\textquotesingle{}.\textquotesingle{}}}{}
Get CoM summary from file.
\begin{description}
\item[{Args:}] \leavevmode\begin{description}
\item[{galaxy (str): }] \leavevmode
‘MW’, ‘M31’, ‘M33’

\item[{datadir (str):}] \leavevmode
path to file

\end{description}

\item[{Returns:}] \leavevmode
np.array with 802 rows, one per snap

\end{description}

\end{fulllineitems}

\index{read\_total\_com\_file() (galaxy.timecourse.TimeCourse method)@\spxentry{read\_total\_com\_file()}\spxextra{galaxy.timecourse.TimeCourse method}}

\begin{fulllineitems}
\phantomsection\label{\detokenize{timecourse:galaxy.timecourse.TimeCourse.read_total_com_file}}\pysiglinewithargsret{\sphinxbfcode{\sphinxupquote{read\_total\_com\_file}}}{\emph{datadir=\textquotesingle{}.\textquotesingle{}}}{}
Get CoM summary from file.
\begin{description}
\item[{Args:}] \leavevmode\begin{description}
\item[{datadir (str):}] \leavevmode
path to file

\end{description}

\item[{Returns:}] \leavevmode
np.array with 802 rows, one per snap

\end{description}

\end{fulllineitems}

\index{read\_normals\_file() (galaxy.timecourse.TimeCourse method)@\spxentry{read\_normals\_file()}\spxextra{galaxy.timecourse.TimeCourse method}}

\begin{fulllineitems}
\phantomsection\label{\detokenize{timecourse:galaxy.timecourse.TimeCourse.read_normals_file}}\pysiglinewithargsret{\sphinxbfcode{\sphinxupquote{read\_normals\_file}}}{\emph{datadir=\textquotesingle{}.\textquotesingle{}}}{}
Get normals to plane containing 3 galaxy CoMs from file.
\begin{description}
\item[{Args:}] \leavevmode\begin{description}
\item[{datadir (str):}] \leavevmode
path to file

\end{description}

\item[{Returns:}] \leavevmode
np.array with 802 rows, one per snap

\end{description}

\end{fulllineitems}

\index{read\_relmotion\_file() (galaxy.timecourse.TimeCourse method)@\spxentry{read\_relmotion\_file()}\spxextra{galaxy.timecourse.TimeCourse method}}

\begin{fulllineitems}
\phantomsection\label{\detokenize{timecourse:galaxy.timecourse.TimeCourse.read_relmotion_file}}\pysiglinewithargsret{\sphinxbfcode{\sphinxupquote{read\_relmotion\_file}}}{\emph{datadir=\textquotesingle{}.\textquotesingle{}}}{}
Get relative CoM distances/velocities from file.
\begin{description}
\item[{Args:}] \leavevmode\begin{description}
\item[{datadir (str):}] \leavevmode
path to file

\end{description}

\item[{Returns:}] \leavevmode
np.array with 802 rows, one per snap

\end{description}

\end{fulllineitems}

\index{write\_db\_tables() (galaxy.timecourse.TimeCourse method)@\spxentry{write\_db\_tables()}\spxextra{galaxy.timecourse.TimeCourse method}}

\begin{fulllineitems}
\phantomsection\label{\detokenize{timecourse:galaxy.timecourse.TimeCourse.write_db_tables}}\pysiglinewithargsret{\sphinxbfcode{\sphinxupquote{write\_db\_tables}}}{\emph{datadir=\textquotesingle{}.\textquotesingle{}}, \emph{do\_com=False}, \emph{do\_angmom=False}, \emph{do\_totalcom=False}, \emph{do\_totalangmom=False}, \emph{do\_normals=False}, \emph{do\_sigmas=False}, \emph{do\_relmotion=False}}{}
Adds data to the various tables in the \sphinxtitleref{galaxy} database

\end{fulllineitems}

\index{read\_com\_db() (galaxy.timecourse.TimeCourse method)@\spxentry{read\_com\_db()}\spxextra{galaxy.timecourse.TimeCourse method}}

\begin{fulllineitems}
\phantomsection\label{\detokenize{timecourse:galaxy.timecourse.TimeCourse.read_com_db}}\pysiglinewithargsret{\sphinxbfcode{\sphinxupquote{read\_com\_db}}}{\emph{galaxy=None}, \emph{snaprange=(0}, \emph{801)}}{}
Retrieves CoM positions from postgres for a range of snaps.
\begin{description}
\item[{Args:}] \leavevmode\begin{description}
\item[{galaxy (str):}] \leavevmode
Optional, defaults to all. Can be ‘MW’, ‘M31’ , ‘M33’

\item[{snaprange (pair of ints):}] \leavevmode
Optional, defaults to all. First and last snap to include.
This is NOT the {[}first, last+1{]} convention of Python.

\end{description}

\end{description}

\end{fulllineitems}

\index{read\_angmom\_db() (galaxy.timecourse.TimeCourse method)@\spxentry{read\_angmom\_db()}\spxextra{galaxy.timecourse.TimeCourse method}}

\begin{fulllineitems}
\phantomsection\label{\detokenize{timecourse:galaxy.timecourse.TimeCourse.read_angmom_db}}\pysiglinewithargsret{\sphinxbfcode{\sphinxupquote{read\_angmom\_db}}}{\emph{galaxy=None}, \emph{snaprange=(0}, \emph{801)}}{}
Retrieves disk angular momentum from postgres for a range of snaps.
\begin{description}
\item[{Args:}] \leavevmode\begin{description}
\item[{galaxy (str):}] \leavevmode
Optional, defaults to all. Can be ‘MW, ‘M31 , ‘M33’

\item[{snaprange (pair of ints):}] \leavevmode
Optional, defaults to all. First and last snap to include.
This is NOT the {[}first, last+1{]} convention of Python.

\end{description}

\end{description}

\end{fulllineitems}

\index{read\_total\_com\_db() (galaxy.timecourse.TimeCourse method)@\spxentry{read\_total\_com\_db()}\spxextra{galaxy.timecourse.TimeCourse method}}

\begin{fulllineitems}
\phantomsection\label{\detokenize{timecourse:galaxy.timecourse.TimeCourse.read_total_com_db}}\pysiglinewithargsret{\sphinxbfcode{\sphinxupquote{read\_total\_com\_db}}}{\emph{snaprange=(0}, \emph{801)}}{}
Retrieves total CoM positions from postgres for a range of snaps.
\begin{description}
\item[{Args:}] \leavevmode\begin{description}
\item[{snaprange (pair of ints):}] \leavevmode
Optional, defaults to all. First and last snap to include.
This is NOT the {[}first, last+1{]} convention of Python.

\end{description}

\end{description}

\end{fulllineitems}

\index{get\_one\_com() (galaxy.timecourse.TimeCourse method)@\spxentry{get\_one\_com()}\spxextra{galaxy.timecourse.TimeCourse method}}

\begin{fulllineitems}
\phantomsection\label{\detokenize{timecourse:galaxy.timecourse.TimeCourse.get_one_com}}\pysiglinewithargsret{\sphinxbfcode{\sphinxupquote{get\_one\_com}}}{\emph{gal}, \emph{snap}}{}
Gets a CoM from postgres for the specified galaxy and snap.
\begin{description}
\item[{Args:}] \leavevmode\begin{description}
\item[{gal (str): }] \leavevmode
Can be ‘MW, ‘M31 , ‘M33’

\item[{snap (int):}] \leavevmode
The timepoint.

\end{description}

\end{description}

\end{fulllineitems}

\index{read\_total\_angmom\_db() (galaxy.timecourse.TimeCourse method)@\spxentry{read\_total\_angmom\_db()}\spxextra{galaxy.timecourse.TimeCourse method}}

\begin{fulllineitems}
\phantomsection\label{\detokenize{timecourse:galaxy.timecourse.TimeCourse.read_total_angmom_db}}\pysiglinewithargsret{\sphinxbfcode{\sphinxupquote{read\_total\_angmom\_db}}}{\emph{snaprange=(0}, \emph{801)}}{}
Gets the total angular momentum of the 3\sphinxhyphen{}galaxy system. In practice, 
this turns out to be near\sphinxhyphen{}zero at all timepoints and can be ignored
in future work.

\end{fulllineitems}

\index{read\_normals\_db() (galaxy.timecourse.TimeCourse method)@\spxentry{read\_normals\_db()}\spxextra{galaxy.timecourse.TimeCourse method}}

\begin{fulllineitems}
\phantomsection\label{\detokenize{timecourse:galaxy.timecourse.TimeCourse.read_normals_db}}\pysiglinewithargsret{\sphinxbfcode{\sphinxupquote{read\_normals\_db}}}{\emph{snaprange=(0}, \emph{801)}}{}
Gets the normals to the 3\sphinxhyphen{}galaxy plane.

\end{fulllineitems}

\index{read\_sigmas\_db() (galaxy.timecourse.TimeCourse method)@\spxentry{read\_sigmas\_db()}\spxextra{galaxy.timecourse.TimeCourse method}}

\begin{fulllineitems}
\phantomsection\label{\detokenize{timecourse:galaxy.timecourse.TimeCourse.read_sigmas_db}}\pysiglinewithargsret{\sphinxbfcode{\sphinxupquote{read\_sigmas\_db}}}{\emph{galaxy=None}, \emph{snaprange=(0}, \emph{801)}}{}
Gets the velocity dispersions (km/s) for one galaxy at a range of snaps.

\end{fulllineitems}

\index{read\_relmotion\_db() (galaxy.timecourse.TimeCourse method)@\spxentry{read\_relmotion\_db()}\spxextra{galaxy.timecourse.TimeCourse method}}

\begin{fulllineitems}
\phantomsection\label{\detokenize{timecourse:galaxy.timecourse.TimeCourse.read_relmotion_db}}\pysiglinewithargsret{\sphinxbfcode{\sphinxupquote{read\_relmotion\_db}}}{\emph{snaprange=(0}, \emph{801)}}{}
Retrieves relative CoM positions and velocities from postgres 
for a range of snaps.
\begin{description}
\item[{Args:}] \leavevmode\begin{description}
\item[{snaprange (pair of ints):}] \leavevmode
Optional, defaults to all. First and last snap to include.
This is NOT the {[}first, last+1{]} convention of Python.

\end{description}

\end{description}

\end{fulllineitems}

\index{snap2time() (galaxy.timecourse.TimeCourse method)@\spxentry{snap2time()}\spxextra{galaxy.timecourse.TimeCourse method}}

\begin{fulllineitems}
\phantomsection\label{\detokenize{timecourse:galaxy.timecourse.TimeCourse.snap2time}}\pysiglinewithargsret{\sphinxbfcode{\sphinxupquote{snap2time}}}{\emph{snap}}{}
\end{fulllineitems}

\index{time2snap() (galaxy.timecourse.TimeCourse method)@\spxentry{time2snap()}\spxextra{galaxy.timecourse.TimeCourse method}}

\begin{fulllineitems}
\phantomsection\label{\detokenize{timecourse:galaxy.timecourse.TimeCourse.time2snap}}\pysiglinewithargsret{\sphinxbfcode{\sphinxupquote{time2snap}}}{\emph{time}}{}~\begin{description}
\item[{Arg:}] \leavevmode
time (float): value in Gyr

\item[{Returns:}] \leavevmode
List of closest values, often but not reliably just one.

\end{description}

\end{fulllineitems}


\end{fulllineitems}



\chapter{utilities module}
\label{\detokenize{utilities:utilities-module}}\label{\detokenize{utilities::doc}}
A collection of useful functions that don’t fit into any of the other classes.

\phantomsection\label{\detokenize{utilities:module-galaxy.utilities}}\index{galaxy.utilities (module)@\spxentry{galaxy.utilities}\spxextra{module}}\index{wolf\_mass() (in module galaxy.utilities)@\spxentry{wolf\_mass()}\spxextra{in module galaxy.utilities}}

\begin{fulllineitems}
\phantomsection\label{\detokenize{utilities:galaxy.utilities.wolf_mass}}\pysiglinewithargsret{\sphinxcode{\sphinxupquote{galaxy.utilities.}}\sphinxbfcode{\sphinxupquote{wolf\_mass}}}{\emph{sigma}, \emph{Re}}{}
Wolf mass estimator from Wolf+ 2010
\begin{description}
\item[{Args:}] \leavevmode\begin{description}
\item[{sigma}] \leavevmode{[}{]}
1D line of sight velocity dispersion in km/s

\item[{Re}] \leavevmode{[}{]}
2D radius enclosing half the stellar mass in pc

\end{description}

\end{description}

Returns: estimate of the dynamical mass within the half light radius in Msun

\end{fulllineitems}

\index{sersic() (in module galaxy.utilities)@\spxentry{sersic()}\spxextra{in module galaxy.utilities}}

\begin{fulllineitems}
\phantomsection\label{\detokenize{utilities:galaxy.utilities.sersic}}\pysiglinewithargsret{\sphinxcode{\sphinxupquote{galaxy.utilities.}}\sphinxbfcode{\sphinxupquote{sersic}}}{\emph{R}, \emph{Re}, \emph{n}, \emph{Mtot}}{}
Function that returns Sersic Profile for an Elliptical System
(See in\sphinxhyphen{}class lab 6)
\begin{description}
\item[{Input}] \leavevmode\begin{description}
\item[{R:}] \leavevmode
radius (kpc)

\item[{Re:}] \leavevmode
half mass radius (kpc)

\item[{n:}] \leavevmode
sersic index

\item[{Mtot:}] \leavevmode
total stellar mass

\end{description}

\item[{Returns}] \leavevmode
Surface Brightness profile in Lsun/kpc\textasciicircum{}2

\end{description}

\end{fulllineitems}

\index{HernquistM() (in module galaxy.utilities)@\spxentry{HernquistM()}\spxextra{in module galaxy.utilities}}

\begin{fulllineitems}
\phantomsection\label{\detokenize{utilities:galaxy.utilities.HernquistM}}\pysiglinewithargsret{\sphinxcode{\sphinxupquote{galaxy.utilities.}}\sphinxbfcode{\sphinxupquote{HernquistM}}}{\emph{r}, \emph{a=\textless{}Quantity 60. kpc\textgreater{}}, \emph{M\_halo=\textless{}Quantity 1.97e+12 solMass\textgreater{}}}{}~\begin{description}
\item[{Args:}] \leavevmode
r (Quantity, units of kpc): distance from center
a (Quantity, units of kpc): scale radius
M\_halo (Quantity, units of M\_sun): total DM mass

\item[{Returns:}] \leavevmode
Total DM mass enclosed within r (M\_sun)

\end{description}

\end{fulllineitems}

\index{jacobi\_radius() (in module galaxy.utilities)@\spxentry{jacobi\_radius()}\spxextra{in module galaxy.utilities}}

\begin{fulllineitems}
\phantomsection\label{\detokenize{utilities:galaxy.utilities.jacobi_radius}}\pysiglinewithargsret{\sphinxcode{\sphinxupquote{galaxy.utilities.}}\sphinxbfcode{\sphinxupquote{jacobi\_radius}}}{\emph{r}, \emph{M\_host}, \emph{M\_sat}}{}
The Jacobi Radius for a satellite on a circular orbit about an extended host, 
where the host is assumed to be well modeled as an isothermal sphere halo:

R\_j = r * (M\_sat / 2 M\_host(\textless{}r))\}\textasciicircum{}(1/3)

For MW/LMC, the Isothermal Sphere approximation is not a bad one within 50 kpc.

In other contexts, can be called the Roche radius, Roche limit or Hill radius.
\begin{description}
\item[{Args:}] \leavevmode\begin{description}
\item[{r: }] \leavevmode
distance between stellite and host (kpc)

\item[{M\_host: }] \leavevmode
host mass enclosed within r (M\_sun)

\item[{M\_sat: }] \leavevmode
satellite mass (M\_sun)

\end{description}

\item[{returns: }] \leavevmode
Jacobi radius (kpc)

\end{description}

\end{fulllineitems}

\index{jacobi\_mass() (in module galaxy.utilities)@\spxentry{jacobi\_mass()}\spxextra{in module galaxy.utilities}}

\begin{fulllineitems}
\phantomsection\label{\detokenize{utilities:galaxy.utilities.jacobi_mass}}\pysiglinewithargsret{\sphinxcode{\sphinxupquote{galaxy.utilities.}}\sphinxbfcode{\sphinxupquote{jacobi\_mass}}}{\emph{Rj}, \emph{r}, \emph{Mhost}}{}
Function that returns min mass of a satellite given its observed size + distance 
from a massive host: Msat = (Rj/r)**3 * 2 * Mhost
\begin{description}
\item[{Args:}] \leavevmode\begin{description}
\item[{Rj: }] \leavevmode
Jacobi radius (approx as observed size) (kpc)

\item[{r: }] \leavevmode
distance between stellite and host (kpc)

\item[{Mhost: }] \leavevmode
mass enclosed within r (M\_sun)

\end{description}

\item[{returns: }] \leavevmode
Minimum mass Msat of a satellite given its current size (M\_sun)

\end{description}

\end{fulllineitems}

\index{rotation\_matrix\_to\_vector() (in module galaxy.utilities)@\spxentry{rotation\_matrix\_to\_vector()}\spxextra{in module galaxy.utilities}}

\begin{fulllineitems}
\phantomsection\label{\detokenize{utilities:galaxy.utilities.rotation_matrix_to_vector}}\pysiglinewithargsret{\sphinxcode{\sphinxupquote{galaxy.utilities.}}\sphinxbfcode{\sphinxupquote{rotation\_matrix\_to\_vector}}}{\emph{old\_axis}, \emph{to\_axis=None}}{}~\begin{description}
\item[{Args: }] \leavevmode\begin{description}
\item[{old\_axis (3\sphinxhyphen{}vector)}] \leavevmode
Vector to be brought into alignment with \sphinxtitleref{to\_axis} by rotation about the origin

\item[{to\_axis (3\sphinxhyphen{}vector)}] \leavevmode
Angular momentum vector will be aligned to this (default z\_hat)

\end{description}

\item[{Returns: }] \leavevmode
3x3 rotation matrix

\end{description}

Based on Rodrigues’ rotation formula
Ref: \sphinxurl{https://en.wikipedia.org/wiki/Rodrigues\%27\_rotation\_formula}

Note that orientation in the plane perpendicular to ‘to\_axis’ is arbitrary

\end{fulllineitems}

\index{z\_rotation\_matrix() (in module galaxy.utilities)@\spxentry{z\_rotation\_matrix()}\spxextra{in module galaxy.utilities}}

\begin{fulllineitems}
\phantomsection\label{\detokenize{utilities:galaxy.utilities.z_rotation_matrix}}\pysiglinewithargsret{\sphinxcode{\sphinxupquote{galaxy.utilities.}}\sphinxbfcode{\sphinxupquote{z\_rotation\_matrix}}}{\emph{pt1}, \emph{pt2}}{}
Rotates about z\sphinxhyphen{}axis to line up two given points along the x\sphinxhyphen{}axis
\begin{description}
\item[{Args:}] \leavevmode\begin{description}
\item[{pt1, pt2 (2\sphinxhyphen{}component iterables)}] \leavevmode
define points to be placed on the  x\sphinxhyphen{}axis

\end{description}

\item[{Returns:}] \leavevmode
3x3 rotation matrix

\end{description}

\end{fulllineitems}

\index{find\_nearest() (in module galaxy.utilities)@\spxentry{find\_nearest()}\spxextra{in module galaxy.utilities}}

\begin{fulllineitems}
\phantomsection\label{\detokenize{utilities:galaxy.utilities.find_nearest}}\pysiglinewithargsret{\sphinxcode{\sphinxupquote{galaxy.utilities.}}\sphinxbfcode{\sphinxupquote{find\_nearest}}}{\emph{array}, \emph{value}}{}
Find the entry in \sphinxtitleref{array} which is closest to \sphinxtitleref{value}
Modified from \sphinxurl{https://stackoverflow.com/questions/2566412/find-nearest-value-in-numpy-array}

Returns: index and corresponding value

\end{fulllineitems}

\phantomsection\label{\detokenize{db:module-galaxy.db}}\index{galaxy.db (module)@\spxentry{galaxy.db}\spxextra{module}}

\chapter{DB class}
\label{\detokenize{db:db-class}}\label{\detokenize{db::doc}}
A wrapper for connections to the PostgreSQL database
\index{DB (class in galaxy.db)@\spxentry{DB}\spxextra{class in galaxy.db}}

\begin{fulllineitems}
\phantomsection\label{\detokenize{db:galaxy.db.DB}}\pysigline{\sphinxbfcode{\sphinxupquote{class }}\sphinxcode{\sphinxupquote{galaxy.db.}}\sphinxbfcode{\sphinxupquote{DB}}}
A simple wrapper class for connecting to the PostgreSQL database.

Takes no arguments. Relies on having connection information in
\sphinxtitleref{\textasciitilde{}/dbconn.yaml}.
\index{read\_params() (galaxy.db.DB method)@\spxentry{read\_params()}\spxextra{galaxy.db.DB method}}

\begin{fulllineitems}
\phantomsection\label{\detokenize{db:galaxy.db.DB.read_params}}\pysiglinewithargsret{\sphinxbfcode{\sphinxupquote{read\_params}}}{}{}
Needs the yaml parameter file to be in the user’s home directory

\end{fulllineitems}

\index{get\_cursor() (galaxy.db.DB method)@\spxentry{get\_cursor()}\spxextra{galaxy.db.DB method}}

\begin{fulllineitems}
\phantomsection\label{\detokenize{db:galaxy.db.DB.get_cursor}}\pysiglinewithargsret{\sphinxbfcode{\sphinxupquote{get\_cursor}}}{}{}
A simple getter method

\end{fulllineitems}

\index{run\_query() (galaxy.db.DB method)@\spxentry{run\_query()}\spxextra{galaxy.db.DB method}}

\begin{fulllineitems}
\phantomsection\label{\detokenize{db:galaxy.db.DB.run_query}}\pysiglinewithargsret{\sphinxbfcode{\sphinxupquote{run\_query}}}{\emph{query}}{}
Runs a SQL query (typically SELECT)

Returns results in Python list format 
(not numpy, which would need a dtype list)

\end{fulllineitems}

\index{get\_xyz() (galaxy.db.DB method)@\spxentry{get\_xyz()}\spxextra{galaxy.db.DB method}}

\begin{fulllineitems}
\phantomsection\label{\detokenize{db:galaxy.db.DB.get_xyz}}\pysiglinewithargsret{\sphinxbfcode{\sphinxupquote{get\_xyz}}}{\emph{gal}, \emph{snap}}{}
\end{fulllineitems}


\end{fulllineitems}

\phantomsection\label{\detokenize{remnant:module-galaxy.remnant}}\index{galaxy.remnant (module)@\spxentry{galaxy.remnant}\spxextra{module}}

\chapter{Remnant class}
\label{\detokenize{remnant:remnant-class}}\label{\detokenize{remnant::doc}}
Treats the post\sphinxhyphen{}merger remnant of MW\sphinxhyphen{}M31 as a galaxy.
\index{Remnant (class in galaxy.remnant)@\spxentry{Remnant}\spxextra{class in galaxy.remnant}}

\begin{fulllineitems}
\phantomsection\label{\detokenize{remnant:galaxy.remnant.Remnant}}\pysiglinewithargsret{\sphinxbfcode{\sphinxupquote{class }}\sphinxcode{\sphinxupquote{galaxy.remnant.}}\sphinxbfcode{\sphinxupquote{Remnant}}}{\emph{snap=801}, \emph{datadir=None}, \emph{usesql=False}, \emph{stride=1}, \emph{ptype=(2}, \emph{3)}}{}
A class to work with the MW\sphinxhyphen{}M31 post\sphinxhyphen{}merger remant.
\begin{description}
\item[{Args:}] \leavevmode\begin{description}
\item[{snap (int):}] \leavevmode
Snap number, equivalent to time elapsed. 
Defaults to the last timepoint.

\item[{datadir (str):}] \leavevmode
Directory to search first for the required file. Optional, and a
default list of locations will be searched.

\item[{usesql (bool):}] \leavevmode
If True, data will be taken from a PostgreSQL database instead of
text files.

\item[{stride (int):}] \leavevmode
Optional. For stride=n, get every nth row in the table.
Only valid with usesql=True.

\item[{ptype (int or iterable of int):}] \leavevmode
Particle type: 1, 2, 3 or a combination of these

\end{description}

\item[{Class attributes:}] \leavevmode\begin{description}
\item[{data (np.ndarray):}] \leavevmode
type, mass, position\_xyz, velocity\_xyz for each particle

\end{description}

\end{description}
\index{read\_db() (galaxy.remnant.Remnant method)@\spxentry{read\_db()}\spxextra{galaxy.remnant.Remnant method}}

\begin{fulllineitems}
\phantomsection\label{\detokenize{remnant:galaxy.remnant.Remnant.read_db}}\pysiglinewithargsret{\sphinxbfcode{\sphinxupquote{read\_db}}}{\emph{stride}}{}
Get relevant data from a PostgreSQL database and format it to be 
identical to that read from test files.

Ex\sphinxhyphen{}disk and ex\sphinxhyphen{}bulge particles are included, not DM particles.
\begin{description}
\item[{Args:}] \leavevmode\begin{description}
\item[{stride (int):}] \leavevmode
Optional. For stride=n, get every nth row in the table.

\end{description}

\item[{Changes:}] \leavevmode
\sphinxtitleref{self.time}, \sphinxtitleref{self.particle\_count} and \sphinxtitleref{self.data} are set.

\end{description}

Returns: nothing

\end{fulllineitems}

\index{xyz() (galaxy.remnant.Remnant method)@\spxentry{xyz()}\spxextra{galaxy.remnant.Remnant method}}

\begin{fulllineitems}
\phantomsection\label{\detokenize{remnant:galaxy.remnant.Remnant.xyz}}\pysiglinewithargsret{\sphinxbfcode{\sphinxupquote{xyz}}}{}{}
Convenience method to get positions as a np.array of shape (3,N)

\end{fulllineitems}

\index{vxyz() (galaxy.remnant.Remnant method)@\spxentry{vxyz()}\spxextra{galaxy.remnant.Remnant method}}

\begin{fulllineitems}
\phantomsection\label{\detokenize{remnant:galaxy.remnant.Remnant.vxyz}}\pysiglinewithargsret{\sphinxbfcode{\sphinxupquote{vxyz}}}{}{}
Convenience method to get velocities as a np.array of shape (3,N)

\end{fulllineitems}

\index{I\_tensor() (galaxy.remnant.Remnant method)@\spxentry{I\_tensor()}\spxextra{galaxy.remnant.Remnant method}}

\begin{fulllineitems}
\phantomsection\label{\detokenize{remnant:galaxy.remnant.Remnant.I_tensor}}\pysiglinewithargsret{\sphinxbfcode{\sphinxupquote{I\_tensor}}}{\emph{m}, \emph{x}, \emph{y}, \emph{z}}{}~\begin{description}
\item[{Args:}] \leavevmode\begin{description}
\item[{m, x, y, z:}] \leavevmode
1\sphinxhyphen{}D arrays with mass and coordinates (no units)

\end{description}

\item[{Returns:}] \leavevmode
3x3 array representing the moment of inertia tensor

\end{description}

\end{fulllineitems}

\index{ellipsoid\_axes() (galaxy.remnant.Remnant method)@\spxentry{ellipsoid\_axes()}\spxextra{galaxy.remnant.Remnant method}}

\begin{fulllineitems}
\phantomsection\label{\detokenize{remnant:galaxy.remnant.Remnant.ellipsoid_axes}}\pysiglinewithargsret{\sphinxbfcode{\sphinxupquote{ellipsoid\_axes}}}{\emph{m}, \emph{x}, \emph{y}, \emph{z}, \emph{r\_lim=None}}{}~\begin{description}
\item[{Args:}] \leavevmode\begin{description}
\item[{m, x, y, z:}] \leavevmode
1\sphinxhyphen{}D arrays with mass and coordinates (no units)

\item[{r\_lim}] \leavevmode{[}float{]}
Radius to include in calculation (implicit kpc, no units)

\end{description}

\item[{Returns:}] \leavevmode
Two 3\sphinxhyphen{}tuples: relative semimajor axes and principal axis vectors

\end{description}

\end{fulllineitems}

\index{sub\_mass\_enclosed() (galaxy.remnant.Remnant method)@\spxentry{sub\_mass\_enclosed()}\spxextra{galaxy.remnant.Remnant method}}

\begin{fulllineitems}
\phantomsection\label{\detokenize{remnant:galaxy.remnant.Remnant.sub_mass_enclosed}}\pysiglinewithargsret{\sphinxbfcode{\sphinxupquote{sub\_mass\_enclosed}}}{\emph{radii}, \emph{m}, \emph{xyz}}{}
Calculate the mass within a given radius of the origin.
Based on code in MassProfile, but this version assumes
CoM\sphinxhyphen{}centric coordinates are supplied.
\begin{description}
\item[{Args:}] \leavevmode
radii (array of distances): spheres to integrate over
m (array of masses): shape (N,)
xyz (array of Cartesian coordinates): shape (3,N)

\item[{Returns:}] \leavevmode
array of masses, in units of M\_sun

\end{description}

\end{fulllineitems}

\index{hernquist\_mass() (galaxy.remnant.Remnant method)@\spxentry{hernquist\_mass()}\spxextra{galaxy.remnant.Remnant method}}

\begin{fulllineitems}
\phantomsection\label{\detokenize{remnant:galaxy.remnant.Remnant.hernquist_mass}}\pysiglinewithargsret{\sphinxbfcode{\sphinxupquote{hernquist\_mass}}}{\emph{r}, \emph{a}, \emph{M\_halo}}{}
Calculate the mass enclosed for a theoretical profile
\begin{description}
\item[{Args:}] \leavevmode\begin{description}
\item[{r (Quantity, units of kpc): }] \leavevmode
distance from center

\item[{a (Quantity, units of kpc): }] \leavevmode
scale radius

\item[{M\_halo (Quantity, units of M\_sun): }] \leavevmode
total DM mass

\end{description}

\item[{Returns:}] \leavevmode
Total DM mass enclosed within r (M\_sun)

\end{description}

\end{fulllineitems}

\index{fit\_hernquist\_a() (galaxy.remnant.Remnant method)@\spxentry{fit\_hernquist\_a()}\spxextra{galaxy.remnant.Remnant method}}

\begin{fulllineitems}
\phantomsection\label{\detokenize{remnant:galaxy.remnant.Remnant.fit_hernquist_a}}\pysiglinewithargsret{\sphinxbfcode{\sphinxupquote{fit\_hernquist\_a}}}{\emph{m}, \emph{xyz}, \emph{r\_inner=1}, \emph{r\_outer=100}}{}
Get \sphinxtitleref{scipy.optimize} to do a non\sphinxhyphen{}linear least squares fit to find
the best scale radius \sphinxtitleref{a} for the Hernquist profile.
\begin{description}
\item[{Args:}] \leavevmode\begin{description}
\item[{r\_inner (numeric):}] \leavevmode
Optional. Minimum radius to consider (implicit kpc). 
Avoid values \textless{} 1 as they cause numeric problems.

\item[{r\_outer (numeric):}] \leavevmode
Optional. Maximum radius to consider (implicit kpc).

\end{description}

\end{description}

\end{fulllineitems}

\index{sersic() (galaxy.remnant.Remnant method)@\spxentry{sersic()}\spxextra{galaxy.remnant.Remnant method}}

\begin{fulllineitems}
\phantomsection\label{\detokenize{remnant:galaxy.remnant.Remnant.sersic}}\pysiglinewithargsret{\sphinxbfcode{\sphinxupquote{sersic}}}{\emph{R}, \emph{Re}, \emph{n}, \emph{Mtot}}{}
Function that returns Sersic Profile for an Elliptical System
(See in\sphinxhyphen{}class lab 6)
\begin{description}
\item[{Input}] \leavevmode\begin{description}
\item[{R:}] \leavevmode
radius (kpc)

\item[{Re:}] \leavevmode
half mass radius (kpc)

\item[{n:}] \leavevmode
sersic index

\item[{Mtot:}] \leavevmode
total stellar mass

\end{description}

\item[{Returns}] \leavevmode
Surface Brightness profile in Lsun/kpc\textasciicircum{}2

\end{description}

\end{fulllineitems}

\index{Re() (galaxy.remnant.Remnant method)@\spxentry{Re()}\spxextra{galaxy.remnant.Remnant method}}

\begin{fulllineitems}
\phantomsection\label{\detokenize{remnant:galaxy.remnant.Remnant.Re}}\pysiglinewithargsret{\sphinxbfcode{\sphinxupquote{Re}}}{\emph{R}, \emph{m}, \emph{xyz}}{}
Find the radius enclosing half the mass.
\begin{description}
\item[{Args:}] \leavevmode\begin{description}
\item[{R (array of Quantity):}] \leavevmode
Radii to consider (kpc)

\end{description}

\item[{Returns:}] \leavevmode\begin{description}
\item[{Re (Quantity) :}] \leavevmode
Radius enclosing half light/mass (kpc)

\item[{bulge\_total (numeric):}] \leavevmode
Mass of entire bulge (M\_sun, no units)

\item[{bulgeI (array of Quantity):}] \leavevmode
Surface brightness at radii R (kpc\textasciicircum{}\sphinxhyphen{}2), assuming M/L=1

\end{description}

\end{description}

\end{fulllineitems}

\index{fit\_sersic\_n() (galaxy.remnant.Remnant method)@\spxentry{fit\_sersic\_n()}\spxextra{galaxy.remnant.Remnant method}}

\begin{fulllineitems}
\phantomsection\label{\detokenize{remnant:galaxy.remnant.Remnant.fit_sersic_n}}\pysiglinewithargsret{\sphinxbfcode{\sphinxupquote{fit\_sersic\_n}}}{\emph{R}, \emph{sub\_Re}, \emph{sub\_total}, \emph{subI}}{}
Get \sphinxtitleref{scipy.optimize} to do a non\sphinxhyphen{}linear least squares fit to find
the best value of \sphinxtitleref{n} for a Sersic profile.
\begin{description}
\item[{Args:}] \leavevmode\begin{description}
\item[{R (array of quantity):}] \leavevmode
Radii at which to calculate fit (kpc)

\item[{Re (Quantity) :}] \leavevmode
Radius enclosing half light/mass (kpc)

\item[{bulge\_total (numeric):}] \leavevmode
Mass of entire bulge (M\_sun, no units)

\item[{bulgeI (array of Quantity):}] \leavevmode
Surface brightness at radii R (kpc\textasciicircum{}\sphinxhyphen{}2)

\end{description}

\item[{Returns:}] \leavevmode
best \sphinxtitleref{n} value and error estimate

\end{description}

\end{fulllineitems}


\end{fulllineitems}

\phantomsection\label{\detokenize{approaches:module-galaxy.approaches}}\index{galaxy.approaches (module)@\spxentry{galaxy.approaches}\spxextra{module}}

\chapter{Approaches class}
\label{\detokenize{approaches:approaches-class}}\label{\detokenize{approaches::doc}}
Convenience class to get concatenated data for all galaxies.
Mainly used when they are in close proximity and can appear on the same plot.
\index{Approaches (class in galaxy.approaches)@\spxentry{Approaches}\spxextra{class in galaxy.approaches}}

\begin{fulllineitems}
\phantomsection\label{\detokenize{approaches:galaxy.approaches.Approaches}}\pysiglinewithargsret{\sphinxbfcode{\sphinxupquote{class }}\sphinxcode{\sphinxupquote{galaxy.approaches.}}\sphinxbfcode{\sphinxupquote{Approaches}}}{\emph{snap=801}, \emph{datadir=None}, \emph{usesql=False}, \emph{stride=1}, \emph{ptype=\textquotesingle{}lum\textquotesingle{}}}{}
A class to work with all 3 galaxies when in close proximity.
\begin{description}
\item[{Args:}] \leavevmode\begin{description}
\item[{snap (int):}] \leavevmode
Snap number, equivalent to time elapsed. 
Defaults to the last timepoint.

\item[{datadir (str):}] \leavevmode
Directory to search first for the required file. Optional, and a
default list of locations will be searched.

\item[{usesql (bool):}] \leavevmode
If True, data will be taken from a PostgreSQL database instead of
text files.

\item[{stride (int):}] \leavevmode
Optional. For stride=n, get every nth row in the table.
Only valid with usesql=True.

\item[{ptype (str):}] \leavevmode
can be ‘lum’ for disk+bulge, ‘dm’ for halo

\end{description}

\item[{Class attributes:}] \leavevmode\begin{description}
\item[{data (np.ndarray):}] \leavevmode
type, mass, position\_xyz, velocity\_xyz for each particle

\end{description}

\end{description}
\index{read\_db() (galaxy.approaches.Approaches method)@\spxentry{read\_db()}\spxextra{galaxy.approaches.Approaches method}}

\begin{fulllineitems}
\phantomsection\label{\detokenize{approaches:galaxy.approaches.Approaches.read_db}}\pysiglinewithargsret{\sphinxbfcode{\sphinxupquote{read\_db}}}{\emph{stride}}{}
Get relevant data from a PostgreSQL database and format it to be 
identical to that read from test files.
\begin{description}
\item[{Args:}] \leavevmode\begin{description}
\item[{stride (int):}] \leavevmode
Optional. For stride=n, get every nth row in the table.

\end{description}

\item[{Changes:}] \leavevmode
\sphinxtitleref{self.time}, \sphinxtitleref{self.particle\_count} and \sphinxtitleref{self.data} are set.

\end{description}

Returns: nothing

\end{fulllineitems}

\index{xyz() (galaxy.approaches.Approaches method)@\spxentry{xyz()}\spxextra{galaxy.approaches.Approaches method}}

\begin{fulllineitems}
\phantomsection\label{\detokenize{approaches:galaxy.approaches.Approaches.xyz}}\pysiglinewithargsret{\sphinxbfcode{\sphinxupquote{xyz}}}{}{}
Convenience method to get positions as a np.array of shape (3,N)

\end{fulllineitems}

\index{vxyz() (galaxy.approaches.Approaches method)@\spxentry{vxyz()}\spxextra{galaxy.approaches.Approaches method}}

\begin{fulllineitems}
\phantomsection\label{\detokenize{approaches:galaxy.approaches.Approaches.vxyz}}\pysiglinewithargsret{\sphinxbfcode{\sphinxupquote{vxyz}}}{}{}
Convenience method to get velocities as a np.array of shape (3,N)

\end{fulllineitems}


\end{fulllineitems}

\phantomsection\label{\detokenize{surfacedensity:module-galaxy.surfacedensity}}\index{galaxy.surfacedensity (module)@\spxentry{galaxy.surfacedensity}\spxextra{module}}

\chapter{SurfaceDensityProfile class}
\label{\detokenize{surfacedensity:surfacedensityprofile-class}}\label{\detokenize{surfacedensity::doc}}
Calculate the surface density profile of a galactic disk.
\index{SurfaceDensityProfile (class in galaxy.surfacedensity)@\spxentry{SurfaceDensityProfile}\spxextra{class in galaxy.surfacedensity}}

\begin{fulllineitems}
\phantomsection\label{\detokenize{surfacedensity:galaxy.surfacedensity.SurfaceDensityProfile}}\pysiglinewithargsret{\sphinxbfcode{\sphinxupquote{class }}\sphinxcode{\sphinxupquote{galaxy.surfacedensity.}}\sphinxbfcode{\sphinxupquote{SurfaceDensityProfile}}}{\emph{gal}, \emph{snap}, \emph{radii=None}, \emph{r\_step=0.5}, \emph{ptype=2}, \emph{usesql=False}}{}
Calculate the surface density profile of a galaxy at a snapshot.
Modified from code supplied by Rixin Li.
\index{plot\_xy() (galaxy.surfacedensity.SurfaceDensityProfile method)@\spxentry{plot\_xy()}\spxextra{galaxy.surfacedensity.SurfaceDensityProfile method}}

\begin{fulllineitems}
\phantomsection\label{\detokenize{surfacedensity:galaxy.surfacedensity.SurfaceDensityProfile.plot_xy}}\pysiglinewithargsret{\sphinxbfcode{\sphinxupquote{plot\_xy}}}{\emph{figsize=(8}, \emph{8)}, \emph{xlim=(0}, \emph{60)}, \emph{ylim=(0}, \emph{60)}, \emph{pfc=\textquotesingle{}b\textquotesingle{}}, \emph{ngout=False}, \emph{fname=None}}{}
\end{fulllineitems}


\end{fulllineitems}

\phantomsection\label{\detokenize{remvdisp:module-galaxy.remvdisp}}\index{galaxy.remvdisp (module)@\spxentry{galaxy.remvdisp}\spxextra{module}}

\chapter{Vdisp class}
\label{\detokenize{remvdisp:vdisp-class}}\label{\detokenize{remvdisp::doc}}
To calculate velocity dispersions and rotation curves for the post\sphinxhyphen{}merger remnant of MW\sphinxhyphen{}M31.
\index{Vdisp (class in galaxy.remvdisp)@\spxentry{Vdisp}\spxextra{class in galaxy.remvdisp}}

\begin{fulllineitems}
\phantomsection\label{\detokenize{remvdisp:galaxy.remvdisp.Vdisp}}\pysiglinewithargsret{\sphinxbfcode{\sphinxupquote{class }}\sphinxcode{\sphinxupquote{galaxy.remvdisp.}}\sphinxbfcode{\sphinxupquote{Vdisp}}}{\emph{snap=801}, \emph{ptype=(2}, \emph{3)}, \emph{r\_lim=60}}{}
A class to work with rotation and velocity dispersion of the MW\sphinxhyphen{}M31 
post\sphinxhyphen{}merger remant.
\begin{description}
\item[{Args:}] \leavevmode\begin{description}
\item[{snap (int):}] \leavevmode
Snap number, equivalent to time elapsed. 
Defaults to the last timepoint.

\item[{usesql (bool):}] \leavevmode
If True, data will be taken from a PostgreSQL database instead of
text files.

\item[{ptype (int or iterable of int):}] \leavevmode
Particle type: 1, 2, 3 or a combination of these

\end{description}

\end{description}
\index{calc\_centered() (galaxy.remvdisp.Vdisp method)@\spxentry{calc\_centered()}\spxextra{galaxy.remvdisp.Vdisp method}}

\begin{fulllineitems}
\phantomsection\label{\detokenize{remvdisp:galaxy.remvdisp.Vdisp.calc_centered}}\pysiglinewithargsret{\sphinxbfcode{\sphinxupquote{calc\_centered}}}{}{}
Sets the CoM position and velocity, plus particle coordinates centered on the CoM.

No args, no return value.

\end{fulllineitems}

\index{rotate() (galaxy.remvdisp.Vdisp method)@\spxentry{rotate()}\spxextra{galaxy.remvdisp.Vdisp method}}

\begin{fulllineitems}
\phantomsection\label{\detokenize{remvdisp:galaxy.remvdisp.Vdisp.rotate}}\pysiglinewithargsret{\sphinxbfcode{\sphinxupquote{rotate}}}{\emph{r\_lim}}{}
Creates transformed coordinates with the angular momentum vector along z.
\begin{description}
\item[{Arg:}] \leavevmode\begin{description}
\item[{r\_lim (float):}] \leavevmode
Only consider particles within this radius when computing L\sphinxhyphen{}hat. 
Implicit kpc, no units.

\end{description}

\end{description}

\end{fulllineitems}

\index{calc\_v\_sigma() (galaxy.remvdisp.Vdisp method)@\spxentry{calc\_v\_sigma()}\spxextra{galaxy.remvdisp.Vdisp method}}

\begin{fulllineitems}
\phantomsection\label{\detokenize{remvdisp:galaxy.remvdisp.Vdisp.calc_v_sigma}}\pysiglinewithargsret{\sphinxbfcode{\sphinxupquote{calc\_v\_sigma}}}{\emph{pos\_index}, \emph{v\_index}}{}
Calculate mean radial velocities and dispersions for bins along an axis.
\begin{description}
\item[{Args:}] \leavevmode\begin{description}
\item[{pos\_index, v\_index (integers in (0,1,2)):}] \leavevmode
Axis numbers for binning and for radial velocity.
x=0, y=1, z=2

\end{description}

\item[{Returns:}] \leavevmode
Binned v\_radial and dispersions, v\_max and central dispersion. 
All implicit km/s, no units.

\end{description}

\end{fulllineitems}

\index{set\_xbins() (galaxy.remvdisp.Vdisp method)@\spxentry{set\_xbins()}\spxextra{galaxy.remvdisp.Vdisp method}}

\begin{fulllineitems}
\phantomsection\label{\detokenize{remvdisp:galaxy.remvdisp.Vdisp.set_xbins}}\pysiglinewithargsret{\sphinxbfcode{\sphinxupquote{set\_xbins}}}{\emph{xbins}}{}
Sets new x\sphinxhyphen{}boundaries for binning, invalidates any previous calculations.
\begin{description}
\item[{Args:}] \leavevmode\begin{description}
\item[{xbins (array of float):}] \leavevmode
Distances from CoM along chosen axis (signed, not just radius)

\end{description}

\end{description}

\end{fulllineitems}

\index{set\_yx() (galaxy.remvdisp.Vdisp method)@\spxentry{set\_yx()}\spxextra{galaxy.remvdisp.Vdisp method}}

\begin{fulllineitems}
\phantomsection\label{\detokenize{remvdisp:galaxy.remvdisp.Vdisp.set_yx}}\pysiglinewithargsret{\sphinxbfcode{\sphinxupquote{set\_yx}}}{}{}
Convenience method to call calc\_v\_sigma() with y\sphinxhyphen{}axis and x\sphinxhyphen{}velocities

\end{fulllineitems}

\index{set\_zx() (galaxy.remvdisp.Vdisp method)@\spxentry{set\_zx()}\spxextra{galaxy.remvdisp.Vdisp method}}

\begin{fulllineitems}
\phantomsection\label{\detokenize{remvdisp:galaxy.remvdisp.Vdisp.set_zx}}\pysiglinewithargsret{\sphinxbfcode{\sphinxupquote{set\_zx}}}{}{}
Convenience method to call calc\_v\_sigma() with z\sphinxhyphen{}axis and x\sphinxhyphen{}velocities

\end{fulllineitems}

\index{plot\_yx() (galaxy.remvdisp.Vdisp method)@\spxentry{plot\_yx()}\spxextra{galaxy.remvdisp.Vdisp method}}

\begin{fulllineitems}
\phantomsection\label{\detokenize{remvdisp:galaxy.remvdisp.Vdisp.plot_yx}}\pysiglinewithargsret{\sphinxbfcode{\sphinxupquote{plot\_yx}}}{\emph{particles=\textquotesingle{}Stellar\textquotesingle{}}, \emph{xlim=(\sphinxhyphen{}40}, \emph{40)}, \emph{ylim1=(\sphinxhyphen{}120}, \emph{120)}, \emph{ylim2=(0}, \emph{200)}, \emph{xbins=None}, \emph{pngout=False}, \emph{fname=None}}{}
Wrapper for Plots.plot\_v\sphinxhyphen{}sigma()

\end{fulllineitems}

\index{plot\_zx() (galaxy.remvdisp.Vdisp method)@\spxentry{plot\_zx()}\spxextra{galaxy.remvdisp.Vdisp method}}

\begin{fulllineitems}
\phantomsection\label{\detokenize{remvdisp:galaxy.remvdisp.Vdisp.plot_zx}}\pysiglinewithargsret{\sphinxbfcode{\sphinxupquote{plot\_zx}}}{\emph{particles=\textquotesingle{}Stellar\textquotesingle{}}, \emph{xlim=(\sphinxhyphen{}40}, \emph{40)}, \emph{ylim1=(\sphinxhyphen{}120}, \emph{120)}, \emph{ylim2=(0}, \emph{200)}, \emph{xbins=None}, \emph{pngout=False}, \emph{fname=None}}{}
Wrapper for Plots.plot\_v\sphinxhyphen{}sigma()

\end{fulllineitems}


\end{fulllineitems}



\chapter{Indices and tables}
\label{\detokenize{index:indices-and-tables}}\begin{itemize}
\item {} 
\DUrole{xref,std,std-ref}{genindex}

\item {} 
\DUrole{xref,std,std-ref}{modindex}

\item {} 
\DUrole{xref,std,std-ref}{search}

\end{itemize}


\renewcommand{\indexname}{Python Module Index}
\begin{sphinxtheindex}
\let\bigletter\sphinxstyleindexlettergroup
\bigletter{g}
\item\relax\sphinxstyleindexentry{galaxy.approaches}\sphinxstyleindexpageref{approaches:\detokenize{module-galaxy.approaches}}
\item\relax\sphinxstyleindexentry{galaxy.centerofmass}\sphinxstyleindexpageref{centerofmass:\detokenize{module-galaxy.centerofmass}}
\item\relax\sphinxstyleindexentry{galaxy.db}\sphinxstyleindexpageref{db:\detokenize{module-galaxy.db}}
\item\relax\sphinxstyleindexentry{galaxy.galaxies}\sphinxstyleindexpageref{galaxies:\detokenize{module-galaxy.galaxies}}
\item\relax\sphinxstyleindexentry{galaxy.galaxy}\sphinxstyleindexpageref{galaxy:\detokenize{module-galaxy.galaxy}}
\item\relax\sphinxstyleindexentry{galaxy.massprofile}\sphinxstyleindexpageref{massprofile:\detokenize{module-galaxy.massprofile}}
\item\relax\sphinxstyleindexentry{galaxy.remnant}\sphinxstyleindexpageref{remnant:\detokenize{module-galaxy.remnant}}
\item\relax\sphinxstyleindexentry{galaxy.remvdisp}\sphinxstyleindexpageref{remvdisp:\detokenize{module-galaxy.remvdisp}}
\item\relax\sphinxstyleindexentry{galaxy.surfacedensity}\sphinxstyleindexpageref{surfacedensity:\detokenize{module-galaxy.surfacedensity}}
\item\relax\sphinxstyleindexentry{galaxy.timecourse}\sphinxstyleindexpageref{timecourse:\detokenize{module-galaxy.timecourse}}
\item\relax\sphinxstyleindexentry{galaxy.utilities}\sphinxstyleindexpageref{utilities:\detokenize{module-galaxy.utilities}}
\end{sphinxtheindex}

\renewcommand{\indexname}{Index}
\printindex
\end{document}